\chapter{Background} \label{ch:background}

\section{Submodularity} \label{sect:bg_submod}

Modeling notions such as coverage, representativeness, or diversity is an important challenge in many machine learning problems.
These notions are well captured by submodular set functions.
Analogously, supermodular functions capture notions of smoothness, regularity, or cooperation. 
As a result, submodularity and supermodularity have found numerous applications in machine learning problems of discrete nature, akin to concavity and convexity for continuous optimization.

\subsection{Basics}
We consider set functions $F : 2^V \to \mathbb{R}$, where $V$ is a finite ground set of size $|V| = n$.
Without loss of generality, if not otherwise stated, we will hereafter assume that $V = [n] \defeq \{1, 2, \ldots,n\}$.
Adding an element $i$ to a set $S$ results in a difference in the value of $F$ that is called marginal gain, and is defined as follows.
\begin{definition}[Marginal gain]
For any $i \in V$, and $S \subseteq V$, the marginal gain of adding $i$ to $S$ is
\begin{align*}
F(i \mid S) \defeq F(S \cup \{i\}) - F(S).
\end{align*}
\end{definition}

Intuitively, submodularity expresses a notion of diminishing returns; that is, adding an element to a larger set provides less benefit than adding it to a smaller one.
\begin{definition}[Submodularity]
$F$ is submodular if, for any $S \subseteq T \subseteq V$, and any $v \in V \setminus T$, it holds that
\begin{align*}
F(v\mid T) \leq F(v\mid S).
\end{align*}
\end{definition}
\noindent The following is an equivalent definition of submodularity that will also be useful later in the thesis.
\begin{definition}[Submodularity] \label{def:submod}
$F$ if submodular if, for any $A, B \subseteq V$, it holds that
\begin{align*}
F(A \cup B) + F(A \cap B) \leq F(A) + F(B).
\end{align*}
\end{definition}

Supermodularity is defined analogously by reversing the sign of the above inequalities.
\begin{definition}[Supermodularity] \label{def:supermod}
A function $F$ is supermodular if and only if $-F$ is submodular.
\end{definition}

If a function $m$ is both submodular and supermodular, then it is called modular.
Modular functions can be seen as the discrete analogue of linear continuous functions, and can be defined using a sum over real-numbered ``utilities''.
\begin{definition}[Modularity]
A function $m$ is called modular if it is both submodular and supermodular; it can be written as
\begin{align*}
F(S) = c + \sum_{i \in S} m_i,
\end{align*}
where $c \in \mathbb{R}$, and $m_i \in \mathbb{R}$, for all $i \in V$.
\end{definition}

A function is called monotone when adding an element can never decrease its value.
\begin{definition}[Monotonicity]
A function $F$ is monotone if, for any $i \in V$, and $S \subseteq V$, it holds that
\begin{align*}
F(i \mid S) \geq 0.
\end{align*}
\end{definition}
Furthermore, a function $F$ is called normalized if $F(\varnothing) = 0$.
In some of our results we will use the fact that we can separate the non-normalized, and non-monotone parts of any submodular function according to the following decomposition.
\begin{definition}[Submodular decomposition] \label{def:decomp}
Any submodular function $F$ can be decomposed as
\begin{align} \label{eq:decomp}
  F(S) = c + m(S) + f(S),
\end{align}
for all $S \subseteq V$, where $c \in \mathbb{R}$ is a constant, $m$ is a normalized modular function, and $f$ is a normalized monotone submodular function.
\end{definition}
An analogous decomposition using a monotone supermodular function $f$ is possible for any supermodular function $F$ as well.

\subsection{Submodular Maximization}
Perhaps the most celebrated result pertaining to submodular functions is the approximation guarantee for maximizing a monotone submodular function under a cardinality constraint.
Although the maximization problem itself is NP-hard, \cite{nemhauser78} showed that the simple greedy \algoref{alg:greedy}, which repeatedly adds the element with the maximum marginal gain, identifies a solution that is within a factor of $1 - 1/e$ of the optimal value.
\begin{theorem}[\hspace{1sp}\citealp{nemhauser78}]
For any normalized monotone submodular function $F$, the solution $S^*$ returned by \algoref{alg:greedy} satisfies
\begin{align*}
F(S^*) \geq \left(1 - \frac{1}{e}\right) \max_{S \subseteq V, |S| \leq k} F(S).
\end{align*}
\end{theorem}

\begin{algorithm}[tb]
  \setstretch{1.3}
  \DontPrintSemicolon
  \caption{\strut Greedy submodular maximization}
  \label{alg:greedy}
  \vspace{0.5em}
  \SetKwInOut{Input}{Input}
  \Input{Set function $F$, cardinality constraint $k$}
  $S^*$ $\gets$ $\varnothing$\;
  \For{$j = 1$ \KwTo $k$}{
  Select $i^*$ $\in$ $\argmax_{i \in V \setminus S^*} F(i \mid S^*)$\;
  $S^*$ $\gets$ $S^* \cup \{i^*\}$\;
  }
  \Return{$S$}\;
\end{algorithm}

Numerous extensions and generalizations of this result have been studied, including approximation guarantees for the non-monotone setting \citep{feige11,buchbinder14}; for different kinds of constraints, such as matroid \citep{lee09,calinescu11} and knapsack \citep{chekuri11}; and for the adaptive setting \citep{golovin11,gotovos15}.


\section{Discrete Probabilistic Models, Inference, and Learning}
As stated in the introduction, in the interest of venturing beyond discrete optimization, we consider discrete probabilistic models, that is, distributions over finite subsets of the ground set $V$ defined as
\begin{align*}
p(S; \btheta) = \frac{1}{Z(\btheta)} \exp\left( F(S; \btheta) \right),
\end{align*}
for all $S \subseteq V$.
The function $F$ is parameterized by a (possibly to be learned) vector $\btheta$, and $Z(\btheta)$ denotes the normalizing constant of the distribution, which is also often referred to as the partition function, and defined as
\begin{align*}
Z(\btheta) \defeq \sum_{S \subseteq V} \exp\left( F(S; \btheta) \right).
\end{align*}
An alternative and equivalent way of defining distributions of the above form is via binary random vectors $X \in \{0, 1\}^n$.
If we define $V(X) \defeq \sdef{v \in V}{X_v = 1}$, it is easy to see that the distribution $p_X(X) \propto \exp(\beta F(V(X)))$ over binary vectors is isomorphic to the above distribution over sets.
With a slight abuse of notation, we will use $F(X)$ to denote $F(V(X))$, and use $p$ to refer to both distributions.

For large parts of this thesis, we will focus on such distributions with $F$ being submodular or supermodular.
\begin{definition}[Probabilistic submodular model]
A probabilistic submodular model \citep{djolonga14,gotovos15} is a distribution of the form
\begin{align*}
p(S; \btheta) \propto \exp\left( F(S; \btheta) \right),
\end{align*}
for all $S \subseteq V$, where $F$ is a submodular or supermodular function.
\end{definition}
The resulting models of this form are also referred to as log-submodular and log-supermodular distributions respectively.
Note that the most likely configurations in these distributions directly correspond to the maximizers of the sub- or supermodular function $F$.
Some commonly used discrete models fall under these categories; for example, the standard Ising and Potts models are log-supermodular, while determinantal point processes are log-submodular.
We now present some examples models in more detail.

\begin{example}[Product distribution]
Product or log-modular distributions describe a collection of $n$ independent binary random variables.
The corresponding function $F$ is modular, that is, $F(S) = c + \sum_{i \in S} m_i$, and the partition function can be derived in closed form as
\begin{align*}
Z = \exp(c) \prod_{i \in V} \left( 1 + \exp(m_i) \right).
\end{align*}
Consequently, a log-modular distribution can be written as
\begin{align*}
  p(S) = \frac{\exp\big( \sum_{i \in S} m_i \big)}{\prod_{i \in V} \left( 1 + \exp(m_i) \right)}.
\end{align*}
Note that the constant $c$ does not appear in the distribution.
More generally, the discrete models we consider are invariant to adding a constant to $F$, since that constant gets cancelled by the partition function $Z$.
\end{example}

\begin{example}[Ising model]
In its simplest form, the (ferromagnetic) Ising model \citep{ising} is defined via an undirected graph $(V, E)$, and a set of ``attractive'' pairwise potentials
\begin{align*}
\sigma_{i,j}(S) \defeq 4\left(\llbracket\{i \in S\}\rrbracket - 0.5\right)(\llbracket\{j \in S\}\rrbracket - 0.5),
\end{align*}
for all $\{i, j\} \in E$, where we use $\llbracket \cdot \rrbracket$ to denote the Iverson bracket, which has value $1$ when the enclosed condition is true, and $0$ otherwise.
We can see that $\sigma_{i,j}$ takes value $1$ if $S$ contains both or neither of $i, j$, and value $-1$ if it contains only one of $i$ or $j$.
It follows that $\sigma_{i, j}$ is a supermodular set function.
The Ising distribution is defined as
\begin{align*}
p(S) \propto \exp\left(\sum_{\{i,j\} \in E} \sigma_{i,j}(S)\right).
\end{align*}
It is log-supermodular, since each $\sigma_{i,j}$ is supermodular, and supermodular functions are closed under addition.

We can also define the anti-ferromagnetic Ising model by a different set of ``repulsive'' pairwise potentials $\hat{\sigma}_{i,j}(S) \defeq \sigma_{i,j}(S)$.
In this case, each $\hat{\sigma}_{i,j}$ is a submodular set function, and the resulting distribution is log-submodular.

Ising models, and Potts models \citep{potts}, which generalize Ising models from binary to $k$-state variables, originate in statistical physics, but have also found numerous applications in computer vision \citep{wang13}.
\end{example}

\begin{example}[Determinantal point process]
A determinantal point process \citep{lyons03,kulesza12} is defined via a positive semidefinite matrix $L \in \mathbb{R}^{n \times n}$, and has a distribution of the form
\begin{align*}
p(S) = \frac{\det(L_S)}{\det(L + I)},
\end{align*}
where $L_S$ denotes the square submatrix indexed by set $S$, and $I$ is the $n \times n$ identity matrix.
(We only describe here the form known as an $L$-ensemble.)
Since $F(S) = \log \det(L_S)$ is a submodular function, determinantal point processes (DPPs) are log-submodular distributions.
Interestingly, as we can see from the above equation, the partition function $Z = \det(L + I)$ can be easily computed, which makes DPPs one of very few known tractable higher-order models.

DPPs also originate in statistical physics, but have been used to encourage diversity in various machine learning applications, such as image and video summarization \citep{kulesza12,gong14}.
\end{example}

\begin{example}[\flid{}]
\cite{tschiatschek16} defined the class of facility location diversity (\flid) models by means of facility location functions, that is, functions of the form
\begin{align*}
F(S) = \sum_{i \in S} u_i + \sum_{j=1}^{L} \left(\max_{i \in S} w_{ij} - \sum_{i \in S} w_{ij}\right),
\end{align*}
where $w_{ij} \geq 0$.
This is a submodular set function, therefore the resulting distribution $p(S) \propto \exp(F(S))$ is log-submodular.

The above function $F$ is parameterized by a utility vector $\bu \in \mathbb{R}^n$, and a diversity matrix $\bw \in \mathbb{R}^{n\times L}$.
Increasing the utility $u_i$ of an element $i \in S$ intuitively increases the probability of all sets containing that element, therefore also increases its marginal probability.
The diversity matrix $\bw$ can be thought of as consisting of $L$ dimensions (columns).
Elements of the ground set that have large value in the same column $j$ will tend to appear together less frequently, since the term $\max_{i \in S} w_{ij} - \sum_{i \in S} w_{ij}$ will be negative for sets $S$ that contain combinations of such items.

\todo{Show 3-element example figure.}
\end{example}

\begin{example}[\fldc{}]
\citep{djolonga16mixed}
\begin{align*}
F(S) = \sum_{i \in S} u_i + \sum_{j=1}^{L} \left(\max_{i \in S} w_{ij} - \sum_{i \in S} w_{ij}\right) - \sum_{j=1}^{L} \left(\max_{i \in S} v_{ij} - \sum_{i \in S} v_{ij}\right).
\end{align*}
\end{example}

Note that, both the facility location model and the Ising model use decomposable functions, that is, functions that can be written as a sum of simpler submodular (resp. supermodular) functions $F_{\ell}$:
\begin{align} \label{eq:fdec}
F(S) = \sum_{\ell \in [L]} F_{\ell}(S).
\end{align}


\subsection{Inference}

Recently, \citet{djolonga14} considered a more general treatment of such models, and proposed a variational approach for performing approximate probabilistic inference for them.

\todo{Exponential family}

\todo{Contrastive divergence}

Iyer and Bilmes \citep{iyer15} recently considered a different class of probabilistic models, called submodular point processes, which are also defined through submodular functions, and have the form $p(S) \propto F(S)$.
They showed that inference in SPPs is, in general, also a hard problem, and provided approximations and closed-form solutions for some subclasses.

\todo{Add Chengtao constrained etc. paper}
\todo{Add DPP -- Rayleigh papers}

\section{Sampling and Mixing Times}

\paragraph{Gibbs sampler.}
One of the most commonly used chains is the (single-site) Gibbs sampler, which adds or removes a single element %of the ground set
at a time.
It first selects uniformly at random an element $v \in V$; subsequently, it adds or removes $v$ to the current state $X_t$ according to the probability of the resulting state.
We denote by $P : \Omega \times \Omega \to \mathbb{R}$ the transition matrix of a Markov chain, that is, for all $S, R \in \Omega$, $P(S, R) \defeq \P\left[ X_{t+1} = R \mid X_t = S \right]$.
Then, if we define
\begin{align*}
p_{S \rightarrow R} = \displaystyle\frac{\exp(F(R))}{\exp(F(R)) + \exp(F(S))},
\end{align*}
and denote by $S \sim R$ states that differ by exactly one element (i.e., $\big||R| - |S|\big| = 1$),
the transition matrix $\Pg$ of the Gibbs sampler is
\begin{align*}
  \Pg(S, R) = 
  \threepartdefo{\displaystyle\frac{1}{n}p_{S \rightarrow R}}{R \sim S}{1 - \displaystyle\sum_{T \sim S} \displaystyle\frac{1}{n}p_{S \rightarrow T}}{R = S}{0}.
\end{align*}

\paragraph{Approximating the log-partition function.}
There are two straightforward methods for estimating the log-partition function using sampling.
The first one, importance sampling (IS) \citep{ais}, assumes that we have a normalized distribution $\pi$ from which we draw $M$ samples $\{x\}$

The second, reverse important sampling (RIS) \citep{ris},

\paragraph{Mixing times.}
Approximating quantities of interest using MCMC methods is based on using time averages to estimate expectations over the desired distribution.
In particular, we estimate the expected value of function $f : \ss \to \mathbb{R}$ by $\E_p[f(X)] \approx (1/T)\sum_{r=1}^{T} f(X_{s + r})$.
For example, to estimate the marginal $p(v \in S)$, for some $v \in V$, we would define $f(x) = \mathds{1}_{\{x_v = 1\}}$, for all $x \in \ss$.
The choice of burn-in time $s$ and number of samples $T$ in the above expression presents a tradeoff between computational efficiency and approximation accuracy.
It turns out that the effect of both $s$ and $T$ is largely dependent on a fundamental quantity of the chain called \emph{mixing time} \cite{levin08}.

The mixing time of a chain quantifies the number of iterations $t$ required for the distribution of $X_t$ to be close to the stationary distribution $\pi$.
More formally, it is defined as $\tme \defeq \min \sdef{t}{d(t) \leq \epsilon}$, where $d(t)$ denotes the worst-case (over the starting state $X_0$ of the chain) total variation distance between the distribution of $X_t$ and $\pi$.
Establishing upper bounds on the mixing time of our Gibbs sampler is, therefore, sufficient to guarantee efficient approximate marginal inference (e.g., see Theorem 12.19 of \citet{levin08}).