\chapter{Improved Sampling using Semigradients} \label{ch:m3}

\emph{The majority of the content of this chapter has already been published in conference proceedings \citep{gotovos18}.}

\section{Introduction}
Discrete probabilistic models have played a fundamental role in machine learning.
Examples range from classic graphical models, such as Ising and Potts models \citep{koller09}, which have long been used in computer vision applications \citep{boykov01}, to determinantal point processes \citep{kulesza12} used in video summarization \citep{gong14}, and facility location diversity models used for product recommentation \citep{tschiatschek16}.
Recently, there has been increased interest in general distributions over subsets of a finite ground set $V$; that is, given a set function $F : 2^V \to \mathbb{R}$, distributions of the form $\pi(S) \propto \exp(F(S))$, for all $S \subseteq V$.
These can be equivalently seen as distributions over binary random vectors, if $S$ is replaced by the indicator function of the corresponding vector.
All the aforementioned examples can be expressed in this form for a suitable choice of $F$.

While exact inference in such models is known to be intractable in general \citep{jerrum93}, there has been recent work on analyzing approximate inference techniques, such as variational methods \citep{djolonga14, djolonga16mixed}, and Markov chain Monte Carlo (MCMC) sampling \citep{gotovos15, rebeschini15}.
The sampling analyses, in particular, focus on the Gibbs sampler, and derive sufficient conditions under which it mixes---converges toward the target distribution---sufficiently fast.

Unfortunately, oftentimes in practice these conditions do not hold and the Gibbs sampler mixes prohibitively slowly.
A fundamental reason for this slow mixing behavior is the existence of bottlenecks in the state space of the Markov chain.
Conceptually, one can think about the state-space graph containing several isolated components that are poorly connected to each other, thus making it hard for the Gibbs sampler to move between them.

In this work, we propose a novel sampling strategy that allows for global moves in the state space, thereby avoiding bottlenecks, and, thus, accelerating mixing.
Our sampler is based on using a proposal distribution that approximates the target $\pi$ by a mixture of product distributions.
We further propose an algorithm for constructing such a mixture using discrete semigradient information of the associated function $F$.
This idea makes a step towards bridging optimization and sampling, a theme that has been successful in continuous spaces.
Our sampler is readily combined with other existing samplers, and we show provable theoretical, as well as empirical examples of speedups.

\paragraph{Contributions.}
The main contributions of this paper are as follows.
\begin{itemize}
\item We propose the \Ms{} sampler, which makes global moves according to a specific mixture of product distributions.
\item We theoretically analyze mixing times on an illustrative family of Ising models, and prove that adding the \Ms{} sampler results in an exponential improvement over the Gibbs sampler.
\item We demonstrate the effectiveness of combining the \Ms{} and Gibbs samplers in practice on three models learned from real-world data.
\end{itemize}

\paragraph{Related work.}
Recent work on analyzing the mixing time of MCMC samplers for discrete probabilistic models includes deriving general conditions on $F$ to achieve fast mixing \citep{gotovos15, rebeschini15, li16}, as well as looking at specific subclasses, such as strongly Rayleigh distributions \citep{li16, anari16}.

There has also been work on mapping discrete inference to continuous domains \citep{zhang12, pakman13, dinh17, nishimura18} to enable the use of well-established continuous samplers, such as Hamiltonian Monte Carlo \citep{neal12, betancourt17}.
It is worth pointing out that, while these methods usually outperform simple Gibbs or Metropolis samplers, they still tend to suffer from considerable slowdowns in multimodal distributions \citep{neal12}.
Our work is orthogonal to these methods, in the sense that our proposed sampler can be combined with any of the existing ones to provide a principled way for performing global moves that can lead to improved mixing.

Both darting Monte Carlo \citep{sminchisescu07,ahn13} and variational MCMC \citep{defreitas01} share the high-level concept of combining two chains, one making global moves between high-probability regions, and another making local moves around those regions.
However, their proposed global samplers for continuous spaces are generally not applicable to the class of discrete distributions we consider.

There are several well-known results on mixing of the Gibbs sampler for the Ising model on different graph structures \citep{jerrum93,berger05,levin08,levin08book}.
Other (non-MCMC) approaches to discrete sampling include Perturb-and-MAP \citep{papandreou11,hazan13}, and random projections \citep{zhu15}.
Semigradients of submodular set functions have recently been exploited for optimization \citep{iyer13, jegelka11} and variational inference \citep{djolonga16}, but, to our knowledge, no prior work  has used them for sampling.


%\section{Background}
%We consider set functions $F : 2^V \to \mathbb{R}$, where $V$ is a finite ground set of size $n$ that can be assumed to be $V = \{1, \ldots, n\}$ without loss of generality.
%In this paper, we focus on distributions over $\Omega \defeq 2^V$ of the form
%\begin{align} \label{eq:pdef}
%  \pi(S) = \frac{1}{Z} \exp\left( F(S) \right),
%\end{align}
%for all $S \in \Omega$.
%The partition function $Z \defeq \sum_{S \in \Omega} \exp(F(S))$ serves as the normalizer of the distribution.
%Alternatively, we can describe distributions of the above form via binary vectors $X \in \{0, 1\}^n$.
%If we define $V(X) \defeq \{v \in V \mid X_v = 1\}$, then the distribution $p_X(X) \propto \exp(F(V(X)))$ over binary vectors is isomorphic to the distribution \eqref{eq:pdef} over sets.
%
%Perhaps the simplest family of such models are log-modular distributions, which describe a collection of independent binary random variables.
%Equivalently, they are distributions of the form \eqref{eq:pdef} where $F$ is a modular function, that is, a function of the form $F(S) = c + \sum_{v \in S}m_v$, where $c, m_v \in \mathbb{R}$, for all $v \in V$.
%The partition function of a log-modular distribution can be derived in closed form as $Z_m = \exp(c) \prod_{v \in V} \left( 1 + \exp(m_v) \right)$.
%Consequently, the corresponding log-modular distribution is
%\begin{align*}
%  \pi_m(S) = \frac{\exp\big( \sum_{v \in S} m_v \big)}{\prod_{v \in V} \left( 1 + \exp(m_v) \right)}.
%\end{align*}
%
%\paragraph{Inference and sampling.}
%Performing exact inference in models of the form \eqref{eq:pdef}, that is, computing conditional probabilities such as $\pi(A \subseteq S \subseteq B \mid C \subseteq S \subseteq D)$, is known to be in general \#P-hard \citep{jerrum93}.
%As a result, we have to resort to approximate inference algorithms, such as Markov chain Monte Carlo sampling \citep{levin08book}, which is the primary focus of this paper.
%An MCMC algorithm for distribution $\pi$ simulates a Markov chain in state space $\Omega$ in such a way that the sequence of visited states $(X_0, X_1, \ldots) \in \Omega^{\mathbb{N}}$ ultimately converges to $\pi$.
%
%\paragraph{Gibbs sampler.}
%One of the most commonly used chains is the (single-site) Gibbs sampler, which adds or removes a single element %of the ground set
%at a time.
%It first selects uniformly at random an element $v \in V$; subsequently, it adds or removes $v$ to the current state $X_t$ according to the probability of the resulting state.
%We denote by $P : \Omega \times \Omega \to \mathbb{R}$ the transition matrix of a Markov chain, that is, for all $S, R \in \Omega$, $P(S, R) \defeq \P\left[ X_{t+1} = R \mid X_t = S \right]$.
%Then, if we define
%\begin{align*}
%p_{S \rightarrow R} = \displaystyle\frac{\exp(F(R))}{\exp(F(R)) + \exp(F(S))},
%\end{align*}
%and denote by $S \sim R$ states that differ by exactly one element (i.e., $\big||R| - |S|\big| = 1$),
%the transition matrix $\Pg$ of the Gibbs sampler is
%\begin{align*}
%  \Pg(S, R) = 
%  \threepartdefo{\displaystyle\frac{1}{n}p_{S \rightarrow R}}{R \sim S}{1 - \displaystyle\sum_{T \sim S} \displaystyle\frac{1}{n}p_{S \rightarrow T}}{R = S}{0}.
%\end{align*}
%
%\paragraph{Mixing.}
%The efficiency of a Markov chain in approximating its target distribution depends largely on the speed of convergence of the chain, which is quantified by the chain's mixing time.
%Most commonly, distance from stationarity is measured by the maximum total variation distance, over all starting states, between $X_t$ and the target distribution $\pi$, that is, $d(t) \defeq \max_{X_0 \in \Omega} \dtv{P^t(X_0, \cdot)}{\pi}$.
%Then, the mixing time denotes the minimum number of iterations required to get $\epsilon$-close to stationarity, $\tme \defeq \min \{ t \mid d(t) \leq \epsilon \}$.
%
%A common way to obtain an upper bound on the mixing time of a chain is by lower bounding its spectral gap, defined as $\gamma \defeq 1 - \lambda_2$, where $\lambda_2$ is the second largest eigenvalue of the transition matrix $P$.
%The following well-known theorem connects the spectral gap to mixing time.
%\begin{theorem}[cf. Theorems 12.3, 12.4 in \citep{levin08book}] \label{thm:spectral}
%  Let $P$ be the transition matrix of a lazy, irreducible, and reversible Markov chain, and let $\gamma$ be its spectral gap, and $\pmin \defeq \min_{S \in \Omega} \pi(S)$. Then,
%  \begin{align*}
%    \left( \frac{1}{\gamma} - 1 \right)\log\left( \frac{1}{2\epsilon} \right) \leq \tme \leq \frac{1}{\gamma} \log\left( \frac{1}{\epsilon\pmin} \right).
%  \end{align*}
%\end{theorem}

\section{The Mixture Chain}
Despite the simplicity and computational efficiency of the Gibbs sampler, the fact that it is constrained to performing local moves makes it susceptible to state-space bottlenecks, which hinder the movement of the chain around the state space.
Intuitively, the state space may contain several high-probability regions arranged in such a way that moving from one to another using only single-element additions and deletions requires passing through states of very low probability.
As a result, the Gibbs sampler may mix extremely slowly on the whole state space, despite the fact that it can move sufficiently fast within each of the high-probability regions.

To alleviate this shortcoming, it is natural to ask whether it is possible to bypass such bottlenecks by using a chain that performs larger moves.
In this paper, we introduce a novel approach that uses a Metropolis chain based on a specific mixture of log-modular distributions, which we call the \Ms{} chain, to perform global moves in state space.
Concretely, we define a proposal distribution
\begin{align} \label{eq:qprop}
  q(S, R) = q(R) &= \frac{1}{Z_q} \sum_{i = 1}^{r} \exp\left( \Fi(R) \right) \nonumber\\
                 &= \frac{1}{Z_q} \sum_{i = 1}^{r} \wi \exp\left(\mi(R) \right),
\end{align}
where each $\Fi(R) = \ci + \sum_{v \in R}m_{iv}$ is a modular function, while each $\mi(R) = \sum_{v \in R}m_{iv}$ is a normalized modular function ($\mi(\emptyset) = 0$), and $\wi = \exp(\ci) > 0$.
If we denote by $\Zi$ the normalizer of $\mi$, then the normalizer of the mixture can be written in closed form as
\begin{align*}
  Z_q = \sum_{R \in \Omega}q(R) &= \sum_{R \in \Omega}\sum_{i = 1}^{r} \wi \exp\left(\mi(R) \right)\\
                                &= \sum_{i = 1}^{r} \wi \sum_{R \in \Omega} \exp\left(\mi(R) \right)\\
                                &= \sum_{i = 1}^r \wi \Zi.
\end{align*}
We define the \Ms{} chain as a Metropolis chain \citep{levin08book} using $q$ as a proposal distribution; its transition matrix $\Pm : \Omega \times \Omega \to \mathbb{R}$ is given by
\begin{align*}
  \Pm(S, R) = \twopartdefo{q(R) p_a(S, R)}{R \neq S}{1 - \displaystyle\sum_{T \neq S} q(T) p_a(S, T)},
\end{align*}
where
\begin{align*}
  p_{a}(S, R) \defeq \min\left\{1, \displaystyle\frac{\pi(R)q(S)}{\pi(S)q(R)}\right\}.
\end{align*}

Note that, contrary to usual practice, the proposal $q$ only depends on the proposed state, but not on the current state of the chain.
As a result, the chain is not constrained to local moves, but rather can potentially jump to any part of the state space.
In practice, \Ms{} sampling proceeds in two steps: first, a candidate set $R$ is sampled according to $q$; then, the move to $R$ is accepted with probability $p_a$.
Sampling from $q$ can be done in $\bO(n)$ time---first, sample a log-modular component, then sample a set from that component.
Computing $p_a$ requires $\bO(r)$ time for the sum in \eqref{eq:qprop}, and it can be straightforwardly improved by parallelizing this computation.
All in all, the total time for one step of \Ms{} is $\bO(n + r)$.

As is always the case with Metropolis chains, the mixing time of the \Ms{} sampler will depend on how well the proposal $q$ approximates the target distribution $\pi$.
The following observation shows that, in theory, we can approximate any distribution of the form \eqref{eq:pdef} by a mixture of the form \eqref{eq:qprop}.

\begin{prop} \label{prop:decomp}
  For any $\pi$ on $\Omega$ as in \eqref{eq:pdef}, and any $\epsilon > 0$, there are positive constants $\wi = \wi(\epsilon) > 0$, and normalized modular functions $\mi = \mi(\epsilon)$, such that, if we define $q(S) \defeq \sum_{i = 1}^r \wi \exp(\mi(S))$, for all $S \in \Omega$, then $\dtv{\pi}{q} \leq \epsilon$.
\end{prop}
Conceptually, the proof relies on having one log-modular term per set in $\Omega$.\footnote{Detailed proofs of all our results can be found in the appendix.}
Therefore, while the above result shows that mixtures of log-modulars are expressive enough, the constructed mixture of exponential size in $n$ is not useful for practical purposes.
On the other hand, it is not necessary for us to have $q$ be an accurate approximation of $\pi$ everywhere, as long as the corresponding \Ms{} chain is able to bypass state-space bottlenecks.
With this in mind, we suggest combining the \Ms{} and Gibbs chains, so that each of them serve complementary purposes in the final chain; the role of \Ms{} is to make global moves and avoid bottlenecks, while the role of Gibbs is to move fast within well-connected regions of the state space.
To make this happen, we define the transition matrix $\Pc : \Omega \times \Omega \to \mathbb{R}$ of the combined chain as
\begin{align} \label{eq:cdef}
  \Pc(S, R) = \alpha\Pg(S, R) + (1-\alpha)\Pm(S, R),
\end{align}
where $0 < \alpha < 1$.
\todo {Use some other symbol instead of $\alpha$.}
It is easy to see that $\Pc$ is reversible, and has stationary distribution $\pi$.

We next illustrate how combining the two chains works on a simple example, where a mixture of only a few log-modular distributions can dramatically improve mixing compared to running the vanilla Gibbs chain.
Then %, %in the next section
we propose an algorithm for automatically creating such a mixture.

\subsection{Example: Ising Model on the Complete Graph} \label{sect:ising}
We consider the Ising model on a finite complete graph \citep{levin08}, also known as the Curie-Weiss model in statistical physics, which can be written in the form of \eqref{eq:pdef} as follows:
\begin{align*}
  \pib(S) = \frac{1}{Z(\beta)}\exp\left(-\frac{2\beta}{n} |S|(n-|S|)\right). \tag{\isingb}
\end{align*}
In particular, we focus on the case where $\beta = \ln(n)$, that is,
\begin{align*}
  \pi(S) = \frac{1}{Z}\exp\left(-\frac{2\ln(n)}{n} |S|(n-|S|)\right). \tag{\ising}
\end{align*}
In this case, if we define $\dn \defeq 2 \ln(n) / n$, then $F(S) = -\dn |S|(n-|S|)$.

The Gibbs sampler is known to experience poor mixing in this model; the following is an immediate corollary of Theorem 15.3 in \citep{levin08book}.
\begin{cor}[cf. Theorem 15.3 in \citep{levin08book}]
  For $n \geq 3$, the Gibbs sampler on \ising{} has spectral gap $\gg = \bO\left(e^{-cn}\right)$, where $c > 0$ is a constant.
\end{cor}
\noindent From \theoremref{thm:spectral} it follows that the mixing time of Gibbs is
\begin{align*}
    \tme = \Omega\left((e^{cn} - 1)\log\left(\frac{1}{2\epsilon}\right) \right).
\end{align*}
Yet, it has been shown that the only reason for this is a single bottleneck in the state space \citep{levin08}.
To make this statement more formal, let us define a decomposition of $\Omega$ into two disjoint sets, $\Omega_0 \defeq \{S \in \Omega \mid |S| < n/2\}$, and $\Omega_1 \defeq \{S \in \Omega \mid |S| > n/2\}$ \citep{jerrum04poincare}.
To keep things simple, we will assume for the remainder of this section that $n$ is odd; the analysis when $n$ is even follows from the same arguments with only a minor technical adjustment.
Our goal is to separately examine two characteristics of the sampler: (i) its movement between the two sets $\Omega_0$, $\Omega_1$, and (ii) its movement when restricted to stay within each of these sets.

For analyzing the ``between-sets'' behavior, we define the projection $\bar{\pi} : \{0, 1\} \to \mathbb{R}$ of $\pi$ as
\begin{align*}
  \bar{\pi}(i) \defeq \sum_{S \in \Omega_i} \pi(S),
\end{align*}
and, for any reversible chain $P$, we define its projection chain $\bar{P} : \{0, 1\} \times \{0, 1\} \to \mathbb{R}$ as
\begin{align*}
  \bar{P}(i, j) \defeq \frac{1}{\bar{\pi}(i)} \sum_{\subalign{S \in \Omega_i, R \in \Omega_j}} \pi(S) P(S, R).
\end{align*}
It is easy to see that $\bar{P}$ is also reversible and has stationary distribution $\bar{\pi}$. For analyzing the ``within-set'' behavior, we define the restrictions $\pi_i : \Omega_i \to \mathbb{R}$ of $\pi$ as
\begin{align*}
  \pi_i(S) \defeq \frac{\pi_i(S)}{\bar{\pi}(i)},
\end{align*}
and the two restriction chains $P_i : \Omega_i \times \Omega_i \to \mathbb{R}$ of $P$ as
\begin{align*}
  P_i(S, R) \defeq \twopartdefo{P(S, R)}{S \neq R}{1 - \displaystyle\sum_{\subalign{T \in \Omega_i: T \neq S}}P(S, T)}.
\end{align*}
Again, it is easy to see that each of the $P_i$ is also reversible and has stationary distribution $\pi_i$.

Coming back to the Gibbs sampler, if we could show that it mixes fast within each of $\Omega_0$ and $\Omega_1$, then we could deduce that the only reason for the slow mixing on $\Omega$ is the bottleneck between these two sets.
Indeed, the following corollary of a theorem by \cite{ding09} shows exactly that.
\begin{cor}[cf. Theorem 2 in \citep{ding09}] \label{thm:grest}
  For all $n \geq 3$, the restriction chains of the Gibbs sampler $\Pg_i$, $i = 0, 1$, on \ising{} have spectral gap $\gg_i = \Theta\big(\displaystyle\tfrac{2\ln(n) - 1}{n}\big)$.
\end{cor}

To improve mixing we want to create an \Ms{} chain that is able to bypass the aforementioned bottleneck.
For this purpose, we use a mixture of two log-modular distributions, the first of which puts most of its mass on $\Omega_0$, and the second on $\Omega_1$.
We define the mixture of the form \eqref{eq:qdef} by
\begin{align*}
  m_1(S) &= \sum_{v \in S} -\dn (n-1) = -\dn (n-1) |S|,\\
  m_2(S) &= \sum_{v \in S} \dn (n-1) = \dn (n-1) |S|.
  %w_1    &= 1 / Z_1 = 1 / \prod_{v \in V} \left( 1 + \exp(-\dn (n-1)) \right) = \left(1 + \exp(-\dn(n-1))\right)^{-n}\\
  %w_2    &= 1 / Z_2 = 1 / \prod_{v \in V} \left( 1 + \exp(-\dn (n-1)) \right) = \left(1 + \exp(-\dn(n-1))\right)^{-n}.
\end{align*}
We also use $w_1 = 1 / Z_1$ and $w_2 = 1 / Z_2$, where $Z_1$ and $Z_2$ are the normalizers of $m_1$ and $m_2$ respectively.
It follows that $Z_q = 1 / 2$, and, furthermore, the mixture $q$ is symmetric, that is, $q(S) = q(V \setminus S)$.
Since the proposal $q$ is symmetric and state independent, we would expect the \Ms{} chain to jump between $\Omega_0$ and $\Omega_1$ without being hindered by the bottleneck described previously.
We verify this intuition by proving the following lemma.
\begin{lemma} \label{lem:mproj}
  For all $n \geq 10$, the projection chain $\bPm$ of the \Ms{} sampler on \ising{} has spectral gap $\bgm = \Omega(1)$.
\end{lemma}

Putting everything together we show the following result about the combined chain $\Pc$.
\begin{theorem}
  For all $n \geq 10$, the combined chain $\Pc$ on \ising{} has spectral gap
  \begin{align*}
    \gc = \Omega\left( \displaystyle\frac{2\ln(n) - 1}{2n} \right).
  \end{align*}
\end{theorem}
The proof consists of two steps.
In the first step we make a comparison argument \citep{diaconis93,levin08book} to show that the spectral gaps of the projection and restriction chains of the combined sampler are smaller by at most a constant factor in $\alpha$ compared to those of Gibbs and \Ms{}.
In particular, we use the \Ms{} bound (\lemmaref{lem:mproj}) for the projection chain, and the Gibbs bound (\theoremref{thm:grest}) for the restriction chains.
The second step, then, combines the projection and restriction bounds to establish a bound on the spectral gap of the combined chain.
To accomplish this we use a result by \cite{jerrum04poincare}, which, roughly speaking, states that the spectral gap of the whole chain cannot be much smaller than the smallest of the projection and restriction spectral gaps.

Finally, using \theoremref{thm:spectral}, and noting that, in this case, $\pmin = \bO(e^{-n})$ (cf. proof of \lemmaref{lem:mproj}), we get a mixing time of $\tme = \bO(n^2 \log(1 / \epsilon))$ for the combined chain.
This shows that the addition of the \Ms{} sampler results in an exponential improvement in mixing time over the Gibbs sampler by itself.

\section{Constructing the Mixture}
Having seen the positive effect of the \Ms{} sampler, we now turn to the issue of how to choose the proposal $q$.
While a manual construction like the one we just presented for the Ising model may be feasible in some cases, it is often more practical to have an automated way of obtaining the mixture.

Let us assume, as is usually the case, that we have access to a function oracle for $F$, and we want to create a mixture of size $r$.
Ideally, we would like to construct a proposal $q$ that is as close to $\pi$ as possible, that is, minimize an objective such as the following,
\begin{align*}
  E_1(q) &\defeq \min_q \| \pi - q \|\\
         &= \min_q \left\| \frac{\exp(F(\cdot))}{Z} - \frac{1}{Z_q}\textstyle\sum_{i = 1}^r \wi\exp(\mi(\cdot)) \right\|,
\end{align*}
where $\| \cdot \|$ could be, for example, total variation distance or the maximum norm.
Unfortunately, this problem is hard: both computing the partition function $Z$, and jointly optimizing over all $\wi, \mi$ are infeasible in practice.
To make the problem easier, we could try to get rid of the normalizers and weights $\wi$, and iteratively minimize over each $\mi$ individually:
\begin{align*}
  E_2^{(i)}(m_i) \defeq \min_{m_i} \left\| \exp(F(\cdot)) - \textstyle\sum_{j = 1}^{i-1} \exp(\mi(\cdot)) \right\|,
\end{align*}
for $i \in \{1, \ldots, r\}$.
This problem is still hard, since optimizing $\| \exp(F(\cdot)) \|$ is by itself infeasible in general.

\begin{algorithm}[tb]
  \setstretch{1.2}
  \caption{Iterative semigradient-based mixture construction}
  \label{alg:mixture}
  \small{
    \begin{algorithmic}[1]
      \REQUIRE Set function $F$, mixture size $r$
      \FOR{$i = 1$ \TO $r$}
      \LET{$\sigma$}{\textsc{Greedy}($F$, $\{m_1, \ldots, m_{i-1}\}$)} \label{lin:perm}
      \LET{$m_i$}{\textsc{SemiGradient}($F$, $\sigma$)}
      \ENDFOR
      \RETURN $\{m_1, \ldots, m_r\}$
    \end{algorithmic}
  }
\end{algorithm}

\begin{algorithm}[tb]
  \setstretch{1.2}
  \caption{Greedy difference maximization}
  \label{alg:greedy}
  \small{
    \begin{algorithmic}[1]
      \REQUIRE Set function $F$, modular functions $\{m_1, \ldots, m_{i-1}\}$
      \LET{$D_i(S)$}{$F(S) - \log \sum_{j=1}^{i-1} \exp(m_j(S))$, for all $S \in \Omega$}
      \LET{$\sigma$}{$(1, \ldots, n)$}
      \LET{$A$}{$\emptyset$}
      \FOR{$i = 1$ \TO $n$}
      \LET{$v^*$}{$\argmax_{v \in V} \left( D_i(A \cup \{v\}) - D_i(A) \right)$}
      \LET{$\sigma_i$}{$v^*$}
      \LET{$A$}{$A \cup \{v^*\}$}
      \ENDFOR
      \RETURN $\sigma$
    \end{algorithmic}
  }
\end{algorithm}

To arrive at a practical algorithm, we approximate the above objective using the two-step procedure described in \algoref{alg:mixture}.
In the first step, we generate a permutation $\sigma$ of the ground set $V$ by running the greedy algorithm on function $D_i(S) \defeq F(S) - \log \sum_{j=1}^{i-1} \exp(m_j(S))$, as shown in \algoref{alg:greedy}.
Intuitively, the sets that are formed by elements near the beginning of $\sigma$ are those on which $F$ and the current mixture disagree by the most.
Therefore, in the second step, we would like to add to the mixture a modular function $\mi$ that is a good approximation for $F$ on $\{\sigma_1, \ldots, \sigma_k\}$, for a choice of $1 \leq k \leq n$.
To accomplish this, we propose using discrete semigradients.

Semigradients are modular functions that provide lower (subgradient) or upper (supergradient) approximations of a set function $F$ \citep{fujishige05,iyer13}.
More concretely, given a set $S \in \Omega$, a modular function $m$ is a subgradient of $F$ at $S$, if, for all $R \in \Omega$, $F(R) \geq F(S) + m(R) - m(S)$.
Similarly, $m$ is a supergradient if the inequality is reversed.
Although, in general, a function is not guaranteed to have sub- or supergradients at each $S \in \Omega$, it has been shown that this is true when $F$ is submodular or supermodular \citep{fujishige05, jegelka11, iyer12}.

Submodularity expresses a notion of diminishing returns; that is, adding an element to a larger set provides less benefit than adding that same element to a smaller set.
More formally, $F$ is submodular if, for any $S \subseteq R \subseteq V$, and any $v \in V \setminus R$, it holds that $F(R \cup \{v\}) - F(R) \leq F(S \cup \{v\}) - F(S)$.
Supermodularity is defined in a similar way by reversing the sign of this inequality.
The resulting models of the form \eqref{eq:pdef} are referred to as log-submodular and log-supermodular respectively.
Many commonly used models fall under these categories; Ising and Potts models, including our example in the previous section, are log-supermodular, while determinantal point processes and facility location diversity models are log-submodular.

\begin{algorithm}[tb]
    \setstretch{1.2}
	\caption{Subgradient computation}
	\label{alg:sub}
	\small{
		\begin{algorithmic}[1]
			\REQUIRE Set function $F$, permutation $\sigma$
            \LET{$A$}{$\emptyset$}
            \LET{$f$}{$F(\emptyset)$}
			\FOR{$v = 1$ \TO $n$}
			\LET{$m_v$}{$F(A \cup \{\sigma_v\}) - F(A)$}
            \LET{$A$}{$A \cup \sigma_v$}
			\ENDFOR
            \RETURN $m(S) \defeq \sum_{v \in S} m_v$, for all $S \in \Omega$
		\end{algorithmic}
	}
\end{algorithm}

\begin{algorithm}[tb]
    \setstretch{1.2}
	\caption{Supergradient computation}
	\label{alg:super}
	\small{
		\begin{algorithmic}[1]
			\REQUIRE Set function $F$, permutation $\sigma$
            \LET{$k$}{\textsc{DrawUniform}(1, n)}
			\FOR{$v = 1$ \TO $k$}
			\LET{$m_v$}{$F(V) - F(V \setminus \{v\})$}
            \ENDFOR
			\FOR{$v = k+1$ \TO $n$}
			\LET{$m_v$}{$F(\{v\})$}
			\ENDFOR
            \RETURN $m(S) \defeq \sum_{v \in S} m_v$, for all $S \in \Omega$
		\end{algorithmic}
	}
\end{algorithm}

Coming back to the second step of \algoref{alg:mixture}, to create a subgradient of $F$ given permutation $\sigma$ we just need to define a modular function via marginal gains according to the permutation order \citep{iyer13}, as shown in \algoref{alg:sub}.
Moreover, this is a subgradient of $F$ at $\{\sigma_1, \ldots, \sigma_k\}$, for all $1 \leq k \leq n$.
On the other hand, \algoref{alg:super} creates a supergradient of $F$ at $\{\sigma_1, \ldots, \sigma_k\}$ for a randomly chosen $k$. (This type of supergradient is denoted by $\bar{g}_Y$ by \cite{iyer13}.)
In fact, the modular functions $m_1$, $m_2$ that we used in analyzing the Ising model in the previous section were supergradients of $F$ at sets $S_1 = \emptyset$, and $S_2 = V$ respectively.

In practice, we can use \algoref{alg:mixture} regardless of whether $F$ is sub- or supermodular.
We have, however, noticed that subgradients give better results when $F$ is submodular, and the same goes for supergradients and supermodular functions.

\section{Experiments}
\setlength\figureheight{0.33\textwidth}
\setlength\figurewidth{0.35\textwidth}
\renewcommand{\subflen}{0.328\textwidth}
\newcommand{\scspacey}{-0.5em}
\newcommand{\scspacex}{0.2em}
\begin{figure*}[t!]
  \captionsetup[subfigure]{oneside,margin={2em,0em}}
  \begin{subfigure}[b]{\subflen}
    \centering
    \begin{tikzpicture}

\begin{axis}[%
tick label style={/pgf/number format/fixed,font=\sffamily\small},
label style={font=\sffamily\small},
legend style={font=\sffamily\small},
view={0}{90},
width=\figurewidth,
height=\figureheight,
xmin=0, xmax=10000,
xtick={0, 2000, 4000, 6000, 8000, 10000},
xticklabels={0, 2k, 4k, 6k, 8k, 10k},
scaled x ticks=false,
xlabel={Samples},
xlabel shift=0em,
ymin=1, ymax=1.52,
ytick={1, 1.5},
yticklabels={1, 1.5},
ylabel={PSRF},
ylabel shift=-1em,
major tick length=2pt,
axis lines*=left,
legend cell align=left,
clip marker paths=true,
legend style={anchor=north east,at={(1,1)},draw=none,row sep=0em},
every axis plot/.append style={
  line width=1.5pt,
  opacity=0.8,
}
]

\addplot [
color=gcol1,
densely dashed
]
coordinates{
(100,3.378099) +- (-0.282843,0.282843)(150,2.938422) +- (-0.151760,0.151760)(200,2.532023) +- (-0.159217,0.159217)(250,2.261538) +- (-0.111451,0.111451)(300,2.054314) +- (-0.092123,0.092123)(350,1.874944) +- (-0.070360,0.070360)(400,1.767087) +- (-0.064088,0.064088)(450,1.681454) +- (-0.065038,0.065038)(500,1.592597) +- (-0.053144,0.053144)(550,1.543078) +- (-0.046912,0.046912)(600,1.511899) +- (-0.046000,0.046000)(650,1.486062) +- (-0.044927,0.044927)(700,1.457672) +- (-0.041328,0.041328)(750,1.430176) +- (-0.035585,0.035585)(800,1.401466) +- (-0.030985,0.030985)(850,1.375708) +- (-0.029948,0.029948)(900,1.345175) +- (-0.026733,0.026733)(950,1.315470) +- (-0.026751,0.026751)(1000,1.293519) +- (-0.025797,0.025797)(1000,1.293519) +- (-0.025797,0.025797)(1200,1.251332) +- (-0.026475,0.026475)(1400,1.213421) +- (-0.019653,0.019653)(1600,1.179927) +- (-0.016185,0.016185)(1800,1.152976) +- (-0.013040,0.013040)(2000,1.139469) +- (-0.012673,0.012673)(2200,1.125057) +- (-0.012795,0.012795)(2400,1.112741) +- (-0.008976,0.008976)(2600,1.107873) +- (-0.008505,0.008505)(2800,1.100164) +- (-0.008471,0.008471)(3000,1.092954) +- (-0.008492,0.008492)(3200,1.089254) +- (-0.007482,0.007482)(3400,1.080773) +- (-0.006507,0.006507)(3600,1.075257) +- (-0.006793,0.006793)(3800,1.074780) +- (-0.007282,0.007282)(4000,1.072297) +- (-0.007637,0.007637)(4200,1.067025) +- (-0.006494,0.006494)(4400,1.065540) +- (-0.006410,0.006410)(4600,1.062918) +- (-0.005352,0.005352)(4800,1.059046) +- (-0.004797,0.004797)(5000,1.057516) +- (-0.004167,0.004167)(5200,1.055310) +- (-0.003949,0.003949)(5400,1.053404) +- (-0.004055,0.004055)(5600,1.052183) +- (-0.003841,0.003841)(5800,1.051552) +- (-0.003584,0.003584)(6000,1.050591) +- (-0.003938,0.003938)(6200,1.047492) +- (-0.004053,0.004053)(6400,1.044872) +- (-0.003768,0.003768)(6600,1.043283) +- (-0.003739,0.003739)(6800,1.041803) +- (-0.003703,0.003703)(7000,1.040122) +- (-0.003554,0.003554)(7200,1.038240) +- (-0.003463,0.003463)(7400,1.036134) +- (-0.003056,0.003056)(7600,1.034900) +- (-0.002855,0.002855)(7800,1.034405) +- (-0.002825,0.002825)(8000,1.032886) +- (-0.002893,0.002893)(8200,1.031770) +- (-0.002710,0.002710)(8400,1.031264) +- (-0.002854,0.002854)(8600,1.030591) +- (-0.002817,0.002817)(8800,1.029803) +- (-0.002770,0.002770)(9000,1.029227) +- (-0.002713,0.002713)(9200,1.028724) +- (-0.002527,0.002527)(9400,1.028041) +- (-0.002393,0.002393)(9600,1.027631) +- (-0.002233,0.002233)(9800,1.027045) +- (-0.002216,0.002216)(10000,1.025711) +- (-0.002096,0.002096)
};
\addlegendentry{\textsc{Gibbs}}


\addplot [
color=gcol2
]
coordinates{
(100,1.245774) +- (-0.033535,0.033535)(150,1.160920) +- (-0.020011,0.020011)(200,1.109728) +- (-0.010276,0.010276)(250,1.086212) +- (-0.007441,0.007441)(300,1.073510) +- (-0.006756,0.006756)(350,1.064314) +- (-0.005584,0.005584)(400,1.053786) +- (-0.003881,0.003881)(450,1.048140) +- (-0.004670,0.004670)(500,1.038634) +- (-0.002423,0.002423)(550,1.035454) +- (-0.002357,0.002357)(600,1.033217) +- (-0.002506,0.002506)(650,1.030281) +- (-0.002513,0.002513)(700,1.027761) +- (-0.002301,0.002301)(750,1.027078) +- (-0.002488,0.002488)(800,1.024988) +- (-0.002331,0.002331)(850,1.024239) +- (-0.001891,0.001891)(900,1.022156) +- (-0.001505,0.001505)(950,1.020619) +- (-0.001627,0.001627)(1000,1.019861) +- (-0.001773,0.001773)(1000,1.019861) +- (-0.001773,0.001773)(1200,1.016820) +- (-0.001244,0.001244)(1400,1.014436) +- (-0.001055,0.001055)(1600,1.012749) +- (-0.000940,0.000940)(1800,1.011177) +- (-0.000761,0.000761)(2000,1.009927) +- (-0.000671,0.000671)(2200,1.008491) +- (-0.000571,0.000571)(2400,1.008012) +- (-0.000392,0.000392)(2600,1.007707) +- (-0.000417,0.000417)(2800,1.007143) +- (-0.000457,0.000457)(3000,1.006990) +- (-0.000413,0.000413)(3200,1.006346) +- (-0.000359,0.000359)(3400,1.006063) +- (-0.000338,0.000338)(3600,1.005638) +- (-0.000347,0.000347)(3800,1.005349) +- (-0.000345,0.000345)(4000,1.004896) +- (-0.000340,0.000340)(4200,1.004804) +- (-0.000285,0.000285)(4400,1.004610) +- (-0.000316,0.000316)(4600,1.004518) +- (-0.000303,0.000303)(4800,1.004272) +- (-0.000314,0.000314)(5000,1.004110) +- (-0.000288,0.000288)(5200,1.003938) +- (-0.000245,0.000245)(5400,1.003713) +- (-0.000212,0.000212)(5600,1.003553) +- (-0.000216,0.000216)(5800,1.003460) +- (-0.000207,0.000207)(6000,1.003411) +- (-0.000193,0.000193)(6200,1.003324) +- (-0.000225,0.000225)(6400,1.003243) +- (-0.000202,0.000202)(6600,1.003171) +- (-0.000187,0.000187)(6800,1.003077) +- (-0.000180,0.000180)(7000,1.002930) +- (-0.000182,0.000182)(7200,1.002898) +- (-0.000168,0.000168)(7400,1.002849) +- (-0.000186,0.000186)(7600,1.002778) +- (-0.000183,0.000183)(7800,1.002698) +- (-0.000179,0.000179)(8000,1.002621) +- (-0.000175,0.000175)(8200,1.002526) +- (-0.000173,0.000173)(8400,1.002442) +- (-0.000167,0.000167)(8600,1.002387) +- (-0.000161,0.000161)(8800,1.002376) +- (-0.000166,0.000166)(9000,1.002282) +- (-0.000153,0.000153)(9200,1.002204) +- (-0.000155,0.000155)(9400,1.002200) +- (-0.000154,0.000154)(9600,1.002114) +- (-0.000145,0.000145)(9800,1.002087) +- (-0.000136,0.000136)(10000,1.002027) +- (-0.000124,0.000124)
};
\addlegendentry{\textsc{Combo-R}}


\addplot [
color=gcol3
]
coordinates{
(100,1.346909) +- (-0.036826,0.036826)(150,1.200930) +- (-0.015355,0.015355)(200,1.145036) +- (-0.014122,0.014122)(250,1.107371) +- (-0.008610,0.008610)(300,1.089872) +- (-0.006056,0.006056)(350,1.075572) +- (-0.005809,0.005809)(400,1.066123) +- (-0.004161,0.004161)(450,1.057875) +- (-0.003989,0.003989)(500,1.051135) +- (-0.003911,0.003911)(550,1.047260) +- (-0.002989,0.002989)(600,1.044577) +- (-0.003433,0.003433)(650,1.041867) +- (-0.003330,0.003330)(700,1.038111) +- (-0.003005,0.003005)(750,1.035853) +- (-0.002558,0.002558)(800,1.033717) +- (-0.002574,0.002574)(850,1.032791) +- (-0.002601,0.002601)(900,1.030495) +- (-0.002440,0.002440)(950,1.028133) +- (-0.002260,0.002260)(1000,1.026836) +- (-0.002045,0.002045)(1000,1.026836) +- (-0.002045,0.002045)(1200,1.022948) +- (-0.001406,0.001406)(1400,1.019134) +- (-0.001063,0.001063)(1600,1.017185) +- (-0.001150,0.001150)(1800,1.015185) +- (-0.000983,0.000983)(2000,1.013325) +- (-0.000945,0.000945)(2200,1.012275) +- (-0.000836,0.000836)(2400,1.011077) +- (-0.000846,0.000846)(2600,1.009959) +- (-0.000719,0.000719)(2800,1.009164) +- (-0.000653,0.000653)(3000,1.009077) +- (-0.000666,0.000666)(3200,1.008668) +- (-0.000673,0.000673)(3400,1.008168) +- (-0.000633,0.000633)(3600,1.007867) +- (-0.000621,0.000621)(3800,1.007480) +- (-0.000528,0.000528)(4000,1.006745) +- (-0.000498,0.000498)(4200,1.006737) +- (-0.000481,0.000481)(4400,1.006562) +- (-0.000442,0.000442)(4600,1.006253) +- (-0.000423,0.000423)(4800,1.005926) +- (-0.000413,0.000413)(5000,1.005456) +- (-0.000388,0.000388)(5200,1.005277) +- (-0.000376,0.000376)(5400,1.005025) +- (-0.000353,0.000353)(5600,1.004772) +- (-0.000320,0.000320)(5800,1.004524) +- (-0.000332,0.000332)(6000,1.004534) +- (-0.000352,0.000352)(6200,1.004355) +- (-0.000328,0.000328)(6400,1.004231) +- (-0.000331,0.000331)(6600,1.004111) +- (-0.000330,0.000330)(6800,1.003869) +- (-0.000308,0.000308)(7000,1.003742) +- (-0.000276,0.000276)(7200,1.003613) +- (-0.000289,0.000289)(7400,1.003568) +- (-0.000258,0.000258)(7600,1.003527) +- (-0.000235,0.000235)(7800,1.003379) +- (-0.000233,0.000233)(8000,1.003240) +- (-0.000226,0.000226)(8200,1.003186) +- (-0.000227,0.000227)(8400,1.003065) +- (-0.000217,0.000217)(8600,1.002903) +- (-0.000189,0.000189)(8800,1.002855) +- (-0.000174,0.000174)(9000,1.002832) +- (-0.000185,0.000185)(9200,1.002703) +- (-0.000165,0.000165)(9400,1.002623) +- (-0.000163,0.000163)(9600,1.002542) +- (-0.000151,0.000151)(9800,1.002535) +- (-0.000142,0.000142)(10000,1.002501) +- (-0.000135,0.000135)
};
\addlegendentry{\textsc{Combo-I}}


\addplot [
color=gcol4
]
coordinates{
(100,1.167109) +- (-0.027878,0.027878)(150,1.102454) +- (-0.012394,0.012394)(200,1.078688) +- (-0.008910,0.008910)(250,1.063416) +- (-0.007208,0.007208)(300,1.049136) +- (-0.004693,0.004693)(350,1.040569) +- (-0.003381,0.003381)(400,1.035484) +- (-0.003283,0.003283)(450,1.032803) +- (-0.003334,0.003334)(500,1.028276) +- (-0.002731,0.002731)(550,1.027382) +- (-0.002881,0.002881)(600,1.025636) +- (-0.002789,0.002789)(650,1.022359) +- (-0.002361,0.002361)(700,1.020696) +- (-0.001862,0.001862)(750,1.019722) +- (-0.001640,0.001640)(800,1.018251) +- (-0.001516,0.001516)(850,1.016289) +- (-0.001461,0.001461)(900,1.016462) +- (-0.001700,0.001700)(950,1.016098) +- (-0.001628,0.001628)(1000,1.015015) +- (-0.001380,0.001380)(1000,1.015015) +- (-0.001380,0.001380)(1200,1.013406) +- (-0.001161,0.001161)(1400,1.010840) +- (-0.000956,0.000956)(1600,1.009103) +- (-0.000637,0.000637)(1800,1.007829) +- (-0.000706,0.000706)(2000,1.007327) +- (-0.000517,0.000517)(2200,1.006347) +- (-0.000406,0.000406)(2400,1.005852) +- (-0.000404,0.000404)(2600,1.005511) +- (-0.000387,0.000387)(2800,1.005048) +- (-0.000333,0.000333)(3000,1.004469) +- (-0.000345,0.000345)(3200,1.004287) +- (-0.000304,0.000304)(3400,1.004014) +- (-0.000279,0.000279)(3600,1.003876) +- (-0.000252,0.000252)(3800,1.003830) +- (-0.000266,0.000266)(4000,1.003727) +- (-0.000247,0.000247)(4200,1.003600) +- (-0.000276,0.000276)(4400,1.003388) +- (-0.000244,0.000244)(4600,1.003196) +- (-0.000199,0.000199)(4800,1.003092) +- (-0.000184,0.000184)(5000,1.002929) +- (-0.000159,0.000159)(5200,1.002904) +- (-0.000143,0.000143)(5400,1.002731) +- (-0.000140,0.000140)(5600,1.002553) +- (-0.000123,0.000123)(5800,1.002522) +- (-0.000146,0.000146)(6000,1.002494) +- (-0.000149,0.000149)(6200,1.002358) +- (-0.000170,0.000170)(6400,1.002312) +- (-0.000148,0.000148)(6600,1.002236) +- (-0.000155,0.000155)(6800,1.002144) +- (-0.000146,0.000146)(7000,1.002133) +- (-0.000138,0.000138)(7200,1.002031) +- (-0.000122,0.000122)(7400,1.001990) +- (-0.000121,0.000121)(7600,1.001909) +- (-0.000125,0.000125)(7800,1.001884) +- (-0.000126,0.000126)(8000,1.001830) +- (-0.000124,0.000124)(8200,1.001787) +- (-0.000127,0.000127)(8400,1.001740) +- (-0.000116,0.000116)(8600,1.001684) +- (-0.000093,0.000093)(8800,1.001597) +- (-0.000086,0.000086)(9000,1.001551) +- (-0.000084,0.000084)(9200,1.001536) +- (-0.000076,0.000076)(9400,1.001547) +- (-0.000070,0.000070)(9600,1.001504) +- (-0.000067,0.000067)(9800,1.001491) +- (-0.000070,0.000070)(10000,1.001479) +- (-0.000078,0.000078)
};
\addlegendentry{\textsc{Combo-F}}

\end{axis}
\end{tikzpicture}
    \vspace{\scspacey}
    \caption{\hspace{\scspacex}\textsc{Ising} ($n = 6$)}
    \label{fig:ising6}
  \end{subfigure}
  \begin{subfigure}[b]{\subflen}
    \begin{tikzpicture}

\begin{axis}[%
tick label style={/pgf/number format/fixed,font=\sffamily\small},
label style={font=\sffamily\small},
legend style={font=\sffamily\small},
view={0}{90},
width=\figurewidth,
height=\figureheight,
xmin=0, xmax=10000,
xtick={0, 2000, 4000, 6000, 8000, 10000},
xticklabels={0, 2k, 4k, 6k, 8k, 10k},
scaled x ticks=false,
xlabel={Samples},
xlabel shift=-0.3em,
ymin=1, ymax=1.52,
ytick={1, 1.5},
yticklabels={1, 1.5},
ylabel={PSRF},
ylabel shift=-1.1em,
major tick length=2pt,
axis lines*=left,
legend cell align=left,
clip marker paths=true,
legend style={anchor=north east,at={(1,1)},draw=none,row sep=0em},
every axis plot/.append style={
  line width=1.5pt,
  opacity=0.8,
}
]

\addplot [
color=col1dark,
densely dashed
]
coordinates{
(100,3.896208) +- (-0.282843,0.282843)(150,3.739615) +- (-0.282843,0.282843)(200,3.642484) +- (-0.282843,0.282843)(250,3.557257) +- (-0.282843,0.282843)(300,3.178533) +- (-0.277385,0.277385)(350,2.974985) +- (-0.227180,0.227180)(400,2.708438) +- (-0.190721,0.190721)(450,2.561393) +- (-0.192645,0.192645)(500,2.463448) +- (-0.173053,0.173053)(550,2.337529) +- (-0.148764,0.148764)(600,2.218182) +- (-0.120421,0.120421)(650,2.165834) +- (-0.107972,0.107972)(700,2.109161) +- (-0.107845,0.107845)(750,2.058582) +- (-0.106084,0.106084)(800,2.000832) +- (-0.099719,0.099719)(850,1.937352) +- (-0.087618,0.087618)(900,1.872072) +- (-0.082553,0.082553)(950,1.812060) +- (-0.081620,0.081620)(1000,1.755994) +- (-0.073638,0.073638)(1000,1.755994) +- (-0.073638,0.073638)(1200,1.587606) +- (-0.050434,0.050434)(1400,1.495127) +- (-0.039021,0.039021)(1600,1.444903) +- (-0.036102,0.036102)(1800,1.395842) +- (-0.030115,0.030115)(2000,1.342865) +- (-0.026348,0.026348)(2200,1.305323) +- (-0.023374,0.023374)(2400,1.280074) +- (-0.022377,0.022377)(2600,1.266798) +- (-0.023385,0.023385)(2800,1.246877) +- (-0.023293,0.023293)(3000,1.239355) +- (-0.023334,0.023334)(3200,1.227831) +- (-0.022143,0.022143)(3400,1.215640) +- (-0.021151,0.021151)(3600,1.200760) +- (-0.020098,0.020098)(3800,1.189016) +- (-0.019135,0.019135)(4000,1.178751) +- (-0.019408,0.019408)(4200,1.168659) +- (-0.017415,0.017415)(4400,1.155525) +- (-0.015521,0.015521)(4600,1.146588) +- (-0.014234,0.014234)(4800,1.139240) +- (-0.013193,0.013193)(5000,1.131557) +- (-0.012880,0.012880)(5200,1.123362) +- (-0.011860,0.011860)(5400,1.116154) +- (-0.010758,0.010758)(5600,1.110660) +- (-0.009586,0.009586)(5800,1.108292) +- (-0.008875,0.008875)(6000,1.105119) +- (-0.008731,0.008731)(6200,1.100983) +- (-0.008110,0.008110)(6400,1.097247) +- (-0.007703,0.007703)(6600,1.094453) +- (-0.007182,0.007182)(6800,1.093285) +- (-0.006628,0.006628)(7000,1.090694) +- (-0.006238,0.006238)(7200,1.087508) +- (-0.005763,0.005763)(7400,1.084090) +- (-0.005489,0.005489)(7600,1.082622) +- (-0.005698,0.005698)(7800,1.081452) +- (-0.006137,0.006137)(8000,1.079893) +- (-0.006483,0.006483)(8200,1.076788) +- (-0.006266,0.006266)(8400,1.074402) +- (-0.006139,0.006139)(8600,1.074080) +- (-0.006209,0.006209)(8800,1.073100) +- (-0.006081,0.006081)(9000,1.072107) +- (-0.006115,0.006115)(9200,1.071092) +- (-0.006279,0.006279)(9400,1.069355) +- (-0.006526,0.006526)(9600,1.066838) +- (-0.006230,0.006230)(9800,1.065441) +- (-0.006042,0.006042)(10000,1.064114) +- (-0.006060,0.006060)
};
\addlegendentry{\textsc{Gibbs}}


\addplot [
color=col2
]
coordinates{
(100,1.314417) +- (-0.040330,0.040330)(150,1.209030) +- (-0.024554,0.024554)(200,1.150989) +- (-0.013388,0.013388)(250,1.125998) +- (-0.018067,0.018067)(300,1.090324) +- (-0.007602,0.007602)(350,1.076498) +- (-0.007112,0.007112)(400,1.069500) +- (-0.006393,0.006393)(450,1.061250) +- (-0.004808,0.004808)(500,1.051600) +- (-0.003444,0.003444)(550,1.045732) +- (-0.003368,0.003368)(600,1.041204) +- (-0.002925,0.002925)(650,1.039474) +- (-0.002560,0.002560)(700,1.037444) +- (-0.002689,0.002689)(750,1.034435) +- (-0.002444,0.002444)(800,1.032142) +- (-0.002092,0.002092)(850,1.030101) +- (-0.001856,0.001856)(900,1.028718) +- (-0.002028,0.002028)(950,1.027511) +- (-0.002021,0.002021)(1000,1.026990) +- (-0.001831,0.001831)(1000,1.026990) +- (-0.001831,0.001831)(1200,1.021052) +- (-0.001389,0.001389)(1400,1.018130) +- (-0.001291,0.001291)(1600,1.015379) +- (-0.000941,0.000941)(1800,1.013874) +- (-0.000708,0.000708)(2000,1.012458) +- (-0.000780,0.000780)(2200,1.011068) +- (-0.000769,0.000769)(2400,1.010439) +- (-0.000648,0.000648)(2600,1.009839) +- (-0.000608,0.000608)(2800,1.009137) +- (-0.000598,0.000598)(3000,1.008405) +- (-0.000521,0.000521)(3200,1.007805) +- (-0.000469,0.000469)(3400,1.007401) +- (-0.000484,0.000484)(3600,1.007214) +- (-0.000529,0.000529)(3800,1.006806) +- (-0.000466,0.000466)(4000,1.006600) +- (-0.000436,0.000436)(4200,1.006301) +- (-0.000401,0.000401)(4400,1.005951) +- (-0.000495,0.000495)(4600,1.005535) +- (-0.000425,0.000425)(4800,1.005343) +- (-0.000332,0.000332)(5000,1.005305) +- (-0.000314,0.000314)(5200,1.005179) +- (-0.000344,0.000344)(5400,1.004942) +- (-0.000321,0.000321)(5600,1.004678) +- (-0.000300,0.000300)(5800,1.004468) +- (-0.000308,0.000308)(6000,1.004385) +- (-0.000281,0.000281)(6200,1.004103) +- (-0.000260,0.000260)(6400,1.003915) +- (-0.000235,0.000235)(6600,1.003827) +- (-0.000239,0.000239)(6800,1.003676) +- (-0.000229,0.000229)(7000,1.003487) +- (-0.000203,0.000203)(7200,1.003504) +- (-0.000185,0.000185)(7400,1.003397) +- (-0.000192,0.000192)(7600,1.003317) +- (-0.000167,0.000167)(7800,1.003166) +- (-0.000152,0.000152)(8000,1.003073) +- (-0.000166,0.000166)(8200,1.003101) +- (-0.000156,0.000156)(8400,1.003031) +- (-0.000147,0.000147)(8600,1.002982) +- (-0.000141,0.000141)(8800,1.002947) +- (-0.000148,0.000148)(9000,1.002924) +- (-0.000146,0.000146)(9200,1.002870) +- (-0.000148,0.000148)(9400,1.002856) +- (-0.000151,0.000151)(9600,1.002785) +- (-0.000147,0.000147)(9800,1.002726) +- (-0.000172,0.000172)(10000,1.002621) +- (-0.000173,0.000173)
};
\addlegendentry{\textsc{Combo-R}}


\addplot [
color=col3
]
coordinates{
(100,1.302410) +- (-0.043299,0.043299)(150,1.192313) +- (-0.030443,0.030443)(200,1.136029) +- (-0.019970,0.019970)(250,1.107647) +- (-0.012972,0.012972)(300,1.088088) +- (-0.008265,0.008265)(350,1.073005) +- (-0.006480,0.006480)(400,1.062464) +- (-0.004980,0.004980)(450,1.053361) +- (-0.003315,0.003315)(500,1.049643) +- (-0.003606,0.003606)(550,1.043930) +- (-0.002900,0.002900)(600,1.041073) +- (-0.003059,0.003059)(650,1.038048) +- (-0.002857,0.002857)(700,1.035109) +- (-0.003045,0.003045)(750,1.032760) +- (-0.002621,0.002621)(800,1.030170) +- (-0.002730,0.002730)(850,1.028128) +- (-0.002075,0.002075)(900,1.027046) +- (-0.001771,0.001771)(950,1.025782) +- (-0.001688,0.001688)(1000,1.023793) +- (-0.001515,0.001515)(1000,1.023793) +- (-0.001515,0.001515)(1200,1.020223) +- (-0.001479,0.001479)(1400,1.016721) +- (-0.001164,0.001164)(1600,1.014771) +- (-0.001049,0.001049)(1800,1.012381) +- (-0.000901,0.000901)(2000,1.011654) +- (-0.000751,0.000751)(2200,1.010908) +- (-0.000794,0.000794)(2400,1.010202) +- (-0.000848,0.000848)(2600,1.009230) +- (-0.000826,0.000826)(2800,1.008373) +- (-0.000594,0.000594)(3000,1.008123) +- (-0.000602,0.000602)(3200,1.007506) +- (-0.000482,0.000482)(3400,1.006853) +- (-0.000470,0.000470)(3600,1.006266) +- (-0.000356,0.000356)(3800,1.005991) +- (-0.000345,0.000345)(4000,1.005849) +- (-0.000395,0.000395)(4200,1.005754) +- (-0.000388,0.000388)(4400,1.005485) +- (-0.000341,0.000341)(4600,1.005305) +- (-0.000303,0.000303)(4800,1.004982) +- (-0.000286,0.000286)(5000,1.004676) +- (-0.000254,0.000254)(5200,1.004508) +- (-0.000247,0.000247)(5400,1.004375) +- (-0.000246,0.000246)(5600,1.004133) +- (-0.000203,0.000203)(5800,1.003920) +- (-0.000220,0.000220)(6000,1.003703) +- (-0.000195,0.000195)(6200,1.003629) +- (-0.000186,0.000186)(6400,1.003452) +- (-0.000180,0.000180)(6600,1.003327) +- (-0.000168,0.000168)(6800,1.003261) +- (-0.000164,0.000164)(7000,1.003318) +- (-0.000188,0.000188)(7200,1.003169) +- (-0.000158,0.000158)(7400,1.003172) +- (-0.000177,0.000177)(7600,1.003090) +- (-0.000181,0.000181)(7800,1.003010) +- (-0.000161,0.000161)(8000,1.003009) +- (-0.000164,0.000164)(8200,1.002962) +- (-0.000173,0.000173)(8400,1.002898) +- (-0.000173,0.000173)(8600,1.002858) +- (-0.000183,0.000183)(8800,1.002780) +- (-0.000168,0.000168)(9000,1.002702) +- (-0.000170,0.000170)(9200,1.002611) +- (-0.000170,0.000170)(9400,1.002519) +- (-0.000163,0.000163)(9600,1.002455) +- (-0.000156,0.000156)(9800,1.002426) +- (-0.000154,0.000154)(10000,1.002355) +- (-0.000148,0.000148)
};
\addlegendentry{\textsc{Combo-I}}


\addplot [
color=col4
]
coordinates{
(100,1.180809) +- (-0.022225,0.022225)(150,1.126401) +- (-0.014874,0.014874)(200,1.085758) +- (-0.009005,0.009005)(250,1.066076) +- (-0.007230,0.007230)(300,1.056284) +- (-0.006858,0.006858)(350,1.047681) +- (-0.006919,0.006919)(400,1.041360) +- (-0.005261,0.005261)(450,1.037412) +- (-0.003870,0.003870)(500,1.032972) +- (-0.002726,0.002726)(550,1.029087) +- (-0.002539,0.002539)(600,1.026696) +- (-0.002342,0.002342)(650,1.026147) +- (-0.002264,0.002264)(700,1.024491) +- (-0.002284,0.002284)(750,1.021859) +- (-0.001836,0.001836)(800,1.020478) +- (-0.001601,0.001601)(850,1.019069) +- (-0.001645,0.001645)(900,1.017252) +- (-0.001690,0.001690)(950,1.016957) +- (-0.001471,0.001471)(1000,1.016413) +- (-0.001353,0.001353)(1000,1.016413) +- (-0.001353,0.001353)(1200,1.012701) +- (-0.000900,0.000900)(1400,1.011443) +- (-0.000873,0.000873)(1600,1.009846) +- (-0.000574,0.000574)(1800,1.008508) +- (-0.000649,0.000649)(2000,1.007876) +- (-0.000647,0.000647)(2200,1.007269) +- (-0.000575,0.000575)(2400,1.006581) +- (-0.000490,0.000490)(2600,1.005895) +- (-0.000388,0.000388)(2800,1.005574) +- (-0.000391,0.000391)(3000,1.005168) +- (-0.000457,0.000457)(3200,1.004587) +- (-0.000364,0.000364)(3400,1.004324) +- (-0.000284,0.000284)(3600,1.004170) +- (-0.000273,0.000273)(3800,1.003943) +- (-0.000272,0.000272)(4000,1.003731) +- (-0.000262,0.000262)(4200,1.003555) +- (-0.000268,0.000268)(4400,1.003436) +- (-0.000245,0.000245)(4600,1.003272) +- (-0.000230,0.000230)(4800,1.003141) +- (-0.000240,0.000240)(5000,1.003152) +- (-0.000232,0.000232)(5200,1.003070) +- (-0.000240,0.000240)(5400,1.002956) +- (-0.000246,0.000246)(5600,1.002786) +- (-0.000253,0.000253)(5800,1.002652) +- (-0.000216,0.000216)(6000,1.002575) +- (-0.000207,0.000207)(6200,1.002537) +- (-0.000208,0.000208)(6400,1.002443) +- (-0.000194,0.000194)(6600,1.002250) +- (-0.000140,0.000140)(6800,1.002180) +- (-0.000142,0.000142)(7000,1.002103) +- (-0.000141,0.000141)(7200,1.002085) +- (-0.000148,0.000148)(7400,1.002006) +- (-0.000134,0.000134)(7600,1.001983) +- (-0.000130,0.000130)(7800,1.001955) +- (-0.000119,0.000119)(8000,1.001893) +- (-0.000115,0.000115)(8200,1.001879) +- (-0.000127,0.000127)(8400,1.001866) +- (-0.000122,0.000122)(8600,1.001772) +- (-0.000124,0.000124)(8800,1.001737) +- (-0.000138,0.000138)(9000,1.001643) +- (-0.000119,0.000119)(9200,1.001596) +- (-0.000097,0.000097)(9400,1.001548) +- (-0.000091,0.000091)(9600,1.001541) +- (-0.000097,0.000097)(9800,1.001507) +- (-0.000094,0.000094)(10000,1.001486) +- (-0.000086,0.000086)
};
\addlegendentry{\textsc{Combo-F}}

\end{axis}
\end{tikzpicture}
    \vspace{\scspacey}
    \caption{\hspace{\scspacex}\textsc{Ising} ($n = 7$)}
    \label{fig:ising7}
  \end{subfigure}
  \begin{subfigure}[b]{\subflen}
    \begin{tikzpicture}

\colorlet{col1}{blue!50!black}
\colorlet{col2}{lime!50!black}
\colorlet{col3}{red!70!black}
\colorlet{col4}{cyan!50!black}

\begin{axis}[%
tick label style={font=\tiny},
label style={font=\scriptsize},
legend style={font=\tiny},
view={0}{90},
width=\figurewidth,
height=\figureheight,
xmin=0, xmax=10000,
xtick={0, 2000, 4000, 6000, 8000, 10000},
xticklabels={0, 2k, 4k, 6k, 8k, 10k},
scaled x ticks=false,
xlabel={Samples},
xlabel shift=-0.3em,
ymin=1, ymax=1.52,
ytick={1, 1.5},
ylabel={PSRF},
ylabel shift=-1.5em,
tick label style={/pgf/number format/fixed},
major tick length=2pt,
axis lines*=left,
legend cell align=left,
clip marker paths=true,
legend style={at={(1.05,1.05)},draw=none,row sep=-0.35em}]

\addplot [
mark=none,
mark size=1.0pt,
mark options={solid},
color=col1,
densely dashed,
line width=1pt,
opacity=0.7,
%error bars/.cd,
%error bar style={solid, line width=0.2pt},
%y dir=both,
%y explicit
]
coordinates{
(200,5.427528) +- (-0.282843,0.282843)(250,5.084734) +- (-0.282843,0.282843)(300,4.755111) +- (-0.282843,0.282843)(350,4.364048) +- (-0.282843,0.282843)(400,4.218890) +- (-0.282843,0.282843)(450,4.067331) +- (-0.282843,0.282843)(500,3.829240) +- (-0.282843,0.282843)(550,3.798048) +- (-0.282843,0.282843)(600,3.689457) +- (-0.282843,0.282843)(650,3.603386) +- (-0.282843,0.282843)(700,3.473374) +- (-0.282843,0.282843)(750,3.336994) +- (-0.282843,0.282843)(800,3.202073) +- (-0.281051,0.281051)(850,3.125890) +- (-0.267129,0.267129)(900,3.043449) +- (-0.244012,0.244012)(950,2.944136) +- (-0.210823,0.210823)(1000,2.837069) +- (-0.174730,0.174730)(1000,2.837069) +- (-0.174730,0.174730)(1200,2.547994) +- (-0.133607,0.133607)(1400,2.313947) +- (-0.132654,0.132654)(1600,2.080350) +- (-0.102417,0.102417)(1800,1.970915) +- (-0.091169,0.091169)(2000,1.888327) +- (-0.077586,0.077586)(2200,1.818411) +- (-0.079157,0.079157)(2400,1.757219) +- (-0.070992,0.070992)(2600,1.707643) +- (-0.066953,0.066953)(2800,1.644363) +- (-0.062832,0.062832)(3000,1.586437) +- (-0.057791,0.057791)(3200,1.541493) +- (-0.055043,0.055043)(3400,1.508468) +- (-0.051963,0.051963)(3600,1.481946) +- (-0.048360,0.048360)(3800,1.464721) +- (-0.043825,0.043825)(4000,1.440116) +- (-0.038660,0.038660)(4200,1.417070) +- (-0.036301,0.036301)(4400,1.397990) +- (-0.034385,0.034385)(4600,1.382979) +- (-0.034064,0.034064)(4800,1.373835) +- (-0.034643,0.034643)(5000,1.362791) +- (-0.035449,0.035449)(5200,1.348894) +- (-0.036057,0.036057)(5400,1.332944) +- (-0.035871,0.035871)(5600,1.321601) +- (-0.034242,0.034242)(5800,1.307290) +- (-0.030942,0.030942)(6000,1.292171) +- (-0.027866,0.027866)(6200,1.284529) +- (-0.026780,0.026780)(6400,1.276377) +- (-0.026343,0.026343)(6600,1.267067) +- (-0.024829,0.024829)(6800,1.256202) +- (-0.023464,0.023464)(7000,1.247446) +- (-0.022590,0.022590)(7200,1.238983) +- (-0.022168,0.022168)(7400,1.229817) +- (-0.021976,0.021976)(7600,1.222187) +- (-0.021931,0.021931)(7800,1.216389) +- (-0.021946,0.021946)(8000,1.213026) +- (-0.021953,0.021953)(8200,1.208784) +- (-0.021730,0.021730)(8400,1.204279) +- (-0.021701,0.021701)(8600,1.200526) +- (-0.021376,0.021376)(8800,1.196185) +- (-0.020910,0.020910)(9000,1.192338) +- (-0.020313,0.020313)(9200,1.187264) +- (-0.019207,0.019207)(9400,1.181957) +- (-0.018455,0.018455)(9600,1.176259) +- (-0.017410,0.017410)(9800,1.171533) +- (-0.016607,0.016607)(10000,1.168757) +- (-0.015745,0.015745)
};
\addlegendentry{\textsc{Gibbs}}


\addplot [
mark=none,
mark size=1.0pt,
color=col2,
line width=1pt,
opacity=0.7,
%error bars/.cd,
%error bar style={line width=0.2pt},
%y dir=both,
%y explicit
]
coordinates{
(100,1.836437) +- (-0.225487,0.225487)(150,1.389662) +- (-0.034546,0.034546)(200,1.332339) +- (-0.058548,0.058548)(250,1.243756) +- (-0.028224,0.028224)(300,1.191953) +- (-0.018964,0.018964)(350,1.155793) +- (-0.013018,0.013018)(400,1.136470) +- (-0.013779,0.013779)(450,1.117887) +- (-0.010808,0.010808)(500,1.111575) +- (-0.009380,0.009380)(550,1.100628) +- (-0.008795,0.008795)(600,1.089563) +- (-0.010081,0.010081)(650,1.082914) +- (-0.008692,0.008692)(700,1.077152) +- (-0.008354,0.008354)(750,1.073758) +- (-0.008524,0.008524)(800,1.066387) +- (-0.006605,0.006605)(850,1.064492) +- (-0.006074,0.006074)(900,1.062545) +- (-0.005428,0.005428)(950,1.061159) +- (-0.005386,0.005386)(1000,1.057204) +- (-0.004538,0.004538)(1000,1.057204) +- (-0.004538,0.004538)(1200,1.045715) +- (-0.003776,0.003776)(1400,1.035795) +- (-0.002701,0.002701)(1600,1.030254) +- (-0.002187,0.002187)(1800,1.026719) +- (-0.001900,0.001900)(2000,1.023464) +- (-0.001448,0.001448)(2200,1.021762) +- (-0.001459,0.001459)(2400,1.020202) +- (-0.001358,0.001358)(2600,1.019362) +- (-0.001376,0.001376)(2800,1.018014) +- (-0.001464,0.001464)(3000,1.016429) +- (-0.001048,0.001048)(3200,1.015458) +- (-0.001076,0.001076)(3400,1.014743) +- (-0.000958,0.000958)(3600,1.013626) +- (-0.001025,0.001025)(3800,1.012895) +- (-0.000951,0.000951)(4000,1.012331) +- (-0.000819,0.000819)(4200,1.011388) +- (-0.000744,0.000744)(4400,1.010870) +- (-0.000648,0.000648)(4600,1.010286) +- (-0.000677,0.000677)(4800,1.009729) +- (-0.000613,0.000613)(5000,1.009379) +- (-0.000529,0.000529)(5200,1.009129) +- (-0.000507,0.000507)(5400,1.009011) +- (-0.000476,0.000476)(5600,1.008652) +- (-0.000421,0.000421)(5800,1.008239) +- (-0.000344,0.000344)(6000,1.007777) +- (-0.000316,0.000316)(6200,1.007536) +- (-0.000376,0.000376)(6400,1.007447) +- (-0.000410,0.000410)(6600,1.007189) +- (-0.000457,0.000457)(6800,1.006971) +- (-0.000442,0.000442)(7000,1.006763) +- (-0.000436,0.000436)(7200,1.006541) +- (-0.000441,0.000441)(7400,1.006323) +- (-0.000399,0.000399)(7600,1.006192) +- (-0.000417,0.000417)(7800,1.005997) +- (-0.000360,0.000360)(8000,1.005982) +- (-0.000348,0.000348)(8200,1.005811) +- (-0.000362,0.000362)(8400,1.005675) +- (-0.000344,0.000344)(8600,1.005682) +- (-0.000342,0.000342)(8800,1.005562) +- (-0.000337,0.000337)(9000,1.005469) +- (-0.000341,0.000341)(9200,1.005360) +- (-0.000334,0.000334)(9400,1.005151) +- (-0.000313,0.000313)(9600,1.005067) +- (-0.000297,0.000297)(9800,1.005089) +- (-0.000292,0.000292)(10000,1.005003) +- (-0.000328,0.000328)
};
\addlegendentry{\textsc{Combo-r}}


\addplot [
mark=none,
mark size=1.0pt,
color=col3,
line width=1pt,
opacity=0.7,
%error bars/.cd,
%error bar style={line width=0.2pt},
%y dir=both,
%y explicit
]
coordinates{
(100,1.354438) +- (-0.049676,0.049676)(150,1.261910) +- (-0.043903,0.043903)(200,1.193352) +- (-0.023640,0.023640)(250,1.158053) +- (-0.021180,0.021180)(300,1.124641) +- (-0.017277,0.017277)(350,1.104359) +- (-0.013705,0.013705)(400,1.088457) +- (-0.010286,0.010286)(450,1.080786) +- (-0.008144,0.008144)(500,1.070373) +- (-0.005993,0.005993)(550,1.064068) +- (-0.005595,0.005595)(600,1.055290) +- (-0.004487,0.004487)(650,1.048955) +- (-0.004103,0.004103)(700,1.047470) +- (-0.005201,0.005201)(750,1.045440) +- (-0.004889,0.004889)(800,1.044089) +- (-0.005053,0.005053)(850,1.040761) +- (-0.004259,0.004259)(900,1.039294) +- (-0.003862,0.003862)(950,1.038240) +- (-0.003980,0.003980)(1000,1.036602) +- (-0.004244,0.004244)(1000,1.036602) +- (-0.004244,0.004244)(1200,1.028143) +- (-0.002274,0.002274)(1400,1.023043) +- (-0.001745,0.001745)(1600,1.020388) +- (-0.001699,0.001699)(1800,1.017419) +- (-0.001350,0.001350)(2000,1.014992) +- (-0.001029,0.001029)(2200,1.013166) +- (-0.001057,0.001057)(2400,1.011831) +- (-0.000780,0.000780)(2600,1.010762) +- (-0.000773,0.000773)(2800,1.010364) +- (-0.000756,0.000756)(3000,1.009731) +- (-0.000588,0.000588)(3200,1.009538) +- (-0.000564,0.000564)(3400,1.009158) +- (-0.000598,0.000598)(3600,1.009002) +- (-0.000585,0.000585)(3800,1.008721) +- (-0.000664,0.000664)(4000,1.008040) +- (-0.000623,0.000623)(4200,1.007364) +- (-0.000549,0.000549)(4400,1.007146) +- (-0.000490,0.000490)(4600,1.007188) +- (-0.000510,0.000510)(4800,1.006546) +- (-0.000366,0.000366)(5000,1.006225) +- (-0.000375,0.000375)(5200,1.005990) +- (-0.000362,0.000362)(5400,1.005708) +- (-0.000351,0.000351)(5600,1.005629) +- (-0.000289,0.000289)(5800,1.005278) +- (-0.000268,0.000268)(6000,1.005178) +- (-0.000298,0.000298)(6200,1.005012) +- (-0.000294,0.000294)(6400,1.004922) +- (-0.000290,0.000290)(6600,1.004841) +- (-0.000327,0.000327)(6800,1.004736) +- (-0.000369,0.000369)(7000,1.004649) +- (-0.000320,0.000320)(7200,1.004450) +- (-0.000299,0.000299)(7400,1.004366) +- (-0.000303,0.000303)(7600,1.004133) +- (-0.000285,0.000285)(7800,1.004043) +- (-0.000292,0.000292)(8000,1.003869) +- (-0.000280,0.000280)(8200,1.003778) +- (-0.000265,0.000265)(8400,1.003719) +- (-0.000268,0.000268)(8600,1.003533) +- (-0.000267,0.000267)(8800,1.003494) +- (-0.000261,0.000261)(9000,1.003417) +- (-0.000256,0.000256)(9200,1.003369) +- (-0.000255,0.000255)(9400,1.003313) +- (-0.000261,0.000261)(9600,1.003267) +- (-0.000263,0.000263)(9800,1.003205) +- (-0.000253,0.000253)(10000,1.003078) +- (-0.000229,0.000229)
};
\addlegendentry{\textsc{Combo-i}}


\addplot [
mark=none,
mark size=1.0pt,
color=col4,
line width=1pt,
opacity=0.7,
%error bars/.cd,
%error bar style={line width=0.2pt},
%y dir=both,
%y explicit
]
coordinates{
(100,1.170641) +- (-0.027715,0.027715)(150,1.117583) +- (-0.014676,0.014676)(200,1.096118) +- (-0.010425,0.010425)(250,1.070883) +- (-0.007321,0.007321)(300,1.060162) +- (-0.005966,0.005966)(350,1.048061) +- (-0.004461,0.004461)(400,1.042066) +- (-0.004375,0.004375)(450,1.037266) +- (-0.003616,0.003616)(500,1.033321) +- (-0.002641,0.002641)(550,1.029303) +- (-0.002459,0.002459)(600,1.026878) +- (-0.002678,0.002678)(650,1.025139) +- (-0.001939,0.001939)(700,1.023569) +- (-0.001944,0.001944)(750,1.023765) +- (-0.002575,0.002575)(800,1.022297) +- (-0.002731,0.002731)(850,1.020076) +- (-0.002155,0.002155)(900,1.019601) +- (-0.002252,0.002252)(950,1.018332) +- (-0.002097,0.002097)(1000,1.017969) +- (-0.001808,0.001808)(1000,1.017969) +- (-0.001808,0.001808)(1200,1.014572) +- (-0.001319,0.001319)(1400,1.012313) +- (-0.001092,0.001092)(1600,1.009982) +- (-0.000744,0.000744)(1800,1.009011) +- (-0.000666,0.000666)(2000,1.007777) +- (-0.000569,0.000569)(2200,1.007539) +- (-0.000481,0.000481)(2400,1.006650) +- (-0.000376,0.000376)(2600,1.005992) +- (-0.000335,0.000335)(2800,1.005601) +- (-0.000321,0.000321)(3000,1.005269) +- (-0.000348,0.000348)(3200,1.004877) +- (-0.000312,0.000312)(3400,1.004684) +- (-0.000314,0.000314)(3600,1.004538) +- (-0.000379,0.000379)(3800,1.004325) +- (-0.000328,0.000328)(4000,1.004001) +- (-0.000273,0.000273)(4200,1.003867) +- (-0.000226,0.000226)(4400,1.003703) +- (-0.000228,0.000228)(4600,1.003503) +- (-0.000200,0.000200)(4800,1.003364) +- (-0.000210,0.000210)(5000,1.003156) +- (-0.000207,0.000207)(5200,1.003034) +- (-0.000149,0.000149)(5400,1.002954) +- (-0.000164,0.000164)(5600,1.002849) +- (-0.000144,0.000144)(5800,1.002704) +- (-0.000136,0.000136)(6000,1.002676) +- (-0.000163,0.000163)(6200,1.002598) +- (-0.000171,0.000171)(6400,1.002483) +- (-0.000153,0.000153)(6600,1.002319) +- (-0.000152,0.000152)(6800,1.002266) +- (-0.000137,0.000137)(7000,1.002209) +- (-0.000127,0.000127)(7200,1.002154) +- (-0.000106,0.000106)(7400,1.002083) +- (-0.000099,0.000099)(7600,1.002048) +- (-0.000111,0.000111)(7800,1.001940) +- (-0.000114,0.000114)(8000,1.001904) +- (-0.000114,0.000114)(8200,1.001877) +- (-0.000134,0.000134)(8400,1.001886) +- (-0.000138,0.000138)(8600,1.001794) +- (-0.000116,0.000116)(8800,1.001784) +- (-0.000110,0.000110)(9000,1.001731) +- (-0.000104,0.000104)(9200,1.001675) +- (-0.000100,0.000100)(9400,1.001643) +- (-0.000099,0.000099)(9600,1.001626) +- (-0.000104,0.000104)(9800,1.001598) +- (-0.000100,0.000100)(10000,1.001513) +- (-0.000096,0.000096)
};
\addlegendentry{\textsc{Combo-f}}

\end{axis}
\end{tikzpicture}
    \vspace{\scspacey}
    \caption{\hspace{\scspacex}\textsc{Ising} ($n = 8$)}
    \label{fig:ising8}
  \end{subfigure}\\[0.8em]
  \begin{subfigure}[b]{\subflen}
    \centering
    \begin{tikzpicture}

\colorlet{col1}{blue!50!black}
\colorlet{col2}{lime!50!black}
\colorlet{col3}{red!70!black}
\colorlet{col4}{cyan!50!black}

\begin{axis}[%
tick label style={font=\tiny},
label style={font=\scriptsize},
legend style={font=\tiny},
view={0}{90},
width=\figurewidth,
height=\figureheight,
xmin=0, xmax=5000,
xtick={0, 1000, 2000, 3000, 4000, 5000},
xticklabels={0, 1k, 2k, 3k, 4k, 5k},
scaled x ticks=false,
xlabel={Samples},
xlabel shift=-0.3em,
ymin=1, ymax=1.52,
ytick={1, 1.5},
ylabel={PSRF},
ylabel shift=-1.5em,
tick label style={/pgf/number format/fixed},
major tick length=2pt,
axis lines*=left,
legend cell align=left,
clip marker paths=true,
legend style={at={(1.05,1.05)},draw=none,row sep=-0.35em}]

\addplot [
mark=none,
mark size=1.0pt,
mark options={solid},
color=col1,
densely dashed,
line width=1pt,
opacity=0.7,
%error bars/.cd,
%error bar style={solid, line width=0.2pt},
%y dir=both,
%y explicit
]
coordinates{
(200.000000,3.597939) +- (-0.282843,0.282843)(225.000000,3.040631) +- (-0.197440,0.197440)(250.000000,2.691309) +- (-0.160233,0.160233)(275.000000,2.554452) +- (-0.165238,0.165238)(300.000000,2.413226) +- (-0.160241,0.160241)(325.000000,2.330296) +- (-0.148466,0.148466)(350.000000,2.215460) +- (-0.130796,0.130796)(375.000000,2.169512) +- (-0.130263,0.130263)(400.000000,2.082272) +- (-0.130343,0.130343)(425.000000,2.028475) +- (-0.144012,0.144012)(450.000000,2.014070) +- (-0.189635,0.189635)(475.000000,2.059465) +- (-0.282843,0.282843)(500.000000,1.948577) +- (-0.207447,0.207447)(525.000000,1.888214) +- (-0.187895,0.187895)(550.000000,1.800753) +- (-0.146235,0.146235)(575.000000,1.749965) +- (-0.125414,0.125414)(600.000000,1.717381) +- (-0.149604,0.149604)(625.000000,1.704961) +- (-0.179373,0.179373)(650.000000,1.685182) +- (-0.188536,0.188536)(675.000000,1.651440) +- (-0.194696,0.194696)(700.000000,1.565033) +- (-0.110935,0.110935)(725.000000,1.494971) +- (-0.060617,0.060617)(750.000000,1.448536) +- (-0.037580,0.037580)(775.000000,1.423904) +- (-0.030389,0.030389)(800.000000,1.404557) +- (-0.025724,0.025724)(825.000000,1.388615) +- (-0.024165,0.024165)(850.000000,1.375496) +- (-0.022723,0.022723)(875.000000,1.365812) +- (-0.022431,0.022431)(900.000000,1.359307) +- (-0.022667,0.022667)(925.000000,1.357107) +- (-0.023374,0.023374)(950.000000,1.353670) +- (-0.024974,0.024974)(975.000000,1.349603) +- (-0.026614,0.026614)(1000.000000,1.341914) +- (-0.026115,0.026115)(1000.000000,1.341914) +- (-0.026115,0.026115)(1050.000000,1.327010) +- (-0.025499,0.025499)(1100.000000,1.309038) +- (-0.022776,0.022776)(1150.000000,1.291903) +- (-0.021510,0.021510)(1200.000000,1.276404) +- (-0.019627,0.019627)(1250.000000,1.265232) +- (-0.019791,0.019791)(1300.000000,1.257915) +- (-0.022160,0.022160)(1350.000000,1.245602) +- (-0.022382,0.022382)(1400.000000,1.229363) +- (-0.019007,0.019007)(1450.000000,1.212160) +- (-0.014853,0.014853)(1500.000000,1.202877) +- (-0.014921,0.014921)(1550.000000,1.194305) +- (-0.016019,0.016019)(1600.000000,1.187061) +- (-0.016566,0.016566)(1650.000000,1.181264) +- (-0.015779,0.015779)(1700.000000,1.176979) +- (-0.015684,0.015684)(1750.000000,1.173301) +- (-0.016778,0.016778)(1800.000000,1.166561) +- (-0.015625,0.015625)(1850.000000,1.160849) +- (-0.014327,0.014327)(1900.000000,1.157492) +- (-0.013240,0.013240)(1950.000000,1.152301) +- (-0.012648,0.012648)(2000.000000,1.146251) +- (-0.010389,0.010389)(2050.000000,1.142146) +- (-0.008755,0.008755)(2100.000000,1.138389) +- (-0.008279,0.008279)(2150.000000,1.134830) +- (-0.007781,0.007781)(2200.000000,1.131334) +- (-0.007385,0.007385)(2250.000000,1.129364) +- (-0.007120,0.007120)(2300.000000,1.127263) +- (-0.007972,0.007972)(2350.000000,1.124395) +- (-0.008520,0.008520)(2400.000000,1.119219) +- (-0.008856,0.008856)(2450.000000,1.117366) +- (-0.010033,0.010033)(2500.000000,1.114105) +- (-0.008819,0.008819)(2550.000000,1.113725) +- (-0.008468,0.008468)(2600.000000,1.111648) +- (-0.008166,0.008166)(2650.000000,1.109619) +- (-0.007836,0.007836)(2700.000000,1.107001) +- (-0.007309,0.007309)(2750.000000,1.104818) +- (-0.006968,0.006968)(2800.000000,1.104075) +- (-0.006324,0.006324)(2850.000000,1.102976) +- (-0.006398,0.006398)(2900.000000,1.100722) +- (-0.006288,0.006288)(2950.000000,1.097873) +- (-0.005578,0.005578)(3000.000000,1.095111) +- (-0.005033,0.005033)(3050.000000,1.093478) +- (-0.005158,0.005158)(3100.000000,1.091610) +- (-0.005177,0.005177)(3150.000000,1.090405) +- (-0.005001,0.005001)(3200.000000,1.089527) +- (-0.004982,0.004982)(3250.000000,1.088855) +- (-0.004821,0.004821)(3300.000000,1.087810) +- (-0.004819,0.004819)(3350.000000,1.086178) +- (-0.004890,0.004890)(3400.000000,1.083872) +- (-0.004823,0.004823)(3450.000000,1.082651) +- (-0.004757,0.004757)(3500.000000,1.082000) +- (-0.004411,0.004411)(3550.000000,1.081654) +- (-0.004404,0.004404)(3600.000000,1.080664) +- (-0.004261,0.004261)(3650.000000,1.079681) +- (-0.004000,0.004000)(3700.000000,1.078145) +- (-0.004157,0.004157)(3750.000000,1.077135) +- (-0.004285,0.004285)(3800.000000,1.076356) +- (-0.004383,0.004383)(3850.000000,1.075704) +- (-0.004286,0.004286)(3900.000000,1.074393) +- (-0.004038,0.004038)(3950.000000,1.072686) +- (-0.003808,0.003808)(4000.000000,1.071898) +- (-0.003671,0.003671)(4050.000000,1.070577) +- (-0.003569,0.003569)(4100.000000,1.068997) +- (-0.003373,0.003373)(4150.000000,1.067786) +- (-0.003359,0.003359)(4200.000000,1.067305) +- (-0.003437,0.003437)(4250.000000,1.066197) +- (-0.003413,0.003413)(4300.000000,1.064406) +- (-0.003289,0.003289)(4350.000000,1.062878) +- (-0.003119,0.003119)(4400.000000,1.061796) +- (-0.003118,0.003118)(4450.000000,1.061438) +- (-0.003328,0.003328)(4500.000000,1.061019) +- (-0.003352,0.003352)(4550.000000,1.060327) +- (-0.003275,0.003275)(4600.000000,1.059904) +- (-0.003248,0.003248)(4650.000000,1.059435) +- (-0.003126,0.003126)(4700.000000,1.058691) +- (-0.002941,0.002941)(4750.000000,1.057888) +- (-0.002763,0.002763)(4800.000000,1.057200) +- (-0.002759,0.002759)(4850.000000,1.056826) +- (-0.002693,0.002693)(4900.000000,1.056174) +- (-0.002721,0.002721)(4950.000000,1.055308) +- (-0.002654,0.002654)(5000.000000,1.054398) +- (-0.002602,0.002602)
};
\addlegendentry{\textsc{Gibbs}}


\addplot [
mark=none,
mark size=1.0pt,
color=col2,
line width=1pt,
opacity=0.7,
%error bars/.cd,
%error bar style={line width=0.2pt},
%y dir=both,
%y explicit
]
coordinates{
(75.000000,3.200882) +- (-0.282843,0.282843)(100.000000,2.314380) +- (-0.145936,0.145936)(125.000000,2.004462) +- (-0.127423,0.127423)(150.000000,1.793518) +- (-0.080239,0.080239)(175.000000,1.631944) +- (-0.067338,0.067338)(200.000000,1.547655) +- (-0.050816,0.050816)(225.000000,1.467467) +- (-0.045281,0.045281)(250.000000,1.434199) +- (-0.047724,0.047724)(275.000000,1.387869) +- (-0.026698,0.026698)(300.000000,1.352802) +- (-0.024894,0.024894)(325.000000,1.325987) +- (-0.022519,0.022519)(350.000000,1.300620) +- (-0.018924,0.018924)(375.000000,1.276801) +- (-0.015945,0.015945)(400.000000,1.258876) +- (-0.015658,0.015658)(425.000000,1.247336) +- (-0.015910,0.015910)(450.000000,1.231760) +- (-0.014686,0.014686)(475.000000,1.220108) +- (-0.014804,0.014804)(500.000000,1.209891) +- (-0.016687,0.016687)(525.000000,1.197965) +- (-0.017494,0.017494)(550.000000,1.187772) +- (-0.017134,0.017134)(575.000000,1.181613) +- (-0.016117,0.016117)(600.000000,1.175375) +- (-0.015438,0.015438)(625.000000,1.167047) +- (-0.014963,0.014963)(650.000000,1.162951) +- (-0.014282,0.014282)(675.000000,1.159416) +- (-0.013639,0.013639)(700.000000,1.151083) +- (-0.012802,0.012802)(725.000000,1.143855) +- (-0.013366,0.013366)(750.000000,1.139644) +- (-0.013836,0.013836)(775.000000,1.136401) +- (-0.012880,0.012880)(800.000000,1.130305) +- (-0.011110,0.011110)(825.000000,1.127232) +- (-0.010790,0.010790)(850.000000,1.121431) +- (-0.010224,0.010224)(875.000000,1.116844) +- (-0.009764,0.009764)(900.000000,1.113843) +- (-0.009180,0.009180)(925.000000,1.109543) +- (-0.008657,0.008657)(950.000000,1.105246) +- (-0.008078,0.008078)(975.000000,1.101941) +- (-0.007431,0.007431)(1000.000000,1.097746) +- (-0.006427,0.006427)(1000.000000,1.097746) +- (-0.006427,0.006427)(1050.000000,1.090858) +- (-0.005703,0.005703)(1100.000000,1.085136) +- (-0.004712,0.004712)(1150.000000,1.082785) +- (-0.004480,0.004480)(1200.000000,1.078993) +- (-0.004481,0.004481)(1250.000000,1.075416) +- (-0.004278,0.004278)(1300.000000,1.072572) +- (-0.004408,0.004408)(1350.000000,1.072164) +- (-0.004253,0.004253)(1400.000000,1.071047) +- (-0.003945,0.003945)(1450.000000,1.068267) +- (-0.003995,0.003995)(1500.000000,1.064680) +- (-0.003782,0.003782)(1550.000000,1.062878) +- (-0.003607,0.003607)(1600.000000,1.060127) +- (-0.002978,0.002978)(1650.000000,1.057291) +- (-0.002852,0.002852)(1700.000000,1.056389) +- (-0.002972,0.002972)(1750.000000,1.054495) +- (-0.002850,0.002850)(1800.000000,1.052421) +- (-0.002805,0.002805)(1850.000000,1.052303) +- (-0.003047,0.003047)(1900.000000,1.051432) +- (-0.002882,0.002882)(1950.000000,1.049071) +- (-0.002860,0.002860)(2000.000000,1.048121) +- (-0.002986,0.002986)(2050.000000,1.046968) +- (-0.002974,0.002974)(2100.000000,1.046242) +- (-0.002929,0.002929)(2150.000000,1.046152) +- (-0.002783,0.002783)(2200.000000,1.045838) +- (-0.002628,0.002628)(2250.000000,1.044434) +- (-0.002501,0.002501)(2300.000000,1.043060) +- (-0.002611,0.002611)(2350.000000,1.042289) +- (-0.002473,0.002473)(2400.000000,1.041161) +- (-0.002194,0.002194)(2450.000000,1.040078) +- (-0.002039,0.002039)(2500.000000,1.038894) +- (-0.001963,0.001963)(2550.000000,1.037761) +- (-0.002019,0.002019)(2600.000000,1.036455) +- (-0.001902,0.001902)(2650.000000,1.035295) +- (-0.001652,0.001652)(2700.000000,1.034810) +- (-0.001550,0.001550)(2750.000000,1.034331) +- (-0.001533,0.001533)(2800.000000,1.033830) +- (-0.001393,0.001393)(2850.000000,1.032976) +- (-0.001362,0.001362)(2900.000000,1.032553) +- (-0.001368,0.001368)(2950.000000,1.031991) +- (-0.001444,0.001444)(3000.000000,1.031694) +- (-0.001432,0.001432)(3050.000000,1.031078) +- (-0.001410,0.001410)(3100.000000,1.030946) +- (-0.001536,0.001536)(3150.000000,1.030691) +- (-0.001492,0.001492)(3200.000000,1.030606) +- (-0.001415,0.001415)(3250.000000,1.030274) +- (-0.001521,0.001521)(3300.000000,1.029901) +- (-0.001567,0.001567)(3350.000000,1.029501) +- (-0.001596,0.001596)(3400.000000,1.029197) +- (-0.001490,0.001490)(3450.000000,1.028674) +- (-0.001432,0.001432)(3500.000000,1.028292) +- (-0.001420,0.001420)(3550.000000,1.027638) +- (-0.001465,0.001465)(3600.000000,1.027133) +- (-0.001453,0.001453)(3650.000000,1.027022) +- (-0.001433,0.001433)(3700.000000,1.026421) +- (-0.001474,0.001474)(3750.000000,1.025866) +- (-0.001395,0.001395)(3800.000000,1.025313) +- (-0.001397,0.001397)(3850.000000,1.025126) +- (-0.001384,0.001384)(3900.000000,1.024998) +- (-0.001477,0.001477)(3950.000000,1.024660) +- (-0.001446,0.001446)(4000.000000,1.024171) +- (-0.001362,0.001362)(4050.000000,1.023864) +- (-0.001253,0.001253)(4100.000000,1.023532) +- (-0.001167,0.001167)(4150.000000,1.023254) +- (-0.001154,0.001154)(4200.000000,1.023072) +- (-0.001145,0.001145)(4250.000000,1.022695) +- (-0.001132,0.001132)(4300.000000,1.022507) +- (-0.001124,0.001124)(4350.000000,1.022286) +- (-0.001143,0.001143)(4400.000000,1.021894) +- (-0.001132,0.001132)(4450.000000,1.021687) +- (-0.001104,0.001104)(4500.000000,1.021078) +- (-0.000988,0.000988)(4550.000000,1.020705) +- (-0.000968,0.000968)(4600.000000,1.020400) +- (-0.000900,0.000900)(4650.000000,1.020334) +- (-0.000907,0.000907)(4700.000000,1.020393) +- (-0.000906,0.000906)(4750.000000,1.020249) +- (-0.000837,0.000837)(4800.000000,1.020091) +- (-0.000844,0.000844)(4850.000000,1.019827) +- (-0.000798,0.000798)(4900.000000,1.019461) +- (-0.000836,0.000836)(4950.000000,1.019257) +- (-0.000829,0.000829)(5000.000000,1.019229) +- (-0.000875,0.000875)
};
\addlegendentry{\textsc{Combo-r}}


\addplot [
mark=none,
mark size=1.0pt,
color=col3,
line width=1pt,
opacity=0.7,
%error bars/.cd,
%error bar style={line width=0.2pt},
%y dir=both,
%y explicit
]
coordinates{
(50.000000,3.854406) +- (-0.282843,0.282843)(75.000000,2.516837) +- (-0.198273,0.198273)(100.000000,2.080291) +- (-0.204548,0.204548)(125.000000,1.761562) +- (-0.101370,0.101370)(150.000000,1.588895) +- (-0.068315,0.068315)(175.000000,1.530598) +- (-0.077335,0.077335)(200.000000,1.472095) +- (-0.088279,0.088279)(225.000000,1.407313) +- (-0.069317,0.069317)(250.000000,1.328349) +- (-0.050041,0.050041)(275.000000,1.280189) +- (-0.046740,0.046740)(300.000000,1.241093) +- (-0.028634,0.028634)(325.000000,1.225659) +- (-0.018168,0.018168)(350.000000,1.214143) +- (-0.016083,0.016083)(375.000000,1.199464) +- (-0.015557,0.015557)(400.000000,1.192993) +- (-0.017602,0.017602)(425.000000,1.182354) +- (-0.016598,0.016598)(450.000000,1.173964) +- (-0.016687,0.016687)(475.000000,1.162564) +- (-0.013994,0.013994)(500.000000,1.153207) +- (-0.012485,0.012485)(525.000000,1.146027) +- (-0.011376,0.011376)(550.000000,1.138410) +- (-0.011042,0.011042)(575.000000,1.132240) +- (-0.010125,0.010125)(600.000000,1.124508) +- (-0.009295,0.009295)(625.000000,1.120885) +- (-0.010133,0.010133)(650.000000,1.114675) +- (-0.009175,0.009175)(675.000000,1.110488) +- (-0.008958,0.008958)(700.000000,1.106881) +- (-0.007907,0.007907)(725.000000,1.104418) +- (-0.007560,0.007560)(750.000000,1.099178) +- (-0.007433,0.007433)(775.000000,1.094869) +- (-0.007824,0.007824)(800.000000,1.092353) +- (-0.007203,0.007203)(825.000000,1.089835) +- (-0.006443,0.006443)(850.000000,1.086294) +- (-0.005841,0.005841)(875.000000,1.083016) +- (-0.004871,0.004871)(900.000000,1.080625) +- (-0.004230,0.004230)(925.000000,1.078600) +- (-0.003971,0.003971)(950.000000,1.075129) +- (-0.003631,0.003631)(975.000000,1.073524) +- (-0.003559,0.003559)(1000.000000,1.071891) +- (-0.003385,0.003385)(1000.000000,1.071891) +- (-0.003385,0.003385)(1050.000000,1.066995) +- (-0.003261,0.003261)(1100.000000,1.064854) +- (-0.003031,0.003031)(1150.000000,1.062135) +- (-0.003158,0.003158)(1200.000000,1.060072) +- (-0.002924,0.002924)(1250.000000,1.058380) +- (-0.002652,0.002652)(1300.000000,1.055289) +- (-0.003135,0.003135)(1350.000000,1.053504) +- (-0.002473,0.002473)(1400.000000,1.051133) +- (-0.002215,0.002215)(1450.000000,1.049357) +- (-0.002129,0.002129)(1500.000000,1.048042) +- (-0.002443,0.002443)(1550.000000,1.046165) +- (-0.002292,0.002292)(1600.000000,1.044985) +- (-0.002335,0.002335)(1650.000000,1.043813) +- (-0.002302,0.002302)(1700.000000,1.042142) +- (-0.002126,0.002126)(1750.000000,1.040972) +- (-0.001732,0.001732)(1800.000000,1.038792) +- (-0.001605,0.001605)(1850.000000,1.037901) +- (-0.001482,0.001482)(1900.000000,1.037031) +- (-0.001745,0.001745)(1950.000000,1.036926) +- (-0.002238,0.002238)(2000.000000,1.036016) +- (-0.002091,0.002091)(2050.000000,1.034859) +- (-0.002239,0.002239)(2100.000000,1.033716) +- (-0.002164,0.002164)(2150.000000,1.032822) +- (-0.001992,0.001992)(2200.000000,1.031838) +- (-0.001809,0.001809)(2250.000000,1.030816) +- (-0.001774,0.001774)(2300.000000,1.030378) +- (-0.001692,0.001692)(2350.000000,1.030052) +- (-0.001777,0.001777)(2400.000000,1.029630) +- (-0.001723,0.001723)(2450.000000,1.028859) +- (-0.001648,0.001648)(2500.000000,1.028342) +- (-0.001664,0.001664)(2550.000000,1.027460) +- (-0.001711,0.001711)(2600.000000,1.026793) +- (-0.001637,0.001637)(2650.000000,1.026026) +- (-0.001552,0.001552)(2700.000000,1.025250) +- (-0.001265,0.001265)(2750.000000,1.025001) +- (-0.001289,0.001289)(2800.000000,1.024221) +- (-0.001330,0.001330)(2850.000000,1.023217) +- (-0.001235,0.001235)(2900.000000,1.023087) +- (-0.001262,0.001262)(2950.000000,1.022697) +- (-0.001187,0.001187)(3000.000000,1.022573) +- (-0.001180,0.001180)(3050.000000,1.022419) +- (-0.001269,0.001269)(3100.000000,1.022147) +- (-0.001212,0.001212)(3150.000000,1.021630) +- (-0.001206,0.001206)(3200.000000,1.021636) +- (-0.001243,0.001243)(3250.000000,1.021386) +- (-0.001207,0.001207)(3300.000000,1.021237) +- (-0.001212,0.001212)(3350.000000,1.020897) +- (-0.001196,0.001196)(3400.000000,1.020573) +- (-0.001154,0.001154)(3450.000000,1.020276) +- (-0.001089,0.001089)(3500.000000,1.019911) +- (-0.001184,0.001184)(3550.000000,1.019570) +- (-0.001224,0.001224)(3600.000000,1.019220) +- (-0.001139,0.001139)(3650.000000,1.018847) +- (-0.001054,0.001054)(3700.000000,1.018390) +- (-0.000911,0.000911)(3750.000000,1.018087) +- (-0.000910,0.000910)(3800.000000,1.018215) +- (-0.001009,0.001009)(3850.000000,1.017975) +- (-0.000981,0.000981)(3900.000000,1.017773) +- (-0.000918,0.000918)(3950.000000,1.017623) +- (-0.000880,0.000880)(4000.000000,1.017201) +- (-0.000808,0.000808)(4050.000000,1.017108) +- (-0.000833,0.000833)(4100.000000,1.016947) +- (-0.000744,0.000744)(4150.000000,1.016976) +- (-0.000757,0.000757)(4200.000000,1.016918) +- (-0.000846,0.000846)(4250.000000,1.016655) +- (-0.000857,0.000857)(4300.000000,1.016267) +- (-0.000828,0.000828)(4350.000000,1.016105) +- (-0.000774,0.000774)(4400.000000,1.015787) +- (-0.000799,0.000799)(4450.000000,1.015690) +- (-0.000837,0.000837)(4500.000000,1.015584) +- (-0.000785,0.000785)(4550.000000,1.015477) +- (-0.000837,0.000837)(4600.000000,1.015410) +- (-0.000880,0.000880)(4650.000000,1.015213) +- (-0.000865,0.000865)(4700.000000,1.014935) +- (-0.000832,0.000832)(4750.000000,1.014694) +- (-0.000838,0.000838)(4800.000000,1.014629) +- (-0.000778,0.000778)(4850.000000,1.014533) +- (-0.000751,0.000751)(4900.000000,1.014393) +- (-0.000742,0.000742)(4950.000000,1.014260) +- (-0.000782,0.000782)(5000.000000,1.014056) +- (-0.000720,0.000720)
};
\addlegendentry{\textsc{Combo-i}}

\end{axis}
\end{tikzpicture}

    \vspace{\scspacey}
    \caption{\hspace{\scspacex}\textsc{Water}}
    \label{fig:water1}
  \end{subfigure}
  \begin{subfigure}[b]{\subflen}
    \begin{tikzpicture}


\begin{axis}[%
tick label style={/pgf/number format/fixed,font=\sffamily\small},
label style={font=\sffamily\small},
legend style={font=\sffamily\small},
view={0}{90},
width=\figurewidth,
height=\figureheight,
xmin=0, xmax=5000,
xtick={0, 1000, 2000, 3000, 4000, 5000},
xticklabels={0, 1k, 2k, 3k, 4k, 5k},
scaled x ticks=false,
xlabel={Samples},
xlabel shift=0em,
ymin=1, ymax=1.52,
ytick={1, 1.5},
yticklabels={1, 1.5},
ylabel={PSRF},
ylabel shift=-1em,
major tick length=2pt,
axis lines*=left,
legend cell align=left,
clip marker paths=true,
legend style={anchor=north east,at={(1,1)},draw=none,row sep=0em},
every axis plot/.append style={
  line width=1.5pt,
  opacity=0.8,
}
]

\addplot [
color=col1dark,
densely dashed
]
coordinates{
(200.000000,5.365060) +- (-0.282843,0.282843)(225.000000,4.591755) +- (-0.282843,0.282843)(250.000000,4.574076) +- (-0.282843,0.282843)(275.000000,3.388613) +- (-0.281393,0.281393)(300.000000,3.076058) +- (-0.191520,0.191520)(325.000000,2.822256) +- (-0.146702,0.146702)(350.000000,2.607563) +- (-0.141973,0.141973)(375.000000,2.469923) +- (-0.175838,0.175838)(400.000000,2.410058) +- (-0.229741,0.229741)(425.000000,2.331485) +- (-0.235140,0.235140)(450.000000,2.216657) +- (-0.142948,0.142948)(475.000000,2.086454) +- (-0.098302,0.098302)(500.000000,1.948136) +- (-0.071483,0.071483)(525.000000,1.888816) +- (-0.060779,0.060779)(550.000000,1.843957) +- (-0.057140,0.057140)(575.000000,1.794805) +- (-0.058276,0.058276)(600.000000,1.738106) +- (-0.055554,0.055554)(625.000000,1.699625) +- (-0.050800,0.050800)(650.000000,1.668680) +- (-0.049063,0.049063)(675.000000,1.626034) +- (-0.051438,0.051438)(700.000000,1.604499) +- (-0.052825,0.052825)(725.000000,1.584124) +- (-0.056126,0.056126)(750.000000,1.549530) +- (-0.060329,0.060329)(775.000000,1.514791) +- (-0.060024,0.060024)(800.000000,1.487549) +- (-0.048743,0.048743)(825.000000,1.468202) +- (-0.040792,0.040792)(850.000000,1.444602) +- (-0.033844,0.033844)(875.000000,1.428660) +- (-0.030034,0.030034)(900.000000,1.415871) +- (-0.028287,0.028287)(925.000000,1.400190) +- (-0.027452,0.027452)(950.000000,1.389371) +- (-0.028086,0.028086)(975.000000,1.382080) +- (-0.028082,0.028082)(1000.000000,1.371028) +- (-0.026612,0.026612)(1000.000000,1.371028) +- (-0.026612,0.026612)(1050.000000,1.341921) +- (-0.019217,0.019217)(1100.000000,1.321902) +- (-0.017345,0.017345)(1150.000000,1.310393) +- (-0.016518,0.016518)(1200.000000,1.294551) +- (-0.016810,0.016810)(1250.000000,1.283851) +- (-0.017795,0.017795)(1300.000000,1.279615) +- (-0.018934,0.018934)(1350.000000,1.270379) +- (-0.017502,0.017502)(1400.000000,1.266164) +- (-0.015604,0.015604)(1450.000000,1.257723) +- (-0.014725,0.014725)(1500.000000,1.249316) +- (-0.016415,0.016415)(1550.000000,1.242407) +- (-0.018318,0.018318)(1600.000000,1.234152) +- (-0.017874,0.017874)(1650.000000,1.221854) +- (-0.016611,0.016611)(1700.000000,1.211019) +- (-0.014424,0.014424)(1750.000000,1.202812) +- (-0.013257,0.013257)(1800.000000,1.193349) +- (-0.012043,0.012043)(1850.000000,1.184444) +- (-0.011020,0.011020)(1900.000000,1.180122) +- (-0.009303,0.009303)(1950.000000,1.175485) +- (-0.009021,0.009021)(2000.000000,1.173358) +- (-0.009061,0.009061)(2050.000000,1.169905) +- (-0.009030,0.009030)(2100.000000,1.167314) +- (-0.008256,0.008256)(2150.000000,1.163335) +- (-0.007706,0.007706)(2200.000000,1.157854) +- (-0.007539,0.007539)(2250.000000,1.150624) +- (-0.006815,0.006815)(2300.000000,1.143648) +- (-0.005839,0.005839)(2350.000000,1.140425) +- (-0.005419,0.005419)(2400.000000,1.136364) +- (-0.005506,0.005506)(2450.000000,1.133870) +- (-0.005647,0.005647)(2500.000000,1.130276) +- (-0.005497,0.005497)(2550.000000,1.126294) +- (-0.004847,0.004847)(2600.000000,1.122550) +- (-0.004747,0.004747)(2650.000000,1.120367) +- (-0.005175,0.005175)(2700.000000,1.119356) +- (-0.005542,0.005542)(2750.000000,1.117717) +- (-0.005798,0.005798)(2800.000000,1.115643) +- (-0.005779,0.005779)(2850.000000,1.113188) +- (-0.005633,0.005633)(2900.000000,1.112326) +- (-0.005878,0.005878)(2950.000000,1.109360) +- (-0.006035,0.006035)(3000.000000,1.106965) +- (-0.005931,0.005931)(3050.000000,1.105424) +- (-0.005750,0.005750)(3100.000000,1.104043) +- (-0.005513,0.005513)(3150.000000,1.102278) +- (-0.005122,0.005122)(3200.000000,1.100337) +- (-0.004520,0.004520)(3250.000000,1.098972) +- (-0.004164,0.004164)(3300.000000,1.097375) +- (-0.004132,0.004132)(3350.000000,1.095626) +- (-0.004271,0.004271)(3400.000000,1.094502) +- (-0.004092,0.004092)(3450.000000,1.093405) +- (-0.003681,0.003681)(3500.000000,1.092404) +- (-0.003911,0.003911)(3550.000000,1.091018) +- (-0.004084,0.004084)(3600.000000,1.089621) +- (-0.003939,0.003939)(3650.000000,1.088190) +- (-0.004078,0.004078)(3700.000000,1.086834) +- (-0.004009,0.004009)(3750.000000,1.086578) +- (-0.004154,0.004154)(3800.000000,1.085939) +- (-0.004251,0.004251)(3850.000000,1.085046) +- (-0.004105,0.004105)(3900.000000,1.083951) +- (-0.003726,0.003726)(3950.000000,1.081714) +- (-0.003366,0.003366)(4000.000000,1.079780) +- (-0.003043,0.003043)(4050.000000,1.078333) +- (-0.003047,0.003047)(4100.000000,1.077290) +- (-0.003056,0.003056)(4150.000000,1.076659) +- (-0.003096,0.003096)(4200.000000,1.075023) +- (-0.003116,0.003116)(4250.000000,1.073408) +- (-0.002976,0.002976)(4300.000000,1.072610) +- (-0.002824,0.002824)(4350.000000,1.071975) +- (-0.002715,0.002715)(4400.000000,1.070756) +- (-0.002601,0.002601)(4450.000000,1.069757) +- (-0.002691,0.002691)(4500.000000,1.069134) +- (-0.002538,0.002538)(4550.000000,1.068535) +- (-0.002489,0.002489)(4600.000000,1.068012) +- (-0.002508,0.002508)(4650.000000,1.067101) +- (-0.002472,0.002472)(4700.000000,1.066104) +- (-0.002306,0.002306)(4750.000000,1.065418) +- (-0.002099,0.002099)(4800.000000,1.065554) +- (-0.002185,0.002185)(4850.000000,1.065546) +- (-0.002261,0.002261)(4900.000000,1.065401) +- (-0.002325,0.002325)(4950.000000,1.065132) +- (-0.002230,0.002230)(5000.000000,1.064613) +- (-0.002087,0.002087)
};
\addlegendentry{\textsc{Gibbs}}


\addplot [
color=col2
]
coordinates{
(150.000000,2.861017) +- (-0.189265,0.189265)(175.000000,2.284893) +- (-0.132961,0.132961)(200.000000,1.931278) +- (-0.095620,0.095620)(225.000000,1.702540) +- (-0.089883,0.089883)(250.000000,1.525468) +- (-0.074378,0.074378)(275.000000,1.417393) +- (-0.059349,0.059349)(300.000000,1.356837) +- (-0.044414,0.044414)(325.000000,1.309773) +- (-0.043224,0.043224)(350.000000,1.267663) +- (-0.034687,0.034687)(375.000000,1.237023) +- (-0.024927,0.024927)(400.000000,1.218599) +- (-0.021292,0.021292)(425.000000,1.210187) +- (-0.018530,0.018530)(450.000000,1.197193) +- (-0.016534,0.016534)(475.000000,1.188801) +- (-0.017027,0.017027)(500.000000,1.182191) +- (-0.016036,0.016036)(525.000000,1.177181) +- (-0.017304,0.017304)(550.000000,1.164126) +- (-0.016521,0.016521)(575.000000,1.150835) +- (-0.015308,0.015308)(600.000000,1.142566) +- (-0.013899,0.013899)(625.000000,1.139221) +- (-0.013510,0.013510)(650.000000,1.129109) +- (-0.011488,0.011488)(675.000000,1.122447) +- (-0.010617,0.010617)(700.000000,1.115712) +- (-0.009125,0.009125)(725.000000,1.109586) +- (-0.008040,0.008040)(750.000000,1.105055) +- (-0.007006,0.007006)(775.000000,1.102083) +- (-0.006682,0.006682)(800.000000,1.098686) +- (-0.006089,0.006089)(825.000000,1.096120) +- (-0.005320,0.005320)(850.000000,1.092733) +- (-0.005291,0.005291)(875.000000,1.089636) +- (-0.005873,0.005873)(900.000000,1.087381) +- (-0.005717,0.005717)(925.000000,1.084442) +- (-0.005548,0.005548)(950.000000,1.080266) +- (-0.004843,0.004843)(975.000000,1.077660) +- (-0.004463,0.004463)(1000.000000,1.076389) +- (-0.003999,0.003999)(1000.000000,1.076389) +- (-0.003999,0.003999)(1050.000000,1.074628) +- (-0.004189,0.004189)(1100.000000,1.070641) +- (-0.003513,0.003513)(1150.000000,1.067539) +- (-0.003267,0.003267)(1200.000000,1.066073) +- (-0.002745,0.002745)(1250.000000,1.063376) +- (-0.003073,0.003073)(1300.000000,1.060532) +- (-0.003668,0.003668)(1350.000000,1.058302) +- (-0.003239,0.003239)(1400.000000,1.054618) +- (-0.002982,0.002982)(1450.000000,1.051791) +- (-0.002411,0.002411)(1500.000000,1.049556) +- (-0.002277,0.002277)(1550.000000,1.047740) +- (-0.002597,0.002597)(1600.000000,1.045233) +- (-0.002502,0.002502)(1650.000000,1.043926) +- (-0.002368,0.002368)(1700.000000,1.044117) +- (-0.002562,0.002562)(1750.000000,1.043331) +- (-0.002555,0.002555)(1800.000000,1.042255) +- (-0.002460,0.002460)(1850.000000,1.041731) +- (-0.002406,0.002406)(1900.000000,1.041156) +- (-0.002283,0.002283)(1950.000000,1.040400) +- (-0.002219,0.002219)(2000.000000,1.039378) +- (-0.001978,0.001978)(2050.000000,1.037770) +- (-0.001830,0.001830)(2100.000000,1.037051) +- (-0.001736,0.001736)(2150.000000,1.036132) +- (-0.001768,0.001768)(2200.000000,1.034605) +- (-0.001640,0.001640)(2250.000000,1.034023) +- (-0.001590,0.001590)(2300.000000,1.032596) +- (-0.001523,0.001523)(2350.000000,1.031656) +- (-0.001373,0.001373)(2400.000000,1.031576) +- (-0.001311,0.001311)(2450.000000,1.031104) +- (-0.001292,0.001292)(2500.000000,1.030614) +- (-0.001104,0.001104)(2550.000000,1.029717) +- (-0.001207,0.001207)(2600.000000,1.028907) +- (-0.001217,0.001217)(2650.000000,1.028260) +- (-0.001234,0.001234)(2700.000000,1.027780) +- (-0.001273,0.001273)(2750.000000,1.027333) +- (-0.001296,0.001296)(2800.000000,1.026687) +- (-0.001271,0.001271)(2850.000000,1.026108) +- (-0.001166,0.001166)(2900.000000,1.025506) +- (-0.000982,0.000982)(2950.000000,1.025128) +- (-0.001109,0.001109)(3000.000000,1.024443) +- (-0.001066,0.001066)(3050.000000,1.023717) +- (-0.000975,0.000975)(3100.000000,1.023429) +- (-0.001033,0.001033)(3150.000000,1.023237) +- (-0.001027,0.001027)(3200.000000,1.022865) +- (-0.001080,0.001080)(3250.000000,1.022283) +- (-0.000973,0.000973)(3300.000000,1.022223) +- (-0.000991,0.000991)(3350.000000,1.022119) +- (-0.000957,0.000957)(3400.000000,1.021777) +- (-0.000895,0.000895)(3450.000000,1.021490) +- (-0.000785,0.000785)(3500.000000,1.021050) +- (-0.000815,0.000815)(3550.000000,1.020415) +- (-0.000891,0.000891)(3600.000000,1.020045) +- (-0.000861,0.000861)(3650.000000,1.019814) +- (-0.000793,0.000793)(3700.000000,1.019478) +- (-0.000809,0.000809)(3750.000000,1.019212) +- (-0.000827,0.000827)(3800.000000,1.019139) +- (-0.000814,0.000814)(3850.000000,1.018684) +- (-0.000893,0.000893)(3900.000000,1.018688) +- (-0.000899,0.000899)(3950.000000,1.018489) +- (-0.000940,0.000940)(4000.000000,1.018422) +- (-0.000918,0.000918)(4050.000000,1.018251) +- (-0.000921,0.000921)(4100.000000,1.018088) +- (-0.000869,0.000869)(4150.000000,1.017766) +- (-0.000908,0.000908)(4200.000000,1.017707) +- (-0.000927,0.000927)(4250.000000,1.017635) +- (-0.000922,0.000922)(4300.000000,1.017451) +- (-0.000944,0.000944)(4350.000000,1.016993) +- (-0.000910,0.000910)(4400.000000,1.016784) +- (-0.000848,0.000848)(4450.000000,1.016977) +- (-0.000884,0.000884)(4500.000000,1.016812) +- (-0.000866,0.000866)(4550.000000,1.016552) +- (-0.000828,0.000828)(4600.000000,1.016506) +- (-0.000839,0.000839)(4650.000000,1.016415) +- (-0.000812,0.000812)(4700.000000,1.016248) +- (-0.000774,0.000774)(4750.000000,1.015931) +- (-0.000721,0.000721)(4800.000000,1.015712) +- (-0.000696,0.000696)(4850.000000,1.015534) +- (-0.000704,0.000704)(4900.000000,1.015217) +- (-0.000666,0.000666)(4950.000000,1.014961) +- (-0.000623,0.000623)(5000.000000,1.014804) +- (-0.000575,0.000575)
};
\addlegendentry{\textsc{Combo-R}}


\addplot [
color=col3
]
coordinates{
(150.000000,2.598985) +- (-0.282843,0.282843)(175.000000,2.023906) +- (-0.146382,0.146382)(200.000000,1.701319) +- (-0.074810,0.074810)(225.000000,1.523423) +- (-0.064066,0.064066)(250.000000,1.393206) +- (-0.043368,0.043368)(275.000000,1.323491) +- (-0.030200,0.030200)(300.000000,1.268026) +- (-0.023692,0.023692)(325.000000,1.239034) +- (-0.017776,0.017776)(350.000000,1.204637) +- (-0.013984,0.013984)(375.000000,1.184023) +- (-0.011693,0.011693)(400.000000,1.169874) +- (-0.011702,0.011702)(425.000000,1.157851) +- (-0.011498,0.011498)(450.000000,1.139772) +- (-0.010751,0.010751)(475.000000,1.132183) +- (-0.009576,0.009576)(500.000000,1.121402) +- (-0.008118,0.008118)(525.000000,1.118120) +- (-0.006815,0.006815)(550.000000,1.114281) +- (-0.006261,0.006261)(575.000000,1.111262) +- (-0.006223,0.006223)(600.000000,1.108045) +- (-0.005792,0.005792)(625.000000,1.105352) +- (-0.005880,0.005880)(650.000000,1.100780) +- (-0.005596,0.005596)(675.000000,1.100354) +- (-0.007315,0.007315)(700.000000,1.096997) +- (-0.007348,0.007348)(725.000000,1.096843) +- (-0.008202,0.008202)(750.000000,1.093263) +- (-0.008244,0.008244)(775.000000,1.089028) +- (-0.007317,0.007317)(800.000000,1.085282) +- (-0.006898,0.006898)(825.000000,1.082272) +- (-0.006098,0.006098)(850.000000,1.079688) +- (-0.005911,0.005911)(875.000000,1.076701) +- (-0.005654,0.005654)(900.000000,1.074960) +- (-0.005455,0.005455)(925.000000,1.073440) +- (-0.005015,0.005015)(950.000000,1.072369) +- (-0.005023,0.005023)(975.000000,1.069682) +- (-0.004604,0.004604)(1000.000000,1.067049) +- (-0.004394,0.004394)(1000.000000,1.067049) +- (-0.004394,0.004394)(1050.000000,1.064626) +- (-0.003902,0.003902)(1100.000000,1.061626) +- (-0.003691,0.003691)(1150.000000,1.058781) +- (-0.003089,0.003089)(1200.000000,1.055204) +- (-0.003169,0.003169)(1250.000000,1.053070) +- (-0.003068,0.003068)(1300.000000,1.052601) +- (-0.003043,0.003043)(1350.000000,1.051291) +- (-0.003274,0.003274)(1400.000000,1.049381) +- (-0.002869,0.002869)(1450.000000,1.046464) +- (-0.002446,0.002446)(1500.000000,1.044508) +- (-0.002135,0.002135)(1550.000000,1.042723) +- (-0.001906,0.001906)(1600.000000,1.041596) +- (-0.001609,0.001609)(1650.000000,1.040073) +- (-0.001626,0.001626)(1700.000000,1.038906) +- (-0.001707,0.001707)(1750.000000,1.037068) +- (-0.001711,0.001711)(1800.000000,1.035746) +- (-0.001740,0.001740)(1850.000000,1.035403) +- (-0.001843,0.001843)(1900.000000,1.034326) +- (-0.001854,0.001854)(1950.000000,1.033786) +- (-0.001946,0.001946)(2000.000000,1.032581) +- (-0.001914,0.001914)(2050.000000,1.031418) +- (-0.002025,0.002025)(2100.000000,1.030285) +- (-0.001864,0.001864)(2150.000000,1.029611) +- (-0.001913,0.001913)(2200.000000,1.029166) +- (-0.001784,0.001784)(2250.000000,1.028709) +- (-0.001842,0.001842)(2300.000000,1.027724) +- (-0.001595,0.001595)(2350.000000,1.027930) +- (-0.001725,0.001725)(2400.000000,1.027410) +- (-0.001598,0.001598)(2450.000000,1.026629) +- (-0.001686,0.001686)(2500.000000,1.025749) +- (-0.001567,0.001567)(2550.000000,1.025104) +- (-0.001682,0.001682)(2600.000000,1.024412) +- (-0.001523,0.001523)(2650.000000,1.023945) +- (-0.001585,0.001585)(2700.000000,1.023386) +- (-0.001514,0.001514)(2750.000000,1.023145) +- (-0.001471,0.001471)(2800.000000,1.022896) +- (-0.001247,0.001247)(2850.000000,1.022330) +- (-0.001029,0.001029)(2900.000000,1.022282) +- (-0.000915,0.000915)(2950.000000,1.021905) +- (-0.000910,0.000910)(3000.000000,1.021576) +- (-0.000837,0.000837)(3050.000000,1.020954) +- (-0.000812,0.000812)(3100.000000,1.020817) +- (-0.000730,0.000730)(3150.000000,1.020372) +- (-0.000697,0.000697)(3200.000000,1.020010) +- (-0.000700,0.000700)(3250.000000,1.019539) +- (-0.000698,0.000698)(3300.000000,1.018817) +- (-0.000672,0.000672)(3350.000000,1.018505) +- (-0.000631,0.000631)(3400.000000,1.018239) +- (-0.000707,0.000707)(3450.000000,1.017911) +- (-0.000711,0.000711)(3500.000000,1.017973) +- (-0.000707,0.000707)(3550.000000,1.018098) +- (-0.000686,0.000686)(3600.000000,1.018082) +- (-0.000651,0.000651)(3650.000000,1.017630) +- (-0.000644,0.000644)(3700.000000,1.017207) +- (-0.000540,0.000540)(3750.000000,1.016960) +- (-0.000642,0.000642)(3800.000000,1.016609) +- (-0.000695,0.000695)(3850.000000,1.016238) +- (-0.000718,0.000718)(3900.000000,1.016241) +- (-0.000718,0.000718)(3950.000000,1.015864) +- (-0.000714,0.000714)(4000.000000,1.015742) +- (-0.000717,0.000717)(4050.000000,1.015565) +- (-0.000714,0.000714)(4100.000000,1.015409) +- (-0.000700,0.000700)(4150.000000,1.015350) +- (-0.000724,0.000724)(4200.000000,1.015021) +- (-0.000651,0.000651)(4250.000000,1.014761) +- (-0.000689,0.000689)(4300.000000,1.014615) +- (-0.000644,0.000644)(4350.000000,1.014641) +- (-0.000653,0.000653)(4400.000000,1.014255) +- (-0.000641,0.000641)(4450.000000,1.013964) +- (-0.000609,0.000609)(4500.000000,1.013802) +- (-0.000619,0.000619)(4550.000000,1.013725) +- (-0.000605,0.000605)(4600.000000,1.013625) +- (-0.000664,0.000664)(4650.000000,1.013456) +- (-0.000694,0.000694)(4700.000000,1.013288) +- (-0.000718,0.000718)(4750.000000,1.013158) +- (-0.000668,0.000668)(4800.000000,1.012956) +- (-0.000750,0.000750)(4850.000000,1.012715) +- (-0.000736,0.000736)(4900.000000,1.012572) +- (-0.000697,0.000697)(4950.000000,1.012457) +- (-0.000702,0.000702)(5000.000000,1.012276) +- (-0.000676,0.000676)
};
\addlegendentry{\textsc{Combo-I}}

\end{axis}
\end{tikzpicture}

    \vspace{\scspacey}
    \caption{\hspace{\scspacex}\textsc{Sensor}}
    \label{fig:berkeley1}
  \end{subfigure}
  \begin{subfigure}[b]{\subflen}
    \begin{tikzpicture}


\begin{axis}[%
tick label style={/pgf/number format/fixed,font=\sffamily\small},
label style={font=\sffamily\small},
legend style={font=\sffamily\small},
view={0}{90},
width=\figurewidth,
height=\figureheight,
xmin=0, xmax=5000,
xtick={0, 1000, 2000, 3000, 4000, 5000},
xticklabels={0, 1k, 2k, 3k, 4k, 5k},
scaled x ticks=false,
xlabel={Samples},
xlabel shift=0em,
ymin=1, ymax=1.52,
ytick={1, 1.5},
yticklabels={1, 1.5},
ylabel={PSRF},
ylabel shift=-1em,
major tick length=2pt,
axis lines*=left,
legend cell align=left,
clip marker paths=true,
legend style={anchor=north east,at={(1,1)},draw=none,row sep=0em},
every axis plot/.append style={
  line width=1.5pt,
  opacity=0.8,
}
]

\addplot [
color=gcol1,
densely dashed
]
coordinates{
(150.000000,4.516237) +- (-0.282843,0.282843)(175.000000,3.830157) +- (-0.282843,0.282843)(200.000000,3.273590) +- (-0.229806,0.229806)(225.000000,2.877201) +- (-0.176235,0.176235)(250.000000,2.695993) +- (-0.235574,0.235574)(275.000000,2.523592) +- (-0.261760,0.261760)(300.000000,2.249459) +- (-0.114257,0.114257)(325.000000,2.169126) +- (-0.110096,0.110096)(350.000000,2.111823) +- (-0.106434,0.106434)(375.000000,2.063507) +- (-0.120962,0.120962)(400.000000,1.973199) +- (-0.093382,0.093382)(425.000000,1.901398) +- (-0.078620,0.078620)(450.000000,1.853378) +- (-0.073803,0.073803)(475.000000,1.813945) +- (-0.072097,0.072097)(500.000000,1.779090) +- (-0.068775,0.068775)(525.000000,1.722247) +- (-0.064540,0.064540)(550.000000,1.667105) +- (-0.056937,0.056937)(575.000000,1.640735) +- (-0.052273,0.052273)(600.000000,1.610234) +- (-0.047158,0.047158)(625.000000,1.583779) +- (-0.045573,0.045573)(650.000000,1.561775) +- (-0.046127,0.046127)(675.000000,1.550530) +- (-0.050095,0.050095)(700.000000,1.531675) +- (-0.051499,0.051499)(725.000000,1.513150) +- (-0.049483,0.049483)(750.000000,1.492202) +- (-0.046282,0.046282)(775.000000,1.474442) +- (-0.044483,0.044483)(800.000000,1.452408) +- (-0.044128,0.044128)(825.000000,1.430268) +- (-0.041441,0.041441)(850.000000,1.404966) +- (-0.039547,0.039547)(875.000000,1.385699) +- (-0.037429,0.037429)(900.000000,1.367421) +- (-0.035029,0.035029)(925.000000,1.353516) +- (-0.032093,0.032093)(950.000000,1.343710) +- (-0.029860,0.029860)(975.000000,1.332883) +- (-0.028481,0.028481)(1000.000000,1.324716) +- (-0.027671,0.027671)(1000.000000,1.324716) +- (-0.027671,0.027671)(1050.000000,1.308945) +- (-0.026693,0.026693)(1100.000000,1.296945) +- (-0.026691,0.026691)(1150.000000,1.284966) +- (-0.025629,0.025629)(1200.000000,1.273506) +- (-0.023511,0.023511)(1250.000000,1.263467) +- (-0.021502,0.021502)(1300.000000,1.253765) +- (-0.020964,0.020964)(1350.000000,1.243800) +- (-0.019552,0.019552)(1400.000000,1.233648) +- (-0.019438,0.019438)(1450.000000,1.227089) +- (-0.019500,0.019500)(1500.000000,1.219713) +- (-0.019388,0.019388)(1550.000000,1.208019) +- (-0.019812,0.019812)(1600.000000,1.199651) +- (-0.019014,0.019014)(1650.000000,1.193562) +- (-0.018731,0.018731)(1700.000000,1.187544) +- (-0.018183,0.018183)(1750.000000,1.179802) +- (-0.016622,0.016622)(1800.000000,1.172516) +- (-0.014923,0.014923)(1850.000000,1.168321) +- (-0.014335,0.014335)(1900.000000,1.163460) +- (-0.014075,0.014075)(1950.000000,1.158933) +- (-0.014031,0.014031)(2000.000000,1.154228) +- (-0.013833,0.013833)(2050.000000,1.151817) +- (-0.014465,0.014465)(2100.000000,1.149964) +- (-0.014623,0.014623)(2150.000000,1.145804) +- (-0.014158,0.014158)(2200.000000,1.143005) +- (-0.013952,0.013952)(2250.000000,1.139541) +- (-0.013979,0.013979)(2300.000000,1.137012) +- (-0.013695,0.013695)(2350.000000,1.134188) +- (-0.013797,0.013797)(2400.000000,1.130174) +- (-0.013436,0.013436)(2450.000000,1.125665) +- (-0.013443,0.013443)(2500.000000,1.121501) +- (-0.013190,0.013190)(2550.000000,1.117630) +- (-0.012638,0.012638)(2600.000000,1.114589) +- (-0.011871,0.011871)(2650.000000,1.111642) +- (-0.011341,0.011341)(2700.000000,1.108481) +- (-0.010893,0.010893)(2750.000000,1.106012) +- (-0.009948,0.009948)(2800.000000,1.103306) +- (-0.009098,0.009098)(2850.000000,1.100321) +- (-0.008507,0.008507)(2900.000000,1.099021) +- (-0.008061,0.008061)(2950.000000,1.098218) +- (-0.008023,0.008023)(3000.000000,1.097804) +- (-0.008015,0.008015)(3050.000000,1.096534) +- (-0.007935,0.007935)(3100.000000,1.095241) +- (-0.007874,0.007874)(3150.000000,1.093985) +- (-0.007794,0.007794)(3200.000000,1.092387) +- (-0.007661,0.007661)(3250.000000,1.090922) +- (-0.007757,0.007757)(3300.000000,1.089581) +- (-0.007811,0.007811)(3350.000000,1.088310) +- (-0.007826,0.007826)(3400.000000,1.087297) +- (-0.007722,0.007722)(3450.000000,1.086096) +- (-0.007698,0.007698)(3500.000000,1.085107) +- (-0.007857,0.007857)(3550.000000,1.084093) +- (-0.008249,0.008249)(3600.000000,1.083511) +- (-0.008506,0.008506)(3650.000000,1.082479) +- (-0.008429,0.008429)(3700.000000,1.081092) +- (-0.008038,0.008038)(3750.000000,1.080306) +- (-0.007893,0.007893)(3800.000000,1.078561) +- (-0.007727,0.007727)(3850.000000,1.076938) +- (-0.007521,0.007521)(3900.000000,1.075676) +- (-0.007565,0.007565)(3950.000000,1.074756) +- (-0.007669,0.007669)(4000.000000,1.073462) +- (-0.007553,0.007553)(4050.000000,1.072442) +- (-0.007401,0.007401)(4100.000000,1.072135) +- (-0.007140,0.007140)(4150.000000,1.072124) +- (-0.007132,0.007132)(4200.000000,1.071491) +- (-0.007162,0.007162)(4250.000000,1.070260) +- (-0.007096,0.007096)(4300.000000,1.069295) +- (-0.006853,0.006853)(4350.000000,1.068797) +- (-0.006657,0.006657)(4400.000000,1.068430) +- (-0.006572,0.006572)(4450.000000,1.067382) +- (-0.006425,0.006425)(4500.000000,1.065649) +- (-0.006117,0.006117)(4550.000000,1.064589) +- (-0.005855,0.005855)(4600.000000,1.063943) +- (-0.005688,0.005688)(4650.000000,1.063607) +- (-0.005502,0.005502)(4700.000000,1.062877) +- (-0.005220,0.005220)(4750.000000,1.062406) +- (-0.004978,0.004978)(4800.000000,1.061307) +- (-0.004718,0.004718)(4850.000000,1.060198) +- (-0.004461,0.004461)(4900.000000,1.059266) +- (-0.004180,0.004180)(4950.000000,1.058128) +- (-0.004156,0.004156)(5000.000000,1.057079) +- (-0.004123,0.004123)
};
\addlegendentry{\textsc{Gibbs}}


\addplot [
color=gcol2
]
coordinates{
(150.000000,3.126284) +- (-0.242005,0.242005)(175.000000,2.967250) +- (-0.167716,0.167716)(200.000000,2.623668) +- (-0.170227,0.170227)(225.000000,2.361560) +- (-0.137858,0.137858)(250.000000,2.129161) +- (-0.105712,0.105712)(275.000000,1.989369) +- (-0.094978,0.094978)(300.000000,1.888797) +- (-0.085844,0.085844)(325.000000,1.822956) +- (-0.078461,0.078461)(350.000000,1.774579) +- (-0.075635,0.075635)(375.000000,1.731360) +- (-0.068154,0.068154)(400.000000,1.679225) +- (-0.058198,0.058198)(425.000000,1.649613) +- (-0.066685,0.066685)(450.000000,1.605256) +- (-0.054783,0.054783)(475.000000,1.569987) +- (-0.044377,0.044377)(500.000000,1.546528) +- (-0.041463,0.041463)(525.000000,1.524030) +- (-0.042004,0.042004)(550.000000,1.493081) +- (-0.044614,0.044614)(575.000000,1.466954) +- (-0.042272,0.042272)(600.000000,1.429857) +- (-0.037523,0.037523)(625.000000,1.400105) +- (-0.035049,0.035049)(650.000000,1.376252) +- (-0.032698,0.032698)(675.000000,1.364049) +- (-0.033362,0.033362)(700.000000,1.353922) +- (-0.032834,0.032834)(725.000000,1.340536) +- (-0.029944,0.029944)(750.000000,1.325644) +- (-0.029390,0.029390)(775.000000,1.311411) +- (-0.028004,0.028004)(800.000000,1.295861) +- (-0.024657,0.024657)(825.000000,1.279375) +- (-0.019995,0.019995)(850.000000,1.266085) +- (-0.017414,0.017414)(875.000000,1.259734) +- (-0.015089,0.015089)(900.000000,1.250103) +- (-0.014532,0.014532)(925.000000,1.242180) +- (-0.014514,0.014514)(950.000000,1.234217) +- (-0.014231,0.014231)(975.000000,1.227696) +- (-0.014536,0.014536)(1000.000000,1.222153) +- (-0.014443,0.014443)(1000.000000,1.222153) +- (-0.014443,0.014443)(1050.000000,1.209634) +- (-0.013958,0.013958)(1100.000000,1.201998) +- (-0.013961,0.013961)(1150.000000,1.194390) +- (-0.013502,0.013502)(1200.000000,1.185438) +- (-0.011262,0.011262)(1250.000000,1.175021) +- (-0.009839,0.009839)(1300.000000,1.166439) +- (-0.009338,0.009338)(1350.000000,1.162580) +- (-0.010043,0.010043)(1400.000000,1.157887) +- (-0.009375,0.009375)(1450.000000,1.153160) +- (-0.009534,0.009534)(1500.000000,1.147789) +- (-0.008396,0.008396)(1550.000000,1.142412) +- (-0.008222,0.008222)(1600.000000,1.136853) +- (-0.007681,0.007681)(1650.000000,1.132317) +- (-0.007172,0.007172)(1700.000000,1.128387) +- (-0.007152,0.007152)(1750.000000,1.127322) +- (-0.007361,0.007361)(1800.000000,1.126262) +- (-0.008011,0.008011)(1850.000000,1.123154) +- (-0.008377,0.008377)(1900.000000,1.120948) +- (-0.008648,0.008648)(1950.000000,1.119206) +- (-0.008226,0.008226)(2000.000000,1.115463) +- (-0.008258,0.008258)(2050.000000,1.111110) +- (-0.008232,0.008232)(2100.000000,1.107482) +- (-0.007618,0.007618)(2150.000000,1.104193) +- (-0.007594,0.007594)(2200.000000,1.100875) +- (-0.007420,0.007420)(2250.000000,1.098932) +- (-0.007387,0.007387)(2300.000000,1.095846) +- (-0.006805,0.006805)(2350.000000,1.092678) +- (-0.006562,0.006562)(2400.000000,1.089171) +- (-0.005834,0.005834)(2450.000000,1.086314) +- (-0.005523,0.005523)(2500.000000,1.084169) +- (-0.005493,0.005493)(2550.000000,1.082744) +- (-0.005255,0.005255)(2600.000000,1.080025) +- (-0.005029,0.005029)(2650.000000,1.078441) +- (-0.004886,0.004886)(2700.000000,1.077176) +- (-0.004668,0.004668)(2750.000000,1.075127) +- (-0.004626,0.004626)(2800.000000,1.073319) +- (-0.004958,0.004958)(2850.000000,1.072079) +- (-0.004992,0.004992)(2900.000000,1.070496) +- (-0.005121,0.005121)(2950.000000,1.070135) +- (-0.005211,0.005211)(3000.000000,1.069205) +- (-0.005249,0.005249)(3050.000000,1.067519) +- (-0.005002,0.005002)(3100.000000,1.065964) +- (-0.004782,0.004782)(3150.000000,1.064361) +- (-0.004910,0.004910)(3200.000000,1.064042) +- (-0.004793,0.004793)(3250.000000,1.063152) +- (-0.004550,0.004550)(3300.000000,1.062110) +- (-0.004478,0.004478)(3350.000000,1.061689) +- (-0.004374,0.004374)(3400.000000,1.061245) +- (-0.004361,0.004361)(3450.000000,1.060243) +- (-0.004316,0.004316)(3500.000000,1.059101) +- (-0.004084,0.004084)(3550.000000,1.058265) +- (-0.003884,0.003884)(3600.000000,1.057640) +- (-0.003713,0.003713)(3650.000000,1.056679) +- (-0.003582,0.003582)(3700.000000,1.055696) +- (-0.003427,0.003427)(3750.000000,1.054627) +- (-0.003177,0.003177)(3800.000000,1.052992) +- (-0.002958,0.002958)(3850.000000,1.051297) +- (-0.002805,0.002805)(3900.000000,1.050598) +- (-0.002936,0.002936)(3950.000000,1.050377) +- (-0.002957,0.002957)(4000.000000,1.050465) +- (-0.002902,0.002902)(4050.000000,1.050107) +- (-0.002878,0.002878)(4100.000000,1.049838) +- (-0.002906,0.002906)(4150.000000,1.049470) +- (-0.002780,0.002780)(4200.000000,1.048656) +- (-0.002811,0.002811)(4250.000000,1.047425) +- (-0.002809,0.002809)(4300.000000,1.046920) +- (-0.002838,0.002838)(4350.000000,1.046320) +- (-0.002906,0.002906)(4400.000000,1.045553) +- (-0.002827,0.002827)(4450.000000,1.044695) +- (-0.002820,0.002820)(4500.000000,1.043789) +- (-0.002856,0.002856)(4550.000000,1.043147) +- (-0.002794,0.002794)(4600.000000,1.042884) +- (-0.002687,0.002687)(4650.000000,1.042286) +- (-0.002513,0.002513)(4700.000000,1.041712) +- (-0.002390,0.002390)(4750.000000,1.041256) +- (-0.002369,0.002369)(4800.000000,1.040269) +- (-0.002268,0.002268)(4850.000000,1.039686) +- (-0.002173,0.002173)(4900.000000,1.039570) +- (-0.002113,0.002113)(4950.000000,1.039143) +- (-0.002055,0.002055)(5000.000000,1.038411) +- (-0.001962,0.001962)
};
\addlegendentry{\textsc{Combo-R}}


\addplot [
color=gcol3
]
coordinates{
(100.000000,3.504238) +- (-0.282843,0.282843)(125.000000,2.807509) +- (-0.282843,0.282843)(150.000000,2.267132) +- (-0.132254,0.132254)(175.000000,2.053463) +- (-0.100202,0.100202)(200.000000,1.885144) +- (-0.097802,0.097802)(225.000000,1.798585) +- (-0.084779,0.084779)(250.000000,1.671529) +- (-0.053453,0.053453)(275.000000,1.580282) +- (-0.052860,0.052860)(300.000000,1.508231) +- (-0.043306,0.043306)(325.000000,1.471908) +- (-0.043179,0.043179)(350.000000,1.416112) +- (-0.030592,0.030592)(375.000000,1.386443) +- (-0.027724,0.027724)(400.000000,1.356410) +- (-0.026889,0.026889)(425.000000,1.338293) +- (-0.028158,0.028158)(450.000000,1.326141) +- (-0.028697,0.028697)(475.000000,1.312075) +- (-0.029212,0.029212)(500.000000,1.291193) +- (-0.023838,0.023838)(525.000000,1.272725) +- (-0.019471,0.019471)(550.000000,1.262250) +- (-0.020583,0.020583)(575.000000,1.249389) +- (-0.020325,0.020325)(600.000000,1.233877) +- (-0.016519,0.016519)(625.000000,1.226579) +- (-0.016325,0.016325)(650.000000,1.220713) +- (-0.016387,0.016387)(675.000000,1.213551) +- (-0.016487,0.016487)(700.000000,1.203788) +- (-0.015545,0.015545)(725.000000,1.196004) +- (-0.013996,0.013996)(750.000000,1.190473) +- (-0.013166,0.013166)(775.000000,1.185443) +- (-0.012677,0.012677)(800.000000,1.179233) +- (-0.012541,0.012541)(825.000000,1.171537) +- (-0.011839,0.011839)(850.000000,1.165376) +- (-0.010910,0.010910)(875.000000,1.157642) +- (-0.009882,0.009882)(900.000000,1.151277) +- (-0.008983,0.008983)(925.000000,1.147299) +- (-0.008794,0.008794)(950.000000,1.144925) +- (-0.008791,0.008791)(975.000000,1.141484) +- (-0.008848,0.008848)(1000.000000,1.138509) +- (-0.008576,0.008576)(1000.000000,1.138509) +- (-0.008576,0.008576)(1050.000000,1.129799) +- (-0.008290,0.008290)(1100.000000,1.122778) +- (-0.007940,0.007940)(1150.000000,1.115578) +- (-0.007197,0.007197)(1200.000000,1.109873) +- (-0.007159,0.007159)(1250.000000,1.105875) +- (-0.007242,0.007242)(1300.000000,1.101614) +- (-0.007005,0.007005)(1350.000000,1.097087) +- (-0.006972,0.006972)(1400.000000,1.095587) +- (-0.007530,0.007530)(1450.000000,1.092937) +- (-0.006752,0.006752)(1500.000000,1.087729) +- (-0.005510,0.005510)(1550.000000,1.084350) +- (-0.005065,0.005065)(1600.000000,1.081542) +- (-0.004927,0.004927)(1650.000000,1.077401) +- (-0.004830,0.004830)(1700.000000,1.074216) +- (-0.004113,0.004113)(1750.000000,1.070874) +- (-0.003476,0.003476)(1800.000000,1.069468) +- (-0.003160,0.003160)(1850.000000,1.066446) +- (-0.002897,0.002897)(1900.000000,1.064877) +- (-0.002838,0.002838)(1950.000000,1.062515) +- (-0.002905,0.002905)(2000.000000,1.059906) +- (-0.003026,0.003026)(2050.000000,1.058294) +- (-0.002735,0.002735)(2100.000000,1.056614) +- (-0.002723,0.002723)(2150.000000,1.056915) +- (-0.002597,0.002597)(2200.000000,1.056143) +- (-0.002341,0.002341)(2250.000000,1.054420) +- (-0.002170,0.002170)(2300.000000,1.053020) +- (-0.002251,0.002251)(2350.000000,1.051299) +- (-0.002619,0.002619)(2400.000000,1.050231) +- (-0.002590,0.002590)(2450.000000,1.048521) +- (-0.002377,0.002377)(2500.000000,1.047197) +- (-0.001962,0.001962)(2550.000000,1.045708) +- (-0.001751,0.001751)(2600.000000,1.044521) +- (-0.001872,0.001872)(2650.000000,1.043858) +- (-0.002088,0.002088)(2700.000000,1.042963) +- (-0.002304,0.002304)(2750.000000,1.042081) +- (-0.002218,0.002218)(2800.000000,1.041521) +- (-0.002119,0.002119)(2850.000000,1.041528) +- (-0.001921,0.001921)(2900.000000,1.041374) +- (-0.001909,0.001909)(2950.000000,1.040812) +- (-0.001899,0.001899)(3000.000000,1.039801) +- (-0.001871,0.001871)(3050.000000,1.039643) +- (-0.001855,0.001855)(3100.000000,1.039454) +- (-0.002077,0.002077)(3150.000000,1.038837) +- (-0.002140,0.002140)(3200.000000,1.038101) +- (-0.002097,0.002097)(3250.000000,1.037054) +- (-0.001943,0.001943)(3300.000000,1.036940) +- (-0.001977,0.001977)(3350.000000,1.036014) +- (-0.001714,0.001714)(3400.000000,1.035354) +- (-0.001655,0.001655)(3450.000000,1.034671) +- (-0.001730,0.001730)(3500.000000,1.034072) +- (-0.001843,0.001843)(3550.000000,1.033544) +- (-0.001783,0.001783)(3600.000000,1.033082) +- (-0.001805,0.001805)(3650.000000,1.032906) +- (-0.001851,0.001851)(3700.000000,1.032635) +- (-0.001895,0.001895)(3750.000000,1.032710) +- (-0.001943,0.001943)(3800.000000,1.032106) +- (-0.001749,0.001749)(3850.000000,1.031760) +- (-0.001786,0.001786)(3900.000000,1.031239) +- (-0.001746,0.001746)(3950.000000,1.030631) +- (-0.001834,0.001834)(4000.000000,1.030050) +- (-0.001910,0.001910)(4050.000000,1.029740) +- (-0.001989,0.001989)(4100.000000,1.029658) +- (-0.001968,0.001968)(4150.000000,1.029153) +- (-0.001889,0.001889)(4200.000000,1.028772) +- (-0.001761,0.001761)(4250.000000,1.028243) +- (-0.001698,0.001698)(4300.000000,1.027599) +- (-0.001659,0.001659)(4350.000000,1.027296) +- (-0.001738,0.001738)(4400.000000,1.026791) +- (-0.001802,0.001802)(4450.000000,1.026763) +- (-0.001772,0.001772)(4500.000000,1.026613) +- (-0.001707,0.001707)(4550.000000,1.026540) +- (-0.001617,0.001617)(4600.000000,1.026155) +- (-0.001686,0.001686)(4650.000000,1.025719) +- (-0.001648,0.001648)(4700.000000,1.025094) +- (-0.001527,0.001527)(4750.000000,1.024802) +- (-0.001489,0.001489)(4800.000000,1.024611) +- (-0.001533,0.001533)(4850.000000,1.024479) +- (-0.001440,0.001440)(4900.000000,1.024159) +- (-0.001359,0.001359)(4950.000000,1.023814) +- (-0.001304,0.001304)(5000.000000,1.023418) +- (-0.001323,0.001323)
};
\addlegendentry{\textsc{Combo-I}}

\end{axis}
\end{tikzpicture}

    \vspace{\scspacey}
    \caption{\hspace{\scspacex}\textsc{Game}}
    \label{fig:hots1}
  \end{subfigure}\\[0.8em]
  \begin{subfigure}[b]{\subflen}
    \centering
    \begin{tikzpicture}


\begin{axis}[%
tick label style={/pgf/number format/fixed,font=\sffamily\small},
label style={font=\sffamily\small},
legend style={font=\sffamily\small},
view={0}{90},
width=\figurewidth,
height=\figureheight,
xmin=0, xmax=200,
xtick={0, 100, 200},
xticklabels={0, 100, 200},
scaled x ticks=false,
xlabel={Time (ms)},
xlabel shift=0em,
ymin=1, ymax=1.52,
ytick={1, 1.5},
yticklabels={1, 1.5},
ylabel={PSRF},
ylabel shift=-1em,
major tick length=2pt,
axis lines*=left,
legend cell align=left,
clip marker paths=true,
legend style={anchor=north east,at={(1,1)},draw=none,row sep=0em},
every axis plot/.append style={
  line width=1.5pt,
  opacity=0.8,
}
]

\addplot [
color=col1dark,
densely dashed,
]
coordinates{
(8.110889,3.597939) +- (-0.282843,0.282843)(9.124750,3.040631) +- (-0.197440,0.197440)(10.138612,2.691309) +- (-0.160233,0.160233)(11.152473,2.554452) +- (-0.165238,0.165238)(12.166334,2.413226) +- (-0.160241,0.160241)(13.180195,2.330296) +- (-0.148466,0.148466)(14.194056,2.215460) +- (-0.130796,0.130796)(15.207917,2.169512) +- (-0.130263,0.130263)(16.221778,2.082272) +- (-0.130343,0.130343)(17.235640,2.028475) +- (-0.144012,0.144012)(18.249501,2.014070) +- (-0.189635,0.189635)(19.263362,2.059465) +- (-0.282843,0.282843)(20.277223,1.948577) +- (-0.207447,0.207447)(21.291084,1.888214) +- (-0.187895,0.187895)(22.304945,1.800753) +- (-0.146235,0.146235)(23.318807,1.749965) +- (-0.125414,0.125414)(24.332668,1.717381) +- (-0.149604,0.149604)(25.346529,1.704961) +- (-0.179373,0.179373)(26.360390,1.685182) +- (-0.188536,0.188536)(27.374251,1.651440) +- (-0.194696,0.194696)(28.388112,1.565033) +- (-0.110935,0.110935)(29.401973,1.494971) +- (-0.060617,0.060617)(30.415835,1.448536) +- (-0.037580,0.037580)(31.429696,1.423904) +- (-0.030389,0.030389)(32.443557,1.404557) +- (-0.025724,0.025724)(33.457418,1.388615) +- (-0.024165,0.024165)(34.471279,1.375496) +- (-0.022723,0.022723)(35.485140,1.365812) +- (-0.022431,0.022431)(36.499002,1.359307) +- (-0.022667,0.022667)(37.512863,1.357107) +- (-0.023374,0.023374)(38.526724,1.353670) +- (-0.024974,0.024974)(39.540585,1.349603) +- (-0.026614,0.026614)(40.554446,1.341914) +- (-0.026115,0.026115)(40.554446,1.341914) +- (-0.026115,0.026115)(42.582168,1.327010) +- (-0.025499,0.025499)(44.609891,1.309038) +- (-0.022776,0.022776)(46.637613,1.291903) +- (-0.021510,0.021510)(48.665335,1.276404) +- (-0.019627,0.019627)(50.693058,1.265232) +- (-0.019791,0.019791)(52.720780,1.257915) +- (-0.022160,0.022160)(54.748502,1.245602) +- (-0.022382,0.022382)(56.776225,1.229363) +- (-0.019007,0.019007)(58.803947,1.212160) +- (-0.014853,0.014853)(60.831669,1.202877) +- (-0.014921,0.014921)(62.859391,1.194305) +- (-0.016019,0.016019)(64.887114,1.187061) +- (-0.016566,0.016566)(66.914836,1.181264) +- (-0.015779,0.015779)(68.942558,1.176979) +- (-0.015684,0.015684)(70.970281,1.173301) +- (-0.016778,0.016778)(72.998003,1.166561) +- (-0.015625,0.015625)(75.025725,1.160849) +- (-0.014327,0.014327)(77.053448,1.157492) +- (-0.013240,0.013240)(79.081170,1.152301) +- (-0.012648,0.012648)(81.108892,1.146251) +- (-0.010389,0.010389)(83.136615,1.142146) +- (-0.008755,0.008755)(85.164337,1.138389) +- (-0.008279,0.008279)(87.192059,1.134830) +- (-0.007781,0.007781)(89.219781,1.131334) +- (-0.007385,0.007385)(91.247504,1.129364) +- (-0.007120,0.007120)(93.275226,1.127263) +- (-0.007972,0.007972)(95.302948,1.124395) +- (-0.008520,0.008520)(97.330671,1.119219) +- (-0.008856,0.008856)(99.358393,1.117366) +- (-0.010033,0.010033)(101.386115,1.114105) +- (-0.008819,0.008819)(103.413838,1.113725) +- (-0.008468,0.008468)(105.441560,1.111648) +- (-0.008166,0.008166)(107.469282,1.109619) +- (-0.007836,0.007836)(109.497005,1.107001) +- (-0.007309,0.007309)(111.524727,1.104818) +- (-0.006968,0.006968)(113.552449,1.104075) +- (-0.006324,0.006324)(115.580171,1.102976) +- (-0.006398,0.006398)(117.607894,1.100722) +- (-0.006288,0.006288)(119.635616,1.097873) +- (-0.005578,0.005578)(121.663338,1.095111) +- (-0.005033,0.005033)(123.691061,1.093478) +- (-0.005158,0.005158)(125.718783,1.091610) +- (-0.005177,0.005177)(127.746505,1.090405) +- (-0.005001,0.005001)(129.774228,1.089527) +- (-0.004982,0.004982)(131.801950,1.088855) +- (-0.004821,0.004821)(133.829672,1.087810) +- (-0.004819,0.004819)(135.857395,1.086178) +- (-0.004890,0.004890)(137.885117,1.083872) +- (-0.004823,0.004823)(139.912839,1.082651) +- (-0.004757,0.004757)(141.940561,1.082000) +- (-0.004411,0.004411)(143.968284,1.081654) +- (-0.004404,0.004404)(145.996006,1.080664) +- (-0.004261,0.004261)(148.023728,1.079681) +- (-0.004000,0.004000)(150.051451,1.078145) +- (-0.004157,0.004157)(152.079173,1.077135) +- (-0.004285,0.004285)(154.106895,1.076356) +- (-0.004383,0.004383)(156.134618,1.075704) +- (-0.004286,0.004286)(158.162340,1.074393) +- (-0.004038,0.004038)(160.190062,1.072686) +- (-0.003808,0.003808)(162.217785,1.071898) +- (-0.003671,0.003671)(164.245507,1.070577) +- (-0.003569,0.003569)(166.273229,1.068997) +- (-0.003373,0.003373)(168.300951,1.067786) +- (-0.003359,0.003359)(170.328674,1.067305) +- (-0.003437,0.003437)(172.356396,1.066197) +- (-0.003413,0.003413)(174.384118,1.064406) +- (-0.003289,0.003289)(176.411841,1.062878) +- (-0.003119,0.003119)(178.439563,1.061796) +- (-0.003118,0.003118)(180.467285,1.061438) +- (-0.003328,0.003328)(182.495008,1.061019) +- (-0.003352,0.003352)(184.522730,1.060327) +- (-0.003275,0.003275)(186.550452,1.059904) +- (-0.003248,0.003248)(188.578174,1.059435) +- (-0.003126,0.003126)(190.605897,1.058691) +- (-0.002941,0.002941)(192.633619,1.057888) +- (-0.002763,0.002763)(194.661341,1.057200) +- (-0.002759,0.002759)(196.689064,1.056826) +- (-0.002693,0.002693)(198.716786,1.056174) +- (-0.002721,0.002721)(200.744508,1.055308) +- (-0.002654,0.002654)(202.772231,1.054398) +- (-0.002602,0.002602)
};
\addlegendentry{\textsc{Gibbs}}


\addplot [
color=col2
]
coordinates{
(4.691006,2.314380) +- (-0.145936,0.145936)(5.863757,2.004462) +- (-0.127423,0.127423)(7.036509,1.793518) +- (-0.080239,0.080239)(8.209260,1.631944) +- (-0.067338,0.067338)(9.382012,1.547655) +- (-0.050816,0.050816)(10.554763,1.467467) +- (-0.045281,0.045281)(11.727515,1.434199) +- (-0.047724,0.047724)(12.900266,1.387869) +- (-0.026698,0.026698)(14.073018,1.352802) +- (-0.024894,0.024894)(15.245769,1.325987) +- (-0.022519,0.022519)(16.418521,1.300620) +- (-0.018924,0.018924)(17.591272,1.276801) +- (-0.015945,0.015945)(18.764024,1.258876) +- (-0.015658,0.015658)(19.936775,1.247336) +- (-0.015910,0.015910)(21.109527,1.231760) +- (-0.014686,0.014686)(22.282278,1.220108) +- (-0.014804,0.014804)(23.455030,1.209891) +- (-0.016687,0.016687)(24.627781,1.197965) +- (-0.017494,0.017494)(25.800533,1.187772) +- (-0.017134,0.017134)(26.973284,1.181613) +- (-0.016117,0.016117)(28.146036,1.175375) +- (-0.015438,0.015438)(29.318787,1.167047) +- (-0.014963,0.014963)(30.491539,1.162951) +- (-0.014282,0.014282)(31.664290,1.159416) +- (-0.013639,0.013639)(32.837042,1.151083) +- (-0.012802,0.012802)(34.009793,1.143855) +- (-0.013366,0.013366)(35.182545,1.139644) +- (-0.013836,0.013836)(36.355296,1.136401) +- (-0.012880,0.012880)(37.528048,1.130305) +- (-0.011110,0.011110)(38.700799,1.127232) +- (-0.010790,0.010790)(39.873550,1.121431) +- (-0.010224,0.010224)(41.046302,1.116844) +- (-0.009764,0.009764)(42.219053,1.113843) +- (-0.009180,0.009180)(43.391805,1.109543) +- (-0.008657,0.008657)(44.564556,1.105246) +- (-0.008078,0.008078)(45.737308,1.101941) +- (-0.007431,0.007431)(46.910059,1.097746) +- (-0.006427,0.006427)(46.910059,1.097746) +- (-0.006427,0.006427)(49.255562,1.090858) +- (-0.005703,0.005703)(51.601065,1.085136) +- (-0.004712,0.004712)(53.946568,1.082785) +- (-0.004480,0.004480)(56.292071,1.078993) +- (-0.004481,0.004481)(58.637574,1.075416) +- (-0.004278,0.004278)(60.983077,1.072572) +- (-0.004408,0.004408)(63.328580,1.072164) +- (-0.004253,0.004253)(65.674083,1.071047) +- (-0.003945,0.003945)(68.019586,1.068267) +- (-0.003995,0.003995)(70.365089,1.064680) +- (-0.003782,0.003782)(72.710592,1.062878) +- (-0.003607,0.003607)(75.056095,1.060127) +- (-0.002978,0.002978)(77.401598,1.057291) +- (-0.002852,0.002852)(79.747101,1.056389) +- (-0.002972,0.002972)(82.092604,1.054495) +- (-0.002850,0.002850)(84.438107,1.052421) +- (-0.002805,0.002805)(86.783610,1.052303) +- (-0.003047,0.003047)(89.129113,1.051432) +- (-0.002882,0.002882)(91.474616,1.049071) +- (-0.002860,0.002860)(93.820119,1.048121) +- (-0.002986,0.002986)(96.165622,1.046968) +- (-0.002974,0.002974)(98.511125,1.046242) +- (-0.002929,0.002929)(100.856628,1.046152) +- (-0.002783,0.002783)(103.202131,1.045838) +- (-0.002628,0.002628)(105.547634,1.044434) +- (-0.002501,0.002501)(107.893137,1.043060) +- (-0.002611,0.002611)(110.238640,1.042289) +- (-0.002473,0.002473)(112.584143,1.041161) +- (-0.002194,0.002194)(114.929646,1.040078) +- (-0.002039,0.002039)(117.275149,1.038894) +- (-0.001963,0.001963)(119.620651,1.037761) +- (-0.002019,0.002019)(121.966154,1.036455) +- (-0.001902,0.001902)(124.311657,1.035295) +- (-0.001652,0.001652)(126.657160,1.034810) +- (-0.001550,0.001550)(129.002663,1.034331) +- (-0.001533,0.001533)(131.348166,1.033830) +- (-0.001393,0.001393)(133.693669,1.032976) +- (-0.001362,0.001362)(136.039172,1.032553) +- (-0.001368,0.001368)(138.384675,1.031991) +- (-0.001444,0.001444)(140.730178,1.031694) +- (-0.001432,0.001432)(143.075681,1.031078) +- (-0.001410,0.001410)(145.421184,1.030946) +- (-0.001536,0.001536)(147.766687,1.030691) +- (-0.001492,0.001492)(150.112190,1.030606) +- (-0.001415,0.001415)(152.457693,1.030274) +- (-0.001521,0.001521)(154.803196,1.029901) +- (-0.001567,0.001567)(157.148699,1.029501) +- (-0.001596,0.001596)(159.494202,1.029197) +- (-0.001490,0.001490)(161.839705,1.028674) +- (-0.001432,0.001432)(164.185208,1.028292) +- (-0.001420,0.001420)(166.530711,1.027638) +- (-0.001465,0.001465)(168.876214,1.027133) +- (-0.001453,0.001453)(171.221717,1.027022) +- (-0.001433,0.001433)(173.567220,1.026421) +- (-0.001474,0.001474)(175.912723,1.025866) +- (-0.001395,0.001395)(178.258226,1.025313) +- (-0.001397,0.001397)(180.603729,1.025126) +- (-0.001384,0.001384)(182.949232,1.024998) +- (-0.001477,0.001477)(185.294735,1.024660) +- (-0.001446,0.001446)(187.640238,1.024171) +- (-0.001362,0.001362)(189.985741,1.023864) +- (-0.001253,0.001253)(192.331244,1.023532) +- (-0.001167,0.001167)(194.676747,1.023254) +- (-0.001154,0.001154)(197.022249,1.023072) +- (-0.001145,0.001145)(199.367752,1.022695) +- (-0.001132,0.001132)(201.713255,1.022507) +- (-0.001124,0.001124)(204.058758,1.022286) +- (-0.001143,0.001143)(206.404261,1.021894) +- (-0.001132,0.001132)(208.749764,1.021687) +- (-0.001104,0.001104)(211.095267,1.021078) +- (-0.000988,0.000988)(213.440770,1.020705) +- (-0.000968,0.000968)(215.786273,1.020400) +- (-0.000900,0.000900)(218.131776,1.020334) +- (-0.000907,0.000907)(220.477279,1.020393) +- (-0.000906,0.000906)(222.822782,1.020249) +- (-0.000837,0.000837)(225.168285,1.020091) +- (-0.000844,0.000844)(227.513788,1.019827) +- (-0.000798,0.000798)(229.859291,1.019461) +- (-0.000836,0.000836)(232.204794,1.019257) +- (-0.000829,0.000829)(234.550297,1.019229) +- (-0.000875,0.000875)
};
\addlegendentry{\textsc{Combo-R}}


\addplot [
color=col3
]
coordinates{
(2.353020,3.854406) +- (-0.282843,0.282843)(3.529530,2.516837) +- (-0.198273,0.198273)(4.706040,2.080291) +- (-0.204548,0.204548)(5.882550,1.761562) +- (-0.101370,0.101370)(7.059060,1.588895) +- (-0.068315,0.068315)(8.235570,1.530598) +- (-0.077335,0.077335)(9.412080,1.472095) +- (-0.088279,0.088279)(10.588590,1.407313) +- (-0.069317,0.069317)(11.765099,1.328349) +- (-0.050041,0.050041)(12.941609,1.280189) +- (-0.046740,0.046740)(14.118119,1.241093) +- (-0.028634,0.028634)(15.294629,1.225659) +- (-0.018168,0.018168)(16.471139,1.214143) +- (-0.016083,0.016083)(17.647649,1.199464) +- (-0.015557,0.015557)(18.824159,1.192993) +- (-0.017602,0.017602)(20.000669,1.182354) +- (-0.016598,0.016598)(21.177179,1.173964) +- (-0.016687,0.016687)(22.353689,1.162564) +- (-0.013994,0.013994)(23.530199,1.153207) +- (-0.012485,0.012485)(24.706709,1.146027) +- (-0.011376,0.011376)(25.883219,1.138410) +- (-0.011042,0.011042)(27.059729,1.132240) +- (-0.010125,0.010125)(28.236239,1.124508) +- (-0.009295,0.009295)(29.412749,1.120885) +- (-0.010133,0.010133)(30.589259,1.114675) +- (-0.009175,0.009175)(31.765769,1.110488) +- (-0.008958,0.008958)(32.942279,1.106881) +- (-0.007907,0.007907)(34.118789,1.104418) +- (-0.007560,0.007560)(35.295298,1.099178) +- (-0.007433,0.007433)(36.471808,1.094869) +- (-0.007824,0.007824)(37.648318,1.092353) +- (-0.007203,0.007203)(38.824828,1.089835) +- (-0.006443,0.006443)(40.001338,1.086294) +- (-0.005841,0.005841)(41.177848,1.083016) +- (-0.004871,0.004871)(42.354358,1.080625) +- (-0.004230,0.004230)(43.530868,1.078600) +- (-0.003971,0.003971)(44.707378,1.075129) +- (-0.003631,0.003631)(45.883888,1.073524) +- (-0.003559,0.003559)(47.060398,1.071891) +- (-0.003385,0.003385)(47.060398,1.071891) +- (-0.003385,0.003385)(49.413418,1.066995) +- (-0.003261,0.003261)(51.766438,1.064854) +- (-0.003031,0.003031)(54.119458,1.062135) +- (-0.003158,0.003158)(56.472478,1.060072) +- (-0.002924,0.002924)(58.825497,1.058380) +- (-0.002652,0.002652)(61.178517,1.055289) +- (-0.003135,0.003135)(63.531537,1.053504) +- (-0.002473,0.002473)(65.884557,1.051133) +- (-0.002215,0.002215)(68.237577,1.049357) +- (-0.002129,0.002129)(70.590597,1.048042) +- (-0.002443,0.002443)(72.943617,1.046165) +- (-0.002292,0.002292)(75.296637,1.044985) +- (-0.002335,0.002335)(77.649657,1.043813) +- (-0.002302,0.002302)(80.002677,1.042142) +- (-0.002126,0.002126)(82.355696,1.040972) +- (-0.001732,0.001732)(84.708716,1.038792) +- (-0.001605,0.001605)(87.061736,1.037901) +- (-0.001482,0.001482)(89.414756,1.037031) +- (-0.001745,0.001745)(91.767776,1.036926) +- (-0.002238,0.002238)(94.120796,1.036016) +- (-0.002091,0.002091)(96.473816,1.034859) +- (-0.002239,0.002239)(98.826836,1.033716) +- (-0.002164,0.002164)(101.179856,1.032822) +- (-0.001992,0.001992)(103.532876,1.031838) +- (-0.001809,0.001809)(105.885895,1.030816) +- (-0.001774,0.001774)(108.238915,1.030378) +- (-0.001692,0.001692)(110.591935,1.030052) +- (-0.001777,0.001777)(112.944955,1.029630) +- (-0.001723,0.001723)(115.297975,1.028859) +- (-0.001648,0.001648)(117.650995,1.028342) +- (-0.001664,0.001664)(120.004015,1.027460) +- (-0.001711,0.001711)(122.357035,1.026793) +- (-0.001637,0.001637)(124.710055,1.026026) +- (-0.001552,0.001552)(127.063074,1.025250) +- (-0.001265,0.001265)(129.416094,1.025001) +- (-0.001289,0.001289)(131.769114,1.024221) +- (-0.001330,0.001330)(134.122134,1.023217) +- (-0.001235,0.001235)(136.475154,1.023087) +- (-0.001262,0.001262)(138.828174,1.022697) +- (-0.001187,0.001187)(141.181194,1.022573) +- (-0.001180,0.001180)(143.534214,1.022419) +- (-0.001269,0.001269)(145.887234,1.022147) +- (-0.001212,0.001212)(148.240254,1.021630) +- (-0.001206,0.001206)(150.593273,1.021636) +- (-0.001243,0.001243)(152.946293,1.021386) +- (-0.001207,0.001207)(155.299313,1.021237) +- (-0.001212,0.001212)(157.652333,1.020897) +- (-0.001196,0.001196)(160.005353,1.020573) +- (-0.001154,0.001154)(162.358373,1.020276) +- (-0.001089,0.001089)(164.711393,1.019911) +- (-0.001184,0.001184)(167.064413,1.019570) +- (-0.001224,0.001224)(169.417433,1.019220) +- (-0.001139,0.001139)(171.770453,1.018847) +- (-0.001054,0.001054)(174.123472,1.018390) +- (-0.000911,0.000911)(176.476492,1.018087) +- (-0.000910,0.000910)(178.829512,1.018215) +- (-0.001009,0.001009)(181.182532,1.017975) +- (-0.000981,0.000981)(183.535552,1.017773) +- (-0.000918,0.000918)(185.888572,1.017623) +- (-0.000880,0.000880)(188.241592,1.017201) +- (-0.000808,0.000808)(190.594612,1.017108) +- (-0.000833,0.000833)(192.947632,1.016947) +- (-0.000744,0.000744)(195.300652,1.016976) +- (-0.000757,0.000757)(197.653671,1.016918) +- (-0.000846,0.000846)(200.006691,1.016655) +- (-0.000857,0.000857)(202.359711,1.016267) +- (-0.000828,0.000828)(204.712731,1.016105) +- (-0.000774,0.000774)(207.065751,1.015787) +- (-0.000799,0.000799)(209.418771,1.015690) +- (-0.000837,0.000837)(211.771791,1.015584) +- (-0.000785,0.000785)(214.124811,1.015477) +- (-0.000837,0.000837)(216.477831,1.015410) +- (-0.000880,0.000880)(218.830851,1.015213) +- (-0.000865,0.000865)(221.183870,1.014935) +- (-0.000832,0.000832)(223.536890,1.014694) +- (-0.000838,0.000838)(225.889910,1.014629) +- (-0.000778,0.000778)(228.242930,1.014533) +- (-0.000751,0.000751)(230.595950,1.014393) +- (-0.000742,0.000742)(232.948970,1.014260) +- (-0.000782,0.000782)(235.301990,1.014056) +- (-0.000720,0.000720)
};
\addlegendentry{\textsc{Combo-I}}

\end{axis}
\end{tikzpicture}

    \vspace{\scspacey}
    \caption{\hspace{\scspacex}\textsc{Water}}
    \label{fig:water1-time}
  \end{subfigure}
  \begin{subfigure}[b]{\subflen}
    \begin{tikzpicture}


\begin{axis}[%
tick label style={/pgf/number format/fixed,font=\sffamily\small},
label style={font=\sffamily\small},
legend style={font=\sffamily\small},
view={0}{90},
width=\figurewidth,
height=\figureheight,
xmin=0, xmax=300,
xtick={0, 100, 200, 300},
xticklabels={0, 100, 200, 300},
scaled x ticks=false,
xlabel={Time (ms)},
xlabel shift=0em,
ymin=1, ymax=1.52,
ytick={1, 1.5},
yticklabels={1, 1.5},
ylabel={PSRF},
ylabel shift=-1em,
major tick length=2pt,
axis lines*=left,
legend cell align=left,
clip marker paths=true,
legend style={anchor=north east,at={(1,1)},draw=none,row sep=0em},
every axis plot/.append style={
  line width=1.5pt,
  opacity=0.8,
}
]

\addplot [
color=col1dark,
densely dashed
]
coordinates{
(14.409560,5.365060) +- (-0.282843,0.282843)(16.210755,4.591755) +- (-0.282843,0.282843)(18.011950,4.574076) +- (-0.282843,0.282843)(19.813145,3.388613) +- (-0.281393,0.281393)(21.614340,3.076058) +- (-0.191520,0.191520)(23.415535,2.822256) +- (-0.146702,0.146702)(25.216730,2.607563) +- (-0.141973,0.141973)(27.017925,2.469923) +- (-0.175838,0.175838)(28.819120,2.410058) +- (-0.229741,0.229741)(30.620315,2.331485) +- (-0.235140,0.235140)(32.421510,2.216657) +- (-0.142948,0.142948)(34.222705,2.086454) +- (-0.098302,0.098302)(36.023900,1.948136) +- (-0.071483,0.071483)(37.825095,1.888816) +- (-0.060779,0.060779)(39.626290,1.843957) +- (-0.057140,0.057140)(41.427485,1.794805) +- (-0.058276,0.058276)(43.228680,1.738106) +- (-0.055554,0.055554)(45.029875,1.699625) +- (-0.050800,0.050800)(46.831070,1.668680) +- (-0.049063,0.049063)(48.632265,1.626034) +- (-0.051438,0.051438)(50.433460,1.604499) +- (-0.052825,0.052825)(52.234655,1.584124) +- (-0.056126,0.056126)(54.035851,1.549530) +- (-0.060329,0.060329)(55.837046,1.514791) +- (-0.060024,0.060024)(57.638241,1.487549) +- (-0.048743,0.048743)(59.439436,1.468202) +- (-0.040792,0.040792)(61.240631,1.444602) +- (-0.033844,0.033844)(63.041826,1.428660) +- (-0.030034,0.030034)(64.843021,1.415871) +- (-0.028287,0.028287)(66.644216,1.400190) +- (-0.027452,0.027452)(68.445411,1.389371) +- (-0.028086,0.028086)(70.246606,1.382080) +- (-0.028082,0.028082)(72.047801,1.371028) +- (-0.026612,0.026612)(72.047801,1.371028) +- (-0.026612,0.026612)(75.650191,1.341921) +- (-0.019217,0.019217)(79.252581,1.321902) +- (-0.017345,0.017345)(82.854971,1.310393) +- (-0.016518,0.016518)(86.457361,1.294551) +- (-0.016810,0.016810)(90.059751,1.283851) +- (-0.017795,0.017795)(93.662141,1.279615) +- (-0.018934,0.018934)(97.264531,1.270379) +- (-0.017502,0.017502)(100.866921,1.266164) +- (-0.015604,0.015604)(104.469311,1.257723) +- (-0.014725,0.014725)(108.071701,1.249316) +- (-0.016415,0.016415)(111.674091,1.242407) +- (-0.018318,0.018318)(115.276481,1.234152) +- (-0.017874,0.017874)(118.878871,1.221854) +- (-0.016611,0.016611)(122.481261,1.211019) +- (-0.014424,0.014424)(126.083651,1.202812) +- (-0.013257,0.013257)(129.686041,1.193349) +- (-0.012043,0.012043)(133.288431,1.184444) +- (-0.011020,0.011020)(136.890821,1.180122) +- (-0.009303,0.009303)(140.493211,1.175485) +- (-0.009021,0.009021)(144.095601,1.173358) +- (-0.009061,0.009061)(147.697991,1.169905) +- (-0.009030,0.009030)(151.300381,1.167314) +- (-0.008256,0.008256)(154.902771,1.163335) +- (-0.007706,0.007706)(158.505162,1.157854) +- (-0.007539,0.007539)(162.107552,1.150624) +- (-0.006815,0.006815)(165.709942,1.143648) +- (-0.005839,0.005839)(169.312332,1.140425) +- (-0.005419,0.005419)(172.914722,1.136364) +- (-0.005506,0.005506)(176.517112,1.133870) +- (-0.005647,0.005647)(180.119502,1.130276) +- (-0.005497,0.005497)(183.721892,1.126294) +- (-0.004847,0.004847)(187.324282,1.122550) +- (-0.004747,0.004747)(190.926672,1.120367) +- (-0.005175,0.005175)(194.529062,1.119356) +- (-0.005542,0.005542)(198.131452,1.117717) +- (-0.005798,0.005798)(201.733842,1.115643) +- (-0.005779,0.005779)(205.336232,1.113188) +- (-0.005633,0.005633)(208.938622,1.112326) +- (-0.005878,0.005878)(212.541012,1.109360) +- (-0.006035,0.006035)(216.143402,1.106965) +- (-0.005931,0.005931)(219.745792,1.105424) +- (-0.005750,0.005750)(223.348182,1.104043) +- (-0.005513,0.005513)(226.950572,1.102278) +- (-0.005122,0.005122)(230.552962,1.100337) +- (-0.004520,0.004520)(234.155352,1.098972) +- (-0.004164,0.004164)(237.757742,1.097375) +- (-0.004132,0.004132)(241.360132,1.095626) +- (-0.004271,0.004271)(244.962522,1.094502) +- (-0.004092,0.004092)(248.564912,1.093405) +- (-0.003681,0.003681)(252.167302,1.092404) +- (-0.003911,0.003911)(255.769692,1.091018) +- (-0.004084,0.004084)(259.372082,1.089621) +- (-0.003939,0.003939)(262.974472,1.088190) +- (-0.004078,0.004078)(266.576863,1.086834) +- (-0.004009,0.004009)(270.179253,1.086578) +- (-0.004154,0.004154)(273.781643,1.085939) +- (-0.004251,0.004251)(277.384033,1.085046) +- (-0.004105,0.004105)(280.986423,1.083951) +- (-0.003726,0.003726)(284.588813,1.081714) +- (-0.003366,0.003366)(288.191203,1.079780) +- (-0.003043,0.003043)(291.793593,1.078333) +- (-0.003047,0.003047)(295.395983,1.077290) +- (-0.003056,0.003056)(298.998373,1.076659) +- (-0.003096,0.003096)(302.600763,1.075023) +- (-0.003116,0.003116)(306.203153,1.073408) +- (-0.002976,0.002976)(309.805543,1.072610) +- (-0.002824,0.002824)(313.407933,1.071975) +- (-0.002715,0.002715)(317.010323,1.070756) +- (-0.002601,0.002601)(320.612713,1.069757) +- (-0.002691,0.002691)(324.215103,1.069134) +- (-0.002538,0.002538)(327.817493,1.068535) +- (-0.002489,0.002489)(331.419883,1.068012) +- (-0.002508,0.002508)(335.022273,1.067101) +- (-0.002472,0.002472)(338.624663,1.066104) +- (-0.002306,0.002306)(342.227053,1.065418) +- (-0.002099,0.002099)(345.829443,1.065554) +- (-0.002185,0.002185)(349.431833,1.065546) +- (-0.002261,0.002261)(353.034223,1.065401) +- (-0.002325,0.002325)(356.636613,1.065132) +- (-0.002230,0.002230)(360.239003,1.064613) +- (-0.002087,0.002087)
};
\addlegendentry{\textsc{Gibbs}}


\addplot [
color=col2
]
coordinates{
(10.295715,4.272657) +- (-0.282843,0.282843)(12.354858,2.861017) +- (-0.189265,0.189265)(14.414001,2.284893) +- (-0.132961,0.132961)(16.473144,1.931278) +- (-0.095620,0.095620)(18.532287,1.702540) +- (-0.089883,0.089883)(20.591429,1.525468) +- (-0.074378,0.074378)(22.650572,1.417393) +- (-0.059349,0.059349)(24.709715,1.356837) +- (-0.044414,0.044414)(26.768858,1.309773) +- (-0.043224,0.043224)(28.828001,1.267663) +- (-0.034687,0.034687)(30.887144,1.237023) +- (-0.024927,0.024927)(32.946287,1.218599) +- (-0.021292,0.021292)(35.005430,1.210187) +- (-0.018530,0.018530)(37.064573,1.197193) +- (-0.016534,0.016534)(39.123716,1.188801) +- (-0.017027,0.017027)(41.182859,1.182191) +- (-0.016036,0.016036)(43.242002,1.177181) +- (-0.017304,0.017304)(45.301145,1.164126) +- (-0.016521,0.016521)(47.360288,1.150835) +- (-0.015308,0.015308)(49.419431,1.142566) +- (-0.013899,0.013899)(51.478574,1.139221) +- (-0.013510,0.013510)(53.537717,1.129109) +- (-0.011488,0.011488)(55.596860,1.122447) +- (-0.010617,0.010617)(57.656003,1.115712) +- (-0.009125,0.009125)(59.715146,1.109586) +- (-0.008040,0.008040)(61.774288,1.105055) +- (-0.007006,0.007006)(63.833431,1.102083) +- (-0.006682,0.006682)(65.892574,1.098686) +- (-0.006089,0.006089)(67.951717,1.096120) +- (-0.005320,0.005320)(70.010860,1.092733) +- (-0.005291,0.005291)(72.070003,1.089636) +- (-0.005873,0.005873)(74.129146,1.087381) +- (-0.005717,0.005717)(76.188289,1.084442) +- (-0.005548,0.005548)(78.247432,1.080266) +- (-0.004843,0.004843)(80.306575,1.077660) +- (-0.004463,0.004463)(82.365718,1.076389) +- (-0.003999,0.003999)(82.365718,1.076389) +- (-0.003999,0.003999)(86.484004,1.074628) +- (-0.004189,0.004189)(90.602290,1.070641) +- (-0.003513,0.003513)(94.720576,1.067539) +- (-0.003267,0.003267)(98.838862,1.066073) +- (-0.002745,0.002745)(102.957147,1.063376) +- (-0.003073,0.003073)(107.075433,1.060532) +- (-0.003668,0.003668)(111.193719,1.058302) +- (-0.003239,0.003239)(115.312005,1.054618) +- (-0.002982,0.002982)(119.430291,1.051791) +- (-0.002411,0.002411)(123.548577,1.049556) +- (-0.002277,0.002277)(127.666863,1.047740) +- (-0.002597,0.002597)(131.785149,1.045233) +- (-0.002502,0.002502)(135.903435,1.043926) +- (-0.002368,0.002368)(140.021721,1.044117) +- (-0.002562,0.002562)(144.140006,1.043331) +- (-0.002555,0.002555)(148.258292,1.042255) +- (-0.002460,0.002460)(152.376578,1.041731) +- (-0.002406,0.002406)(156.494864,1.041156) +- (-0.002283,0.002283)(160.613150,1.040400) +- (-0.002219,0.002219)(164.731436,1.039378) +- (-0.001978,0.001978)(168.849722,1.037770) +- (-0.001830,0.001830)(172.968008,1.037051) +- (-0.001736,0.001736)(177.086294,1.036132) +- (-0.001768,0.001768)(181.204580,1.034605) +- (-0.001640,0.001640)(185.322865,1.034023) +- (-0.001590,0.001590)(189.441151,1.032596) +- (-0.001523,0.001523)(193.559437,1.031656) +- (-0.001373,0.001373)(197.677723,1.031576) +- (-0.001311,0.001311)(201.796009,1.031104) +- (-0.001292,0.001292)(205.914295,1.030614) +- (-0.001104,0.001104)(210.032581,1.029717) +- (-0.001207,0.001207)(214.150867,1.028907) +- (-0.001217,0.001217)(218.269153,1.028260) +- (-0.001234,0.001234)(222.387439,1.027780) +- (-0.001273,0.001273)(226.505724,1.027333) +- (-0.001296,0.001296)(230.624010,1.026687) +- (-0.001271,0.001271)(234.742296,1.026108) +- (-0.001166,0.001166)(238.860582,1.025506) +- (-0.000982,0.000982)(242.978868,1.025128) +- (-0.001109,0.001109)(247.097154,1.024443) +- (-0.001066,0.001066)(251.215440,1.023717) +- (-0.000975,0.000975)(255.333726,1.023429) +- (-0.001033,0.001033)(259.452012,1.023237) +- (-0.001027,0.001027)(263.570298,1.022865) +- (-0.001080,0.001080)(267.688583,1.022283) +- (-0.000973,0.000973)(271.806869,1.022223) +- (-0.000991,0.000991)(275.925155,1.022119) +- (-0.000957,0.000957)(280.043441,1.021777) +- (-0.000895,0.000895)(284.161727,1.021490) +- (-0.000785,0.000785)(288.280013,1.021050) +- (-0.000815,0.000815)(292.398299,1.020415) +- (-0.000891,0.000891)(296.516585,1.020045) +- (-0.000861,0.000861)(300.634871,1.019814) +- (-0.000793,0.000793)(304.753157,1.019478) +- (-0.000809,0.000809)(308.871442,1.019212) +- (-0.000827,0.000827)(312.989728,1.019139) +- (-0.000814,0.000814)(317.108014,1.018684) +- (-0.000893,0.000893)(321.226300,1.018688) +- (-0.000899,0.000899)(325.344586,1.018489) +- (-0.000940,0.000940)(329.462872,1.018422) +- (-0.000918,0.000918)(333.581158,1.018251) +- (-0.000921,0.000921)(337.699444,1.018088) +- (-0.000869,0.000869)(341.817730,1.017766) +- (-0.000908,0.000908)(345.936016,1.017707) +- (-0.000927,0.000927)(350.054301,1.017635) +- (-0.000922,0.000922)(354.172587,1.017451) +- (-0.000944,0.000944)(358.290873,1.016993) +- (-0.000910,0.000910)(362.409159,1.016784) +- (-0.000848,0.000848)(366.527445,1.016977) +- (-0.000884,0.000884)(370.645731,1.016812) +- (-0.000866,0.000866)(374.764017,1.016552) +- (-0.000828,0.000828)(378.882303,1.016506) +- (-0.000839,0.000839)(383.000589,1.016415) +- (-0.000812,0.000812)(387.118875,1.016248) +- (-0.000774,0.000774)(391.237160,1.015931) +- (-0.000721,0.000721)(395.355446,1.015712) +- (-0.000696,0.000696)(399.473732,1.015534) +- (-0.000704,0.000704)(403.592018,1.015217) +- (-0.000666,0.000666)(407.710304,1.014961) +- (-0.000623,0.000623)(411.828590,1.014804) +- (-0.000575,0.000575)
};
\addlegendentry{\textsc{Combo-R}}


\addplot [
color=col3
]
coordinates{
(10.407882,3.650328) +- (-0.282843,0.282843)(12.489458,2.598985) +- (-0.282843,0.282843)(14.571034,2.023906) +- (-0.146382,0.146382)(16.652611,1.701319) +- (-0.074810,0.074810)(18.734187,1.523423) +- (-0.064066,0.064066)(20.815763,1.393206) +- (-0.043368,0.043368)(22.897340,1.323491) +- (-0.030200,0.030200)(24.978916,1.268026) +- (-0.023692,0.023692)(27.060492,1.239034) +- (-0.017776,0.017776)(29.142069,1.204637) +- (-0.013984,0.013984)(31.223645,1.184023) +- (-0.011693,0.011693)(33.305221,1.169874) +- (-0.011702,0.011702)(35.386798,1.157851) +- (-0.011498,0.011498)(37.468374,1.139772) +- (-0.010751,0.010751)(39.549950,1.132183) +- (-0.009576,0.009576)(41.631527,1.121402) +- (-0.008118,0.008118)(43.713103,1.118120) +- (-0.006815,0.006815)(45.794679,1.114281) +- (-0.006261,0.006261)(47.876256,1.111262) +- (-0.006223,0.006223)(49.957832,1.108045) +- (-0.005792,0.005792)(52.039408,1.105352) +- (-0.005880,0.005880)(54.120985,1.100780) +- (-0.005596,0.005596)(56.202561,1.100354) +- (-0.007315,0.007315)(58.284137,1.096997) +- (-0.007348,0.007348)(60.365714,1.096843) +- (-0.008202,0.008202)(62.447290,1.093263) +- (-0.008244,0.008244)(64.528866,1.089028) +- (-0.007317,0.007317)(66.610443,1.085282) +- (-0.006898,0.006898)(68.692019,1.082272) +- (-0.006098,0.006098)(70.773595,1.079688) +- (-0.005911,0.005911)(72.855172,1.076701) +- (-0.005654,0.005654)(74.936748,1.074960) +- (-0.005455,0.005455)(77.018324,1.073440) +- (-0.005015,0.005015)(79.099901,1.072369) +- (-0.005023,0.005023)(81.181477,1.069682) +- (-0.004604,0.004604)(83.263053,1.067049) +- (-0.004394,0.004394)(83.263053,1.067049) +- (-0.004394,0.004394)(87.426206,1.064626) +- (-0.003902,0.003902)(91.589359,1.061626) +- (-0.003691,0.003691)(95.752511,1.058781) +- (-0.003089,0.003089)(99.915664,1.055204) +- (-0.003169,0.003169)(104.078817,1.053070) +- (-0.003068,0.003068)(108.241969,1.052601) +- (-0.003043,0.003043)(112.405122,1.051291) +- (-0.003274,0.003274)(116.568275,1.049381) +- (-0.002869,0.002869)(120.731427,1.046464) +- (-0.002446,0.002446)(124.894580,1.044508) +- (-0.002135,0.002135)(129.057733,1.042723) +- (-0.001906,0.001906)(133.220885,1.041596) +- (-0.001609,0.001609)(137.384038,1.040073) +- (-0.001626,0.001626)(141.547191,1.038906) +- (-0.001707,0.001707)(145.710343,1.037068) +- (-0.001711,0.001711)(149.873496,1.035746) +- (-0.001740,0.001740)(154.036649,1.035403) +- (-0.001843,0.001843)(158.199801,1.034326) +- (-0.001854,0.001854)(162.362954,1.033786) +- (-0.001946,0.001946)(166.526107,1.032581) +- (-0.001914,0.001914)(170.689260,1.031418) +- (-0.002025,0.002025)(174.852412,1.030285) +- (-0.001864,0.001864)(179.015565,1.029611) +- (-0.001913,0.001913)(183.178718,1.029166) +- (-0.001784,0.001784)(187.341870,1.028709) +- (-0.001842,0.001842)(191.505023,1.027724) +- (-0.001595,0.001595)(195.668176,1.027930) +- (-0.001725,0.001725)(199.831328,1.027410) +- (-0.001598,0.001598)(203.994481,1.026629) +- (-0.001686,0.001686)(208.157634,1.025749) +- (-0.001567,0.001567)(212.320786,1.025104) +- (-0.001682,0.001682)(216.483939,1.024412) +- (-0.001523,0.001523)(220.647092,1.023945) +- (-0.001585,0.001585)(224.810244,1.023386) +- (-0.001514,0.001514)(228.973397,1.023145) +- (-0.001471,0.001471)(233.136550,1.022896) +- (-0.001247,0.001247)(237.299702,1.022330) +- (-0.001029,0.001029)(241.462855,1.022282) +- (-0.000915,0.000915)(245.626008,1.021905) +- (-0.000910,0.000910)(249.789160,1.021576) +- (-0.000837,0.000837)(253.952313,1.020954) +- (-0.000812,0.000812)(258.115466,1.020817) +- (-0.000730,0.000730)(262.278618,1.020372) +- (-0.000697,0.000697)(266.441771,1.020010) +- (-0.000700,0.000700)(270.604924,1.019539) +- (-0.000698,0.000698)(274.768076,1.018817) +- (-0.000672,0.000672)(278.931229,1.018505) +- (-0.000631,0.000631)(283.094382,1.018239) +- (-0.000707,0.000707)(287.257534,1.017911) +- (-0.000711,0.000711)(291.420687,1.017973) +- (-0.000707,0.000707)(295.583840,1.018098) +- (-0.000686,0.000686)(299.746992,1.018082) +- (-0.000651,0.000651)(303.910145,1.017630) +- (-0.000644,0.000644)(308.073298,1.017207) +- (-0.000540,0.000540)(312.236450,1.016960) +- (-0.000642,0.000642)(316.399603,1.016609) +- (-0.000695,0.000695)(320.562756,1.016238) +- (-0.000718,0.000718)(324.725908,1.016241) +- (-0.000718,0.000718)(328.889061,1.015864) +- (-0.000714,0.000714)(333.052214,1.015742) +- (-0.000717,0.000717)(337.215366,1.015565) +- (-0.000714,0.000714)(341.378519,1.015409) +- (-0.000700,0.000700)(345.541672,1.015350) +- (-0.000724,0.000724)(349.704824,1.015021) +- (-0.000651,0.000651)(353.867977,1.014761) +- (-0.000689,0.000689)(358.031130,1.014615) +- (-0.000644,0.000644)(362.194282,1.014641) +- (-0.000653,0.000653)(366.357435,1.014255) +- (-0.000641,0.000641)(370.520588,1.013964) +- (-0.000609,0.000609)(374.683740,1.013802) +- (-0.000619,0.000619)(378.846893,1.013725) +- (-0.000605,0.000605)(383.010046,1.013625) +- (-0.000664,0.000664)(387.173198,1.013456) +- (-0.000694,0.000694)(391.336351,1.013288) +- (-0.000718,0.000718)(395.499504,1.013158) +- (-0.000668,0.000668)(399.662656,1.012956) +- (-0.000750,0.000750)(403.825809,1.012715) +- (-0.000736,0.000736)(407.988962,1.012572) +- (-0.000697,0.000697)(412.152114,1.012457) +- (-0.000702,0.000702)(416.315267,1.012276) +- (-0.000676,0.000676)
};
\addlegendentry{\textsc{Combo-I}}

\end{axis}
\end{tikzpicture}

    \vspace{\scspacey}
    \caption{\hspace{\scspacex}\textsc{Sensor}}
    \label{fig:berkeley1-time}
  \end{subfigure}
  \begin{subfigure}[b]{\subflen}
    \begin{tikzpicture}


\begin{axis}[%
tick label style={/pgf/number format/fixed,font=\sffamily\small},
label style={font=\sffamily\small},
legend style={font=\sffamily\small},
view={0}{90},
width=\figurewidth,
height=\figureheight,
xmin=0, xmax=8,
xtick={0, 2, 4, 6, 8},
xticklabels={0, 2, 4, 6, 8},
scaled x ticks=false,
xlabel={Time (ms)},
xlabel shift=0em,
ymin=1, ymax=1.52,
ytick={1, 1.5},
yticklabels={1, 1.5},
ylabel={PSRF},
ylabel shift=-1em,
major tick length=2pt,
axis lines*=left,
legend cell align=left,
clip marker paths=true,
legend style={anchor=north east,at={(1,1)},draw=none,row sep=0em},
every axis plot/.append style={
  line width=1.5pt,
  opacity=0.8,
}
]

\addplot [
color=gcol1,
densely dashed
]
coordinates{
(0.225668,4.516237) +- (-0.282843,0.282843)(0.263279,3.830157) +- (-0.282843,0.282843)(0.300890,3.273590) +- (-0.229806,0.229806)(0.338502,2.877201) +- (-0.176235,0.176235)(0.376113,2.695993) +- (-0.235574,0.235574)(0.413724,2.523592) +- (-0.261760,0.261760)(0.451335,2.249459) +- (-0.114257,0.114257)(0.488947,2.169126) +- (-0.110096,0.110096)(0.526558,2.111823) +- (-0.106434,0.106434)(0.564169,2.063507) +- (-0.120962,0.120962)(0.601780,1.973199) +- (-0.093382,0.093382)(0.639392,1.901398) +- (-0.078620,0.078620)(0.677003,1.853378) +- (-0.073803,0.073803)(0.714614,1.813945) +- (-0.072097,0.072097)(0.752226,1.779090) +- (-0.068775,0.068775)(0.789837,1.722247) +- (-0.064540,0.064540)(0.827448,1.667105) +- (-0.056937,0.056937)(0.865059,1.640735) +- (-0.052273,0.052273)(0.902671,1.610234) +- (-0.047158,0.047158)(0.940282,1.583779) +- (-0.045573,0.045573)(0.977893,1.561775) +- (-0.046127,0.046127)(1.015505,1.550530) +- (-0.050095,0.050095)(1.053116,1.531675) +- (-0.051499,0.051499)(1.090727,1.513150) +- (-0.049483,0.049483)(1.128338,1.492202) +- (-0.046282,0.046282)(1.165950,1.474442) +- (-0.044483,0.044483)(1.203561,1.452408) +- (-0.044128,0.044128)(1.241172,1.430268) +- (-0.041441,0.041441)(1.278784,1.404966) +- (-0.039547,0.039547)(1.316395,1.385699) +- (-0.037429,0.037429)(1.354006,1.367421) +- (-0.035029,0.035029)(1.391617,1.353516) +- (-0.032093,0.032093)(1.429229,1.343710) +- (-0.029860,0.029860)(1.466840,1.332883) +- (-0.028481,0.028481)(1.504451,1.324716) +- (-0.027671,0.027671)(1.504451,1.324716) +- (-0.027671,0.027671)(1.579674,1.308945) +- (-0.026693,0.026693)(1.654896,1.296945) +- (-0.026691,0.026691)(1.730119,1.284966) +- (-0.025629,0.025629)(1.805341,1.273506) +- (-0.023511,0.023511)(1.880564,1.263467) +- (-0.021502,0.021502)(1.955787,1.253765) +- (-0.020964,0.020964)(2.031009,1.243800) +- (-0.019552,0.019552)(2.106232,1.233648) +- (-0.019438,0.019438)(2.181454,1.227089) +- (-0.019500,0.019500)(2.256677,1.219713) +- (-0.019388,0.019388)(2.331899,1.208019) +- (-0.019812,0.019812)(2.407122,1.199651) +- (-0.019014,0.019014)(2.482345,1.193562) +- (-0.018731,0.018731)(2.557567,1.187544) +- (-0.018183,0.018183)(2.632790,1.179802) +- (-0.016622,0.016622)(2.708012,1.172516) +- (-0.014923,0.014923)(2.783235,1.168321) +- (-0.014335,0.014335)(2.858457,1.163460) +- (-0.014075,0.014075)(2.933680,1.158933) +- (-0.014031,0.014031)(3.008902,1.154228) +- (-0.013833,0.013833)(3.084125,1.151817) +- (-0.014465,0.014465)(3.159348,1.149964) +- (-0.014623,0.014623)(3.234570,1.145804) +- (-0.014158,0.014158)(3.309793,1.143005) +- (-0.013952,0.013952)(3.385015,1.139541) +- (-0.013979,0.013979)(3.460238,1.137012) +- (-0.013695,0.013695)(3.535460,1.134188) +- (-0.013797,0.013797)(3.610683,1.130174) +- (-0.013436,0.013436)(3.685906,1.125665) +- (-0.013443,0.013443)(3.761128,1.121501) +- (-0.013190,0.013190)(3.836351,1.117630) +- (-0.012638,0.012638)(3.911573,1.114589) +- (-0.011871,0.011871)(3.986796,1.111642) +- (-0.011341,0.011341)(4.062018,1.108481) +- (-0.010893,0.010893)(4.137241,1.106012) +- (-0.009948,0.009948)(4.212463,1.103306) +- (-0.009098,0.009098)(4.287686,1.100321) +- (-0.008507,0.008507)(4.362909,1.099021) +- (-0.008061,0.008061)(4.438131,1.098218) +- (-0.008023,0.008023)(4.513354,1.097804) +- (-0.008015,0.008015)(4.588576,1.096534) +- (-0.007935,0.007935)(4.663799,1.095241) +- (-0.007874,0.007874)(4.739021,1.093985) +- (-0.007794,0.007794)(4.814244,1.092387) +- (-0.007661,0.007661)(4.889466,1.090922) +- (-0.007757,0.007757)(4.964689,1.089581) +- (-0.007811,0.007811)(5.039912,1.088310) +- (-0.007826,0.007826)(5.115134,1.087297) +- (-0.007722,0.007722)(5.190357,1.086096) +- (-0.007698,0.007698)(5.265579,1.085107) +- (-0.007857,0.007857)(5.340802,1.084093) +- (-0.008249,0.008249)(5.416024,1.083511) +- (-0.008506,0.008506)(5.491247,1.082479) +- (-0.008429,0.008429)(5.566470,1.081092) +- (-0.008038,0.008038)(5.641692,1.080306) +- (-0.007893,0.007893)(5.716915,1.078561) +- (-0.007727,0.007727)(5.792137,1.076938) +- (-0.007521,0.007521)(5.867360,1.075676) +- (-0.007565,0.007565)(5.942582,1.074756) +- (-0.007669,0.007669)(6.017805,1.073462) +- (-0.007553,0.007553)(6.093027,1.072442) +- (-0.007401,0.007401)(6.168250,1.072135) +- (-0.007140,0.007140)(6.243473,1.072124) +- (-0.007132,0.007132)(6.318695,1.071491) +- (-0.007162,0.007162)(6.393918,1.070260) +- (-0.007096,0.007096)(6.469140,1.069295) +- (-0.006853,0.006853)(6.544363,1.068797) +- (-0.006657,0.006657)(6.619585,1.068430) +- (-0.006572,0.006572)(6.694808,1.067382) +- (-0.006425,0.006425)(6.770031,1.065649) +- (-0.006117,0.006117)(6.845253,1.064589) +- (-0.005855,0.005855)(6.920476,1.063943) +- (-0.005688,0.005688)(6.995698,1.063607) +- (-0.005502,0.005502)(7.070921,1.062877) +- (-0.005220,0.005220)(7.146143,1.062406) +- (-0.004978,0.004978)(7.221366,1.061307) +- (-0.004718,0.004718)(7.296588,1.060198) +- (-0.004461,0.004461)(7.371811,1.059266) +- (-0.004180,0.004180)(7.447034,1.058128) +- (-0.004156,0.004156)(7.522256,1.057079) +- (-0.004123,0.004123)
};
\addlegendentry{\textsc{Gibbs}}


\addplot [
color=gcol2
]
coordinates{
(0.277275,4.661337) +- (-0.282843,0.282843)(0.332730,3.126284) +- (-0.242005,0.242005)(0.388185,2.967250) +- (-0.167716,0.167716)(0.443640,2.623668) +- (-0.170227,0.170227)(0.499095,2.361560) +- (-0.137858,0.137858)(0.554550,2.129161) +- (-0.105712,0.105712)(0.610005,1.989369) +- (-0.094978,0.094978)(0.665460,1.888797) +- (-0.085844,0.085844)(0.720915,1.822956) +- (-0.078461,0.078461)(0.776371,1.774579) +- (-0.075635,0.075635)(0.831826,1.731360) +- (-0.068154,0.068154)(0.887281,1.679225) +- (-0.058198,0.058198)(0.942736,1.649613) +- (-0.066685,0.066685)(0.998191,1.605256) +- (-0.054783,0.054783)(1.053646,1.569987) +- (-0.044377,0.044377)(1.109101,1.546528) +- (-0.041463,0.041463)(1.164556,1.524030) +- (-0.042004,0.042004)(1.220011,1.493081) +- (-0.044614,0.044614)(1.275466,1.466954) +- (-0.042272,0.042272)(1.330921,1.429857) +- (-0.037523,0.037523)(1.386376,1.400105) +- (-0.035049,0.035049)(1.441831,1.376252) +- (-0.032698,0.032698)(1.497286,1.364049) +- (-0.033362,0.033362)(1.552741,1.353922) +- (-0.032834,0.032834)(1.608196,1.340536) +- (-0.029944,0.029944)(1.663651,1.325644) +- (-0.029390,0.029390)(1.719106,1.311411) +- (-0.028004,0.028004)(1.774561,1.295861) +- (-0.024657,0.024657)(1.830016,1.279375) +- (-0.019995,0.019995)(1.885471,1.266085) +- (-0.017414,0.017414)(1.940926,1.259734) +- (-0.015089,0.015089)(1.996381,1.250103) +- (-0.014532,0.014532)(2.051836,1.242180) +- (-0.014514,0.014514)(2.107291,1.234217) +- (-0.014231,0.014231)(2.162746,1.227696) +- (-0.014536,0.014536)(2.218201,1.222153) +- (-0.014443,0.014443)(2.218201,1.222153) +- (-0.014443,0.014443)(2.329112,1.209634) +- (-0.013958,0.013958)(2.440022,1.201998) +- (-0.013961,0.013961)(2.550932,1.194390) +- (-0.013502,0.013502)(2.661842,1.185438) +- (-0.011262,0.011262)(2.772752,1.175021) +- (-0.009839,0.009839)(2.883662,1.166439) +- (-0.009338,0.009338)(2.994572,1.162580) +- (-0.010043,0.010043)(3.105482,1.157887) +- (-0.009375,0.009375)(3.216392,1.153160) +- (-0.009534,0.009534)(3.327302,1.147789) +- (-0.008396,0.008396)(3.438212,1.142412) +- (-0.008222,0.008222)(3.549122,1.136853) +- (-0.007681,0.007681)(3.660032,1.132317) +- (-0.007172,0.007172)(3.770942,1.128387) +- (-0.007152,0.007152)(3.881853,1.127322) +- (-0.007361,0.007361)(3.992763,1.126262) +- (-0.008011,0.008011)(4.103673,1.123154) +- (-0.008377,0.008377)(4.214583,1.120948) +- (-0.008648,0.008648)(4.325493,1.119206) +- (-0.008226,0.008226)(4.436403,1.115463) +- (-0.008258,0.008258)(4.547313,1.111110) +- (-0.008232,0.008232)(4.658223,1.107482) +- (-0.007618,0.007618)(4.769133,1.104193) +- (-0.007594,0.007594)(4.880043,1.100875) +- (-0.007420,0.007420)(4.990953,1.098932) +- (-0.007387,0.007387)(5.101863,1.095846) +- (-0.006805,0.006805)(5.212773,1.092678) +- (-0.006562,0.006562)(5.323683,1.089171) +- (-0.005834,0.005834)(5.434594,1.086314) +- (-0.005523,0.005523)(5.545504,1.084169) +- (-0.005493,0.005493)(5.656414,1.082744) +- (-0.005255,0.005255)(5.767324,1.080025) +- (-0.005029,0.005029)(5.878234,1.078441) +- (-0.004886,0.004886)(5.989144,1.077176) +- (-0.004668,0.004668)(6.100054,1.075127) +- (-0.004626,0.004626)(6.210964,1.073319) +- (-0.004958,0.004958)(6.321874,1.072079) +- (-0.004992,0.004992)(6.432784,1.070496) +- (-0.005121,0.005121)(6.543694,1.070135) +- (-0.005211,0.005211)(6.654604,1.069205) +- (-0.005249,0.005249)(6.765514,1.067519) +- (-0.005002,0.005002)(6.876424,1.065964) +- (-0.004782,0.004782)(6.987335,1.064361) +- (-0.004910,0.004910)(7.098245,1.064042) +- (-0.004793,0.004793)(7.209155,1.063152) +- (-0.004550,0.004550)(7.320065,1.062110) +- (-0.004478,0.004478)(7.430975,1.061689) +- (-0.004374,0.004374)(7.541885,1.061245) +- (-0.004361,0.004361)(7.652795,1.060243) +- (-0.004316,0.004316)(7.763705,1.059101) +- (-0.004084,0.004084)(7.874615,1.058265) +- (-0.003884,0.003884)(7.985525,1.057640) +- (-0.003713,0.003713)(8.096435,1.056679) +- (-0.003582,0.003582)(8.207345,1.055696) +- (-0.003427,0.003427)(8.318255,1.054627) +- (-0.003177,0.003177)(8.429165,1.052992) +- (-0.002958,0.002958)(8.540076,1.051297) +- (-0.002805,0.002805)(8.650986,1.050598) +- (-0.002936,0.002936)(8.761896,1.050377) +- (-0.002957,0.002957)(8.872806,1.050465) +- (-0.002902,0.002902)(8.983716,1.050107) +- (-0.002878,0.002878)(9.094626,1.049838) +- (-0.002906,0.002906)(9.205536,1.049470) +- (-0.002780,0.002780)(9.316446,1.048656) +- (-0.002811,0.002811)(9.427356,1.047425) +- (-0.002809,0.002809)(9.538266,1.046920) +- (-0.002838,0.002838)(9.649176,1.046320) +- (-0.002906,0.002906)(9.760086,1.045553) +- (-0.002827,0.002827)(9.870996,1.044695) +- (-0.002820,0.002820)(9.981907,1.043789) +- (-0.002856,0.002856)(10.092817,1.043147) +- (-0.002794,0.002794)(10.203727,1.042884) +- (-0.002687,0.002687)(10.314637,1.042286) +- (-0.002513,0.002513)(10.425547,1.041712) +- (-0.002390,0.002390)(10.536457,1.041256) +- (-0.002369,0.002369)(10.647367,1.040269) +- (-0.002268,0.002268)(10.758277,1.039686) +- (-0.002173,0.002173)(10.869187,1.039570) +- (-0.002113,0.002113)(10.980097,1.039143) +- (-0.002055,0.002055)(11.091007,1.038411) +- (-0.001962,0.001962)
};
\addlegendentry{\textsc{Combo-R}}


\addplot [
color=gcol3
]
coordinates{
(0.229735,3.504238) +- (-0.282843,0.282843)(0.287169,2.807509) +- (-0.282843,0.282843)(0.344603,2.267132) +- (-0.132254,0.132254)(0.402036,2.053463) +- (-0.100202,0.100202)(0.459470,1.885144) +- (-0.097802,0.097802)(0.516904,1.798585) +- (-0.084779,0.084779)(0.574338,1.671529) +- (-0.053453,0.053453)(0.631771,1.580282) +- (-0.052860,0.052860)(0.689205,1.508231) +- (-0.043306,0.043306)(0.746639,1.471908) +- (-0.043179,0.043179)(0.804073,1.416112) +- (-0.030592,0.030592)(0.861506,1.386443) +- (-0.027724,0.027724)(0.918940,1.356410) +- (-0.026889,0.026889)(0.976374,1.338293) +- (-0.028158,0.028158)(1.033808,1.326141) +- (-0.028697,0.028697)(1.091242,1.312075) +- (-0.029212,0.029212)(1.148675,1.291193) +- (-0.023838,0.023838)(1.206109,1.272725) +- (-0.019471,0.019471)(1.263543,1.262250) +- (-0.020583,0.020583)(1.320977,1.249389) +- (-0.020325,0.020325)(1.378410,1.233877) +- (-0.016519,0.016519)(1.435844,1.226579) +- (-0.016325,0.016325)(1.493278,1.220713) +- (-0.016387,0.016387)(1.550712,1.213551) +- (-0.016487,0.016487)(1.608145,1.203788) +- (-0.015545,0.015545)(1.665579,1.196004) +- (-0.013996,0.013996)(1.723013,1.190473) +- (-0.013166,0.013166)(1.780447,1.185443) +- (-0.012677,0.012677)(1.837880,1.179233) +- (-0.012541,0.012541)(1.895314,1.171537) +- (-0.011839,0.011839)(1.952748,1.165376) +- (-0.010910,0.010910)(2.010182,1.157642) +- (-0.009882,0.009882)(2.067615,1.151277) +- (-0.008983,0.008983)(2.125049,1.147299) +- (-0.008794,0.008794)(2.182483,1.144925) +- (-0.008791,0.008791)(2.239917,1.141484) +- (-0.008848,0.008848)(2.297351,1.138509) +- (-0.008576,0.008576)(2.297351,1.138509) +- (-0.008576,0.008576)(2.412218,1.129799) +- (-0.008290,0.008290)(2.527086,1.122778) +- (-0.007940,0.007940)(2.641953,1.115578) +- (-0.007197,0.007197)(2.756821,1.109873) +- (-0.007159,0.007159)(2.871688,1.105875) +- (-0.007242,0.007242)(2.986556,1.101614) +- (-0.007005,0.007005)(3.101423,1.097087) +- (-0.006972,0.006972)(3.216291,1.095587) +- (-0.007530,0.007530)(3.331158,1.092937) +- (-0.006752,0.006752)(3.446026,1.087729) +- (-0.005510,0.005510)(3.560893,1.084350) +- (-0.005065,0.005065)(3.675761,1.081542) +- (-0.004927,0.004927)(3.790628,1.077401) +- (-0.004830,0.004830)(3.905496,1.074216) +- (-0.004113,0.004113)(4.020363,1.070874) +- (-0.003476,0.003476)(4.135231,1.069468) +- (-0.003160,0.003160)(4.250099,1.066446) +- (-0.002897,0.002897)(4.364966,1.064877) +- (-0.002838,0.002838)(4.479834,1.062515) +- (-0.002905,0.002905)(4.594701,1.059906) +- (-0.003026,0.003026)(4.709569,1.058294) +- (-0.002735,0.002735)(4.824436,1.056614) +- (-0.002723,0.002723)(4.939304,1.056915) +- (-0.002597,0.002597)(5.054171,1.056143) +- (-0.002341,0.002341)(5.169039,1.054420) +- (-0.002170,0.002170)(5.283906,1.053020) +- (-0.002251,0.002251)(5.398774,1.051299) +- (-0.002619,0.002619)(5.513641,1.050231) +- (-0.002590,0.002590)(5.628509,1.048521) +- (-0.002377,0.002377)(5.743376,1.047197) +- (-0.001962,0.001962)(5.858244,1.045708) +- (-0.001751,0.001751)(5.973111,1.044521) +- (-0.001872,0.001872)(6.087979,1.043858) +- (-0.002088,0.002088)(6.202846,1.042963) +- (-0.002304,0.002304)(6.317714,1.042081) +- (-0.002218,0.002218)(6.432582,1.041521) +- (-0.002119,0.002119)(6.547449,1.041528) +- (-0.001921,0.001921)(6.662317,1.041374) +- (-0.001909,0.001909)(6.777184,1.040812) +- (-0.001899,0.001899)(6.892052,1.039801) +- (-0.001871,0.001871)(7.006919,1.039643) +- (-0.001855,0.001855)(7.121787,1.039454) +- (-0.002077,0.002077)(7.236654,1.038837) +- (-0.002140,0.002140)(7.351522,1.038101) +- (-0.002097,0.002097)(7.466389,1.037054) +- (-0.001943,0.001943)(7.581257,1.036940) +- (-0.001977,0.001977)(7.696124,1.036014) +- (-0.001714,0.001714)(7.810992,1.035354) +- (-0.001655,0.001655)(7.925859,1.034671) +- (-0.001730,0.001730)(8.040727,1.034072) +- (-0.001843,0.001843)(8.155594,1.033544) +- (-0.001783,0.001783)(8.270462,1.033082) +- (-0.001805,0.001805)(8.385330,1.032906) +- (-0.001851,0.001851)(8.500197,1.032635) +- (-0.001895,0.001895)(8.615065,1.032710) +- (-0.001943,0.001943)(8.729932,1.032106) +- (-0.001749,0.001749)(8.844800,1.031760) +- (-0.001786,0.001786)(8.959667,1.031239) +- (-0.001746,0.001746)(9.074535,1.030631) +- (-0.001834,0.001834)(9.189402,1.030050) +- (-0.001910,0.001910)(9.304270,1.029740) +- (-0.001989,0.001989)(9.419137,1.029658) +- (-0.001968,0.001968)(9.534005,1.029153) +- (-0.001889,0.001889)(9.648872,1.028772) +- (-0.001761,0.001761)(9.763740,1.028243) +- (-0.001698,0.001698)(9.878607,1.027599) +- (-0.001659,0.001659)(9.993475,1.027296) +- (-0.001738,0.001738)(10.108342,1.026791) +- (-0.001802,0.001802)(10.223210,1.026763) +- (-0.001772,0.001772)(10.338077,1.026613) +- (-0.001707,0.001707)(10.452945,1.026540) +- (-0.001617,0.001617)(10.567813,1.026155) +- (-0.001686,0.001686)(10.682680,1.025719) +- (-0.001648,0.001648)(10.797548,1.025094) +- (-0.001527,0.001527)(10.912415,1.024802) +- (-0.001489,0.001489)(11.027283,1.024611) +- (-0.001533,0.001533)(11.142150,1.024479) +- (-0.001440,0.001440)(11.257018,1.024159) +- (-0.001359,0.001359)(11.371885,1.023814) +- (-0.001304,0.001304)(11.486753,1.023418) +- (-0.001323,0.001323)
};
\addlegendentry{\textsc{Combo-I}}

\end{axis}
\end{tikzpicture}

    \vspace{\scspacey}
    \caption{\hspace{\scspacex}\textsc{Game}}
    \label{fig:hots1-time}
  \end{subfigure}
  \caption{
    (a)-(c) Ising model results for increasing $n$. Note how the Gibbs sampler gets worse significantly faster than the combined ones.
    (d)-(f) Potential scale reduction factor (PSRF) as a function of sampling iterations.
    (g)-(i) PSRF as a function of wall-clock time in milliseconds.
    The combined sampler outperforms Gibbs both in terms of samples required, as well as actual runtime.
  }
  \label{fig:expising}
\end{figure*}

\setlength\figureheight{0.44\textwidth}
\setlength\figurewidth{0.48\textwidth}
\renewcommand{\subflen}{0.46\textwidth}
\renewcommand{\scspacey}{-0.3em}
\renewcommand{\scspacex}{0.2em}
\begin{figure*}[t!]
  \captionsetup[subfigure]{oneside,margin={2em,0em}}
  \centering
  \begin{subfigure}[b]{\subflen}
    \centering
    \begin{tikzpicture}

\colorlet{col1}{blue!50!black}
\colorlet{col2}{red!10!darkgray}
\colorlet{col3}{red!30!darkgray}
\colorlet{col4}{red!50!darkgray}
\colorlet{col5}{red!70!darkgray}
\colorlet{col6}{red!90!darkgray}


\begin{axis}[%
tick label style={font=\tiny},
label style={font=\scriptsize},
legend style={font=\tiny},
view={0}{90},
width=\figurewidth,
height=\figureheight,
xmin=0, xmax=5000,
xtick={0, 1000, 2000, 3000, 4000, 5000},
xticklabels={0, 1k, 2k, 3k, 4k, 5k},
scaled x ticks=false,
xlabel={Samples},
xlabel shift=-0.3em,
ymin=1, ymax=1.52,
ytick={1, 1.5},
ylabel={PSRF},
ylabel shift=-1.5em,
tick label style={/pgf/number format/fixed},
major tick length=2pt,
axis lines*=left,
legend cell align=left,
clip marker paths=true,
legend style={at={(1.05,1.05)},draw=none,row sep=-0.35em}]

\addplot [
mark=none,
mark size=1.0pt,
mark options={solid},
color=col1,
densely dashed,
line width=1pt,
opacity=0.7,
%error bars/.cd,
%error bar style={solid, line width=0.2pt},
%y dir=both,
%y explicit
]
coordinates{
(200.000000,3.597939) +- (-0.282843,0.282843)(225.000000,3.040631) +- (-0.197440,0.197440)(250.000000,2.691309) +- (-0.160233,0.160233)(275.000000,2.554452) +- (-0.165238,0.165238)(300.000000,2.413226) +- (-0.160241,0.160241)(325.000000,2.330296) +- (-0.148466,0.148466)(350.000000,2.215460) +- (-0.130796,0.130796)(375.000000,2.169512) +- (-0.130263,0.130263)(400.000000,2.082272) +- (-0.130343,0.130343)(425.000000,2.028475) +- (-0.144012,0.144012)(450.000000,2.014070) +- (-0.189635,0.189635)(475.000000,2.059465) +- (-0.282843,0.282843)(500.000000,1.948577) +- (-0.207447,0.207447)(525.000000,1.888214) +- (-0.187895,0.187895)(550.000000,1.800753) +- (-0.146235,0.146235)(575.000000,1.749965) +- (-0.125414,0.125414)(600.000000,1.717381) +- (-0.149604,0.149604)(625.000000,1.704961) +- (-0.179373,0.179373)(650.000000,1.685182) +- (-0.188536,0.188536)(675.000000,1.651440) +- (-0.194696,0.194696)(700.000000,1.565033) +- (-0.110935,0.110935)(725.000000,1.494971) +- (-0.060617,0.060617)(750.000000,1.448536) +- (-0.037580,0.037580)(775.000000,1.423904) +- (-0.030389,0.030389)(800.000000,1.404557) +- (-0.025724,0.025724)(825.000000,1.388615) +- (-0.024165,0.024165)(850.000000,1.375496) +- (-0.022723,0.022723)(875.000000,1.365812) +- (-0.022431,0.022431)(900.000000,1.359307) +- (-0.022667,0.022667)(925.000000,1.357107) +- (-0.023374,0.023374)(950.000000,1.353670) +- (-0.024974,0.024974)(975.000000,1.349603) +- (-0.026614,0.026614)(1000.000000,1.341914) +- (-0.026115,0.026115)(1000.000000,1.341914) +- (-0.026115,0.026115)(1050.000000,1.327010) +- (-0.025499,0.025499)(1100.000000,1.309038) +- (-0.022776,0.022776)(1150.000000,1.291903) +- (-0.021510,0.021510)(1200.000000,1.276404) +- (-0.019627,0.019627)(1250.000000,1.265232) +- (-0.019791,0.019791)(1300.000000,1.257915) +- (-0.022160,0.022160)(1350.000000,1.245602) +- (-0.022382,0.022382)(1400.000000,1.229363) +- (-0.019007,0.019007)(1450.000000,1.212160) +- (-0.014853,0.014853)(1500.000000,1.202877) +- (-0.014921,0.014921)(1550.000000,1.194305) +- (-0.016019,0.016019)(1600.000000,1.187061) +- (-0.016566,0.016566)(1650.000000,1.181264) +- (-0.015779,0.015779)(1700.000000,1.176979) +- (-0.015684,0.015684)(1750.000000,1.173301) +- (-0.016778,0.016778)(1800.000000,1.166561) +- (-0.015625,0.015625)(1850.000000,1.160849) +- (-0.014327,0.014327)(1900.000000,1.157492) +- (-0.013240,0.013240)(1950.000000,1.152301) +- (-0.012648,0.012648)(2000.000000,1.146251) +- (-0.010389,0.010389)(2050.000000,1.142146) +- (-0.008755,0.008755)(2100.000000,1.138389) +- (-0.008279,0.008279)(2150.000000,1.134830) +- (-0.007781,0.007781)(2200.000000,1.131334) +- (-0.007385,0.007385)(2250.000000,1.129364) +- (-0.007120,0.007120)(2300.000000,1.127263) +- (-0.007972,0.007972)(2350.000000,1.124395) +- (-0.008520,0.008520)(2400.000000,1.119219) +- (-0.008856,0.008856)(2450.000000,1.117366) +- (-0.010033,0.010033)(2500.000000,1.114105) +- (-0.008819,0.008819)(2550.000000,1.113725) +- (-0.008468,0.008468)(2600.000000,1.111648) +- (-0.008166,0.008166)(2650.000000,1.109619) +- (-0.007836,0.007836)(2700.000000,1.107001) +- (-0.007309,0.007309)(2750.000000,1.104818) +- (-0.006968,0.006968)(2800.000000,1.104075) +- (-0.006324,0.006324)(2850.000000,1.102976) +- (-0.006398,0.006398)(2900.000000,1.100722) +- (-0.006288,0.006288)(2950.000000,1.097873) +- (-0.005578,0.005578)(3000.000000,1.095111) +- (-0.005033,0.005033)(3050.000000,1.093478) +- (-0.005158,0.005158)(3100.000000,1.091610) +- (-0.005177,0.005177)(3150.000000,1.090405) +- (-0.005001,0.005001)(3200.000000,1.089527) +- (-0.004982,0.004982)(3250.000000,1.088855) +- (-0.004821,0.004821)(3300.000000,1.087810) +- (-0.004819,0.004819)(3350.000000,1.086178) +- (-0.004890,0.004890)(3400.000000,1.083872) +- (-0.004823,0.004823)(3450.000000,1.082651) +- (-0.004757,0.004757)(3500.000000,1.082000) +- (-0.004411,0.004411)(3550.000000,1.081654) +- (-0.004404,0.004404)(3600.000000,1.080664) +- (-0.004261,0.004261)(3650.000000,1.079681) +- (-0.004000,0.004000)(3700.000000,1.078145) +- (-0.004157,0.004157)(3750.000000,1.077135) +- (-0.004285,0.004285)(3800.000000,1.076356) +- (-0.004383,0.004383)(3850.000000,1.075704) +- (-0.004286,0.004286)(3900.000000,1.074393) +- (-0.004038,0.004038)(3950.000000,1.072686) +- (-0.003808,0.003808)(4000.000000,1.071898) +- (-0.003671,0.003671)(4050.000000,1.070577) +- (-0.003569,0.003569)(4100.000000,1.068997) +- (-0.003373,0.003373)(4150.000000,1.067786) +- (-0.003359,0.003359)(4200.000000,1.067305) +- (-0.003437,0.003437)(4250.000000,1.066197) +- (-0.003413,0.003413)(4300.000000,1.064406) +- (-0.003289,0.003289)(4350.000000,1.062878) +- (-0.003119,0.003119)(4400.000000,1.061796) +- (-0.003118,0.003118)(4450.000000,1.061438) +- (-0.003328,0.003328)(4500.000000,1.061019) +- (-0.003352,0.003352)(4550.000000,1.060327) +- (-0.003275,0.003275)(4600.000000,1.059904) +- (-0.003248,0.003248)(4650.000000,1.059435) +- (-0.003126,0.003126)(4700.000000,1.058691) +- (-0.002941,0.002941)(4750.000000,1.057888) +- (-0.002763,0.002763)(4800.000000,1.057200) +- (-0.002759,0.002759)(4850.000000,1.056826) +- (-0.002693,0.002693)(4900.000000,1.056174) +- (-0.002721,0.002721)(4950.000000,1.055308) +- (-0.002654,0.002654)(5000.000000,1.054398) +- (-0.002602,0.002602)
};
\addlegendentry{\textsc{Gibbs}}


\addplot [
mark=none,
mark size=1.0pt,
color=col2,
line width=1pt,
opacity=0.7,
%error bars/.cd,
%error bar style={line width=0.2pt},
%y dir=both,
%y explicit
]
coordinates{
(100.000000,5.403676) +- (-0.282843,0.282843)(125.000000,4.344338) +- (-0.282843,0.282843)(150.000000,3.067032) +- (-0.282843,0.282843)(175.000000,2.696809) +- (-0.282843,0.282843)(200.000000,3.376187) +- (-0.282843,0.282843)(225.000000,4.559771) +- (-0.282843,0.282843)(250.000000,2.246080) +- (-0.282843,0.282843)(275.000000,1.964834) +- (-0.185316,0.185316)(300.000000,1.784628) +- (-0.127888,0.127888)(325.000000,1.680514) +- (-0.107534,0.107534)(350.000000,1.589423) +- (-0.094703,0.094703)(375.000000,1.515449) +- (-0.063199,0.063199)(400.000000,1.472075) +- (-0.053713,0.053713)(425.000000,1.459784) +- (-0.064499,0.064499)(450.000000,1.457546) +- (-0.081104,0.081104)(475.000000,1.451225) +- (-0.079251,0.079251)(500.000000,1.445955) +- (-0.080153,0.080153)(525.000000,1.439693) +- (-0.080812,0.080812)(550.000000,1.432871) +- (-0.081455,0.081455)(575.000000,1.413539) +- (-0.078023,0.078023)(600.000000,1.385705) +- (-0.069002,0.069002)(625.000000,1.367772) +- (-0.060266,0.060266)(650.000000,1.350758) +- (-0.054314,0.054314)(675.000000,1.329151) +- (-0.046619,0.046619)(700.000000,1.305000) +- (-0.036261,0.036261)(725.000000,1.283742) +- (-0.028588,0.028588)(750.000000,1.268143) +- (-0.025150,0.025150)(775.000000,1.259671) +- (-0.024374,0.024374)(800.000000,1.251229) +- (-0.023065,0.023065)(825.000000,1.239646) +- (-0.021773,0.021773)(850.000000,1.230482) +- (-0.019648,0.019648)(875.000000,1.220606) +- (-0.018599,0.018599)(900.000000,1.212447) +- (-0.018796,0.018796)(925.000000,1.204912) +- (-0.018889,0.018889)(950.000000,1.197974) +- (-0.018360,0.018360)(975.000000,1.191052) +- (-0.018503,0.018503)(1000.000000,1.186391) +- (-0.018135,0.018135)(1000.000000,1.186391) +- (-0.018135,0.018135)(1050.000000,1.178539) +- (-0.017122,0.017122)(1100.000000,1.174457) +- (-0.018184,0.018184)(1150.000000,1.171899) +- (-0.017842,0.017842)(1200.000000,1.164697) +- (-0.016771,0.016771)(1250.000000,1.157515) +- (-0.017382,0.017382)(1300.000000,1.151884) +- (-0.017248,0.017248)(1350.000000,1.147094) +- (-0.017157,0.017157)(1400.000000,1.140889) +- (-0.016758,0.016758)(1450.000000,1.135934) +- (-0.015340,0.015340)(1500.000000,1.130417) +- (-0.013671,0.013671)(1550.000000,1.125591) +- (-0.012754,0.012754)(1600.000000,1.122430) +- (-0.011792,0.011792)(1650.000000,1.119861) +- (-0.012723,0.012723)(1700.000000,1.119794) +- (-0.014864,0.014864)(1750.000000,1.115768) +- (-0.014658,0.014658)(1800.000000,1.111706) +- (-0.013932,0.013932)(1850.000000,1.108160) +- (-0.013902,0.013902)(1900.000000,1.103430) +- (-0.012841,0.012841)(1950.000000,1.100379) +- (-0.012126,0.012126)(2000.000000,1.096380) +- (-0.011892,0.011892)(2050.000000,1.094000) +- (-0.011867,0.011867)(2100.000000,1.089561) +- (-0.010691,0.010691)(2150.000000,1.086178) +- (-0.009731,0.009731)(2200.000000,1.084804) +- (-0.009150,0.009150)(2250.000000,1.083735) +- (-0.009293,0.009293)(2300.000000,1.082506) +- (-0.008705,0.008705)(2350.000000,1.079765) +- (-0.008467,0.008467)(2400.000000,1.077203) +- (-0.007935,0.007935)(2450.000000,1.076121) +- (-0.007413,0.007413)(2500.000000,1.075686) +- (-0.007112,0.007112)(2550.000000,1.073912) +- (-0.006443,0.006443)(2600.000000,1.072339) +- (-0.005712,0.005712)(2650.000000,1.070747) +- (-0.005171,0.005171)(2700.000000,1.068923) +- (-0.005035,0.005035)(2750.000000,1.066460) +- (-0.004830,0.004830)(2800.000000,1.065951) +- (-0.004740,0.004740)(2850.000000,1.065168) +- (-0.004830,0.004830)(2900.000000,1.065266) +- (-0.005001,0.005001)(2950.000000,1.064770) +- (-0.005249,0.005249)(3000.000000,1.062497) +- (-0.005139,0.005139)(3050.000000,1.060510) +- (-0.005006,0.005006)(3100.000000,1.059121) +- (-0.004939,0.004939)(3150.000000,1.058333) +- (-0.004818,0.004818)(3200.000000,1.057472) +- (-0.004557,0.004557)(3250.000000,1.056127) +- (-0.004327,0.004327)(3300.000000,1.055808) +- (-0.004154,0.004154)(3350.000000,1.055894) +- (-0.003962,0.003962)(3400.000000,1.056042) +- (-0.003910,0.003910)(3450.000000,1.056239) +- (-0.003987,0.003987)(3500.000000,1.056185) +- (-0.003958,0.003958)(3550.000000,1.055649) +- (-0.003836,0.003836)(3600.000000,1.055032) +- (-0.003845,0.003845)(3650.000000,1.054584) +- (-0.003986,0.003986)(3700.000000,1.053425) +- (-0.003780,0.003780)(3750.000000,1.052105) +- (-0.003515,0.003515)(3800.000000,1.051122) +- (-0.003331,0.003331)(3850.000000,1.050448) +- (-0.003333,0.003333)(3900.000000,1.050297) +- (-0.003594,0.003594)(3950.000000,1.049273) +- (-0.003620,0.003620)(4000.000000,1.048646) +- (-0.003449,0.003449)(4050.000000,1.047940) +- (-0.003289,0.003289)(4100.000000,1.046913) +- (-0.003150,0.003150)(4150.000000,1.046020) +- (-0.003018,0.003018)(4200.000000,1.045200) +- (-0.002939,0.002939)(4250.000000,1.044950) +- (-0.002878,0.002878)(4300.000000,1.044865) +- (-0.003001,0.003001)(4350.000000,1.044050) +- (-0.002956,0.002956)(4400.000000,1.043050) +- (-0.002821,0.002821)(4450.000000,1.042117) +- (-0.002638,0.002638)(4500.000000,1.041479) +- (-0.002572,0.002572)(4550.000000,1.040948) +- (-0.002561,0.002561)(4600.000000,1.040341) +- (-0.002676,0.002676)(4650.000000,1.040258) +- (-0.002852,0.002852)(4700.000000,1.040003) +- (-0.002879,0.002879)(4750.000000,1.039344) +- (-0.002813,0.002813)(4800.000000,1.038651) +- (-0.002776,0.002776)(4850.000000,1.038186) +- (-0.002696,0.002696)(4900.000000,1.037828) +- (-0.002685,0.002685)(4950.000000,1.037114) +- (-0.002590,0.002590)(5000.000000,1.036366) +- (-0.002551,0.002551)
};
\addlegendentry{$r = 20$}


\addplot [
mark=none,
mark size=1.0pt,
color=col3,
line width=1pt,
opacity=0.7,
%error bars/.cd,
%error bar style={line width=0.2pt},
%y dir=both,
%y explicit
]
coordinates{
(100.000000,2.858444) +- (-0.266819,0.266819)(125.000000,2.491132) +- (-0.165815,0.165815)(150.000000,2.246981) +- (-0.159540,0.159540)(175.000000,2.178872) +- (-0.181578,0.181578)(200.000000,2.092144) +- (-0.221822,0.221822)(225.000000,1.954512) +- (-0.254781,0.254781)(250.000000,1.768028) +- (-0.186420,0.186420)(275.000000,1.630692) +- (-0.124710,0.124710)(300.000000,1.567989) +- (-0.092431,0.092431)(325.000000,1.516456) +- (-0.076384,0.076384)(350.000000,1.500579) +- (-0.092892,0.092892)(375.000000,1.463941) +- (-0.064656,0.064656)(400.000000,1.440297) +- (-0.065494,0.065494)(425.000000,1.412152) +- (-0.059590,0.059590)(450.000000,1.387921) +- (-0.057134,0.057134)(475.000000,1.355500) +- (-0.052104,0.052104)(500.000000,1.326939) +- (-0.043413,0.043413)(525.000000,1.309394) +- (-0.037545,0.037545)(550.000000,1.294237) +- (-0.036485,0.036485)(575.000000,1.277982) +- (-0.032771,0.032771)(600.000000,1.265835) +- (-0.028365,0.028365)(625.000000,1.253500) +- (-0.026067,0.026067)(650.000000,1.244431) +- (-0.026889,0.026889)(675.000000,1.236399) +- (-0.028921,0.028921)(700.000000,1.226554) +- (-0.027247,0.027247)(725.000000,1.220846) +- (-0.026363,0.026363)(750.000000,1.212396) +- (-0.022276,0.022276)(775.000000,1.205434) +- (-0.021128,0.021128)(800.000000,1.199273) +- (-0.020399,0.020399)(825.000000,1.191688) +- (-0.018351,0.018351)(850.000000,1.183152) +- (-0.016516,0.016516)(875.000000,1.175926) +- (-0.015391,0.015391)(900.000000,1.169669) +- (-0.013869,0.013869)(925.000000,1.162534) +- (-0.012467,0.012467)(950.000000,1.155901) +- (-0.011695,0.011695)(975.000000,1.152511) +- (-0.011413,0.011413)(1000.000000,1.147128) +- (-0.009896,0.009896)(1000.000000,1.147128) +- (-0.009896,0.009896)(1050.000000,1.144933) +- (-0.009177,0.009177)(1100.000000,1.139846) +- (-0.010169,0.010169)(1150.000000,1.132106) +- (-0.009437,0.009437)(1200.000000,1.124074) +- (-0.008694,0.008694)(1250.000000,1.120460) +- (-0.008444,0.008444)(1300.000000,1.113415) +- (-0.007481,0.007481)(1350.000000,1.108088) +- (-0.006546,0.006546)(1400.000000,1.107634) +- (-0.007093,0.007093)(1450.000000,1.103577) +- (-0.007004,0.007004)(1500.000000,1.098657) +- (-0.006233,0.006233)(1550.000000,1.095849) +- (-0.006941,0.006941)(1600.000000,1.092887) +- (-0.006798,0.006798)(1650.000000,1.089551) +- (-0.006268,0.006268)(1700.000000,1.086480) +- (-0.005759,0.005759)(1750.000000,1.083603) +- (-0.005315,0.005315)(1800.000000,1.081458) +- (-0.004733,0.004733)(1850.000000,1.079249) +- (-0.004610,0.004610)(1900.000000,1.077024) +- (-0.004442,0.004442)(1950.000000,1.075634) +- (-0.004377,0.004377)(2000.000000,1.074012) +- (-0.004383,0.004383)(2050.000000,1.071404) +- (-0.004408,0.004408)(2100.000000,1.069117) +- (-0.003965,0.003965)(2150.000000,1.066990) +- (-0.003683,0.003683)(2200.000000,1.065102) +- (-0.003665,0.003665)(2250.000000,1.064663) +- (-0.003922,0.003922)(2300.000000,1.063129) +- (-0.004062,0.004062)(2350.000000,1.061615) +- (-0.004221,0.004221)(2400.000000,1.060061) +- (-0.004069,0.004069)(2450.000000,1.058164) +- (-0.003811,0.003811)(2500.000000,1.056896) +- (-0.003694,0.003694)(2550.000000,1.055237) +- (-0.003416,0.003416)(2600.000000,1.054155) +- (-0.003209,0.003209)(2650.000000,1.053108) +- (-0.003411,0.003411)(2700.000000,1.052888) +- (-0.003421,0.003421)(2750.000000,1.052100) +- (-0.003305,0.003305)(2800.000000,1.050511) +- (-0.003182,0.003182)(2850.000000,1.049439) +- (-0.003048,0.003048)(2900.000000,1.048016) +- (-0.003093,0.003093)(2950.000000,1.046923) +- (-0.003103,0.003103)(3000.000000,1.045878) +- (-0.003122,0.003122)(3050.000000,1.045242) +- (-0.003067,0.003067)(3100.000000,1.044445) +- (-0.002984,0.002984)(3150.000000,1.043790) +- (-0.002888,0.002888)(3200.000000,1.043360) +- (-0.002603,0.002603)(3250.000000,1.042641) +- (-0.002458,0.002458)(3300.000000,1.041578) +- (-0.002298,0.002298)(3350.000000,1.041374) +- (-0.002344,0.002344)(3400.000000,1.040783) +- (-0.002385,0.002385)(3450.000000,1.040204) +- (-0.002337,0.002337)(3500.000000,1.040409) +- (-0.002210,0.002210)(3550.000000,1.039803) +- (-0.002255,0.002255)(3600.000000,1.039397) +- (-0.002305,0.002305)(3650.000000,1.038764) +- (-0.002230,0.002230)(3700.000000,1.038144) +- (-0.002252,0.002252)(3750.000000,1.037582) +- (-0.002190,0.002190)(3800.000000,1.036933) +- (-0.002211,0.002211)(3850.000000,1.036634) +- (-0.002069,0.002069)(3900.000000,1.036694) +- (-0.002128,0.002128)(3950.000000,1.036327) +- (-0.001999,0.001999)(4000.000000,1.035580) +- (-0.001907,0.001907)(4050.000000,1.035422) +- (-0.001824,0.001824)(4100.000000,1.035429) +- (-0.002016,0.002016)(4150.000000,1.035284) +- (-0.002109,0.002109)(4200.000000,1.034571) +- (-0.002109,0.002109)(4250.000000,1.034300) +- (-0.002130,0.002130)(4300.000000,1.033974) +- (-0.002069,0.002069)(4350.000000,1.033230) +- (-0.002041,0.002041)(4400.000000,1.032784) +- (-0.002024,0.002024)(4450.000000,1.032074) +- (-0.001949,0.001949)(4500.000000,1.031654) +- (-0.001848,0.001848)(4550.000000,1.031360) +- (-0.001727,0.001727)(4600.000000,1.031193) +- (-0.001653,0.001653)(4650.000000,1.031117) +- (-0.001768,0.001768)(4700.000000,1.031055) +- (-0.001815,0.001815)(4750.000000,1.030664) +- (-0.001733,0.001733)(4800.000000,1.030410) +- (-0.001778,0.001778)(4850.000000,1.030256) +- (-0.001650,0.001650)(4900.000000,1.029738) +- (-0.001514,0.001514)(4950.000000,1.029198) +- (-0.001463,0.001463)(5000.000000,1.028846) +- (-0.001549,0.001549)
};
\addlegendentry{$r = 40$}


\addplot [
mark=none,
mark size=1.0pt,
color=col4,
line width=1pt,
opacity=0.7,
%error bars/.cd,
%error bar style={line width=0.2pt},
%y dir=both,
%y explicit
]
coordinates{
(100.000000,2.887674) +- (-0.282843,0.282843)(125.000000,2.154315) +- (-0.232452,0.232452)(150.000000,1.886131) +- (-0.152196,0.152196)(175.000000,1.670072) +- (-0.104888,0.104888)(200.000000,1.598028) +- (-0.116591,0.116591)(225.000000,1.567810) +- (-0.148923,0.148923)(250.000000,1.464175) +- (-0.053544,0.053544)(275.000000,1.430135) +- (-0.055676,0.055676)(300.000000,1.422228) +- (-0.080558,0.080558)(325.000000,1.390277) +- (-0.069929,0.069929)(350.000000,1.346682) +- (-0.052223,0.052223)(375.000000,1.326571) +- (-0.048635,0.048635)(400.000000,1.312761) +- (-0.046422,0.046422)(425.000000,1.297058) +- (-0.041875,0.041875)(450.000000,1.277646) +- (-0.042758,0.042758)(475.000000,1.260397) +- (-0.046207,0.046207)(500.000000,1.251371) +- (-0.052000,0.052000)(525.000000,1.240265) +- (-0.053958,0.053958)(550.000000,1.223396) +- (-0.047093,0.047093)(575.000000,1.207225) +- (-0.037487,0.037487)(600.000000,1.192956) +- (-0.028284,0.028284)(625.000000,1.180323) +- (-0.023791,0.023791)(650.000000,1.170038) +- (-0.020447,0.020447)(675.000000,1.162271) +- (-0.018091,0.018091)(700.000000,1.154909) +- (-0.015529,0.015529)(725.000000,1.148557) +- (-0.012511,0.012511)(750.000000,1.142307) +- (-0.010907,0.010907)(775.000000,1.136904) +- (-0.010663,0.010663)(800.000000,1.132563) +- (-0.011488,0.011488)(825.000000,1.127129) +- (-0.010669,0.010669)(850.000000,1.122797) +- (-0.008925,0.008925)(875.000000,1.117632) +- (-0.008345,0.008345)(900.000000,1.114221) +- (-0.007899,0.007899)(925.000000,1.112350) +- (-0.008337,0.008337)(950.000000,1.109488) +- (-0.007956,0.007956)(975.000000,1.106682) +- (-0.007510,0.007510)(1000.000000,1.103503) +- (-0.007267,0.007267)(1000.000000,1.103503) +- (-0.007267,0.007267)(1050.000000,1.098749) +- (-0.007127,0.007127)(1100.000000,1.095345) +- (-0.007821,0.007821)(1150.000000,1.090872) +- (-0.007193,0.007193)(1200.000000,1.088033) +- (-0.007325,0.007325)(1250.000000,1.083971) +- (-0.007600,0.007600)(1300.000000,1.081325) +- (-0.006990,0.006990)(1350.000000,1.079809) +- (-0.007414,0.007414)(1400.000000,1.076582) +- (-0.007396,0.007396)(1450.000000,1.073844) +- (-0.007008,0.007008)(1500.000000,1.072277) +- (-0.006875,0.006875)(1550.000000,1.068138) +- (-0.005623,0.005623)(1600.000000,1.066610) +- (-0.005612,0.005612)(1650.000000,1.063895) +- (-0.005069,0.005069)(1700.000000,1.061094) +- (-0.004700,0.004700)(1750.000000,1.059636) +- (-0.004278,0.004278)(1800.000000,1.058215) +- (-0.003583,0.003583)(1850.000000,1.056543) +- (-0.003198,0.003198)(1900.000000,1.054705) +- (-0.003410,0.003410)(1950.000000,1.053093) +- (-0.003522,0.003522)(2000.000000,1.052636) +- (-0.003212,0.003212)(2050.000000,1.051480) +- (-0.003006,0.003006)(2100.000000,1.050732) +- (-0.002905,0.002905)(2150.000000,1.049432) +- (-0.002669,0.002669)(2200.000000,1.047261) +- (-0.002588,0.002588)(2250.000000,1.045923) +- (-0.002399,0.002399)(2300.000000,1.045401) +- (-0.002152,0.002152)(2350.000000,1.044952) +- (-0.002281,0.002281)(2400.000000,1.044028) +- (-0.002220,0.002220)(2450.000000,1.042755) +- (-0.002391,0.002391)(2500.000000,1.041612) +- (-0.002199,0.002199)(2550.000000,1.040239) +- (-0.002118,0.002118)(2600.000000,1.038457) +- (-0.001987,0.001987)(2650.000000,1.037662) +- (-0.001870,0.001870)(2700.000000,1.037233) +- (-0.001886,0.001886)(2750.000000,1.036139) +- (-0.001666,0.001666)(2800.000000,1.035108) +- (-0.001490,0.001490)(2850.000000,1.034785) +- (-0.001441,0.001441)(2900.000000,1.034630) +- (-0.001673,0.001673)(2950.000000,1.034518) +- (-0.001680,0.001680)(3000.000000,1.033988) +- (-0.001549,0.001549)(3050.000000,1.033539) +- (-0.001629,0.001629)(3100.000000,1.032695) +- (-0.001471,0.001471)(3150.000000,1.031867) +- (-0.001358,0.001358)(3200.000000,1.030795) +- (-0.001280,0.001280)(3250.000000,1.030160) +- (-0.001428,0.001428)(3300.000000,1.029986) +- (-0.001597,0.001597)(3350.000000,1.029612) +- (-0.001749,0.001749)(3400.000000,1.029439) +- (-0.001910,0.001910)(3450.000000,1.029058) +- (-0.001925,0.001925)(3500.000000,1.028892) +- (-0.001895,0.001895)(3550.000000,1.028531) +- (-0.001889,0.001889)(3600.000000,1.028213) +- (-0.001898,0.001898)(3650.000000,1.027942) +- (-0.001895,0.001895)(3700.000000,1.027303) +- (-0.001819,0.001819)(3750.000000,1.027047) +- (-0.001885,0.001885)(3800.000000,1.026708) +- (-0.001934,0.001934)(3850.000000,1.026611) +- (-0.002093,0.002093)(3900.000000,1.026426) +- (-0.002159,0.002159)(3950.000000,1.026300) +- (-0.002140,0.002140)(4000.000000,1.026256) +- (-0.002154,0.002154)(4050.000000,1.026239) +- (-0.002260,0.002260)(4100.000000,1.025802) +- (-0.002195,0.002195)(4150.000000,1.025438) +- (-0.002181,0.002181)(4200.000000,1.025129) +- (-0.002182,0.002182)(4250.000000,1.024766) +- (-0.002257,0.002257)(4300.000000,1.024572) +- (-0.002170,0.002170)(4350.000000,1.024085) +- (-0.002091,0.002091)(4400.000000,1.023935) +- (-0.001963,0.001963)(4450.000000,1.023537) +- (-0.001746,0.001746)(4500.000000,1.023084) +- (-0.001675,0.001675)(4550.000000,1.022762) +- (-0.001670,0.001670)(4600.000000,1.022505) +- (-0.001602,0.001602)(4650.000000,1.022382) +- (-0.001644,0.001644)(4700.000000,1.022243) +- (-0.001617,0.001617)(4750.000000,1.021878) +- (-0.001589,0.001589)(4800.000000,1.021676) +- (-0.001557,0.001557)(4850.000000,1.021696) +- (-0.001661,0.001661)(4900.000000,1.021505) +- (-0.001692,0.001692)(4950.000000,1.021387) +- (-0.001578,0.001578)(5000.000000,1.020933) +- (-0.001368,0.001368)
};
\addlegendentry{$r = 100$}


\addplot [
mark=none,
mark size=1.0pt,
color=col5,
line width=1pt,
opacity=0.7,
%error bars/.cd,
%error bar style={line width=0.2pt},
%y dir=both,
%y explicit
]
coordinates{
(100.000000,2.276287) +- (-0.209794,0.209794)(125.000000,1.826245) +- (-0.138646,0.138646)(150.000000,1.696247) +- (-0.151026,0.151026)(175.000000,1.600773) +- (-0.126480,0.126480)(200.000000,1.507355) +- (-0.068277,0.068277)(225.000000,1.419504) +- (-0.043888,0.043888)(250.000000,1.357344) +- (-0.037013,0.037013)(275.000000,1.326387) +- (-0.033733,0.033733)(300.000000,1.284577) +- (-0.029398,0.029398)(325.000000,1.259191) +- (-0.023773,0.023773)(350.000000,1.229818) +- (-0.016735,0.016735)(375.000000,1.212661) +- (-0.013385,0.013385)(400.000000,1.203774) +- (-0.012049,0.012049)(425.000000,1.195287) +- (-0.011997,0.011997)(450.000000,1.186596) +- (-0.013961,0.013961)(475.000000,1.176292) +- (-0.016129,0.016129)(500.000000,1.165802) +- (-0.014251,0.014251)(525.000000,1.159712) +- (-0.012483,0.012483)(550.000000,1.150587) +- (-0.011495,0.011495)(575.000000,1.144385) +- (-0.010455,0.010455)(600.000000,1.139392) +- (-0.009405,0.009405)(625.000000,1.135799) +- (-0.009310,0.009310)(650.000000,1.128920) +- (-0.009646,0.009646)(675.000000,1.128734) +- (-0.011430,0.011430)(700.000000,1.123805) +- (-0.010596,0.010596)(725.000000,1.118212) +- (-0.009113,0.009113)(750.000000,1.111856) +- (-0.008629,0.008629)(775.000000,1.108159) +- (-0.007413,0.007413)(800.000000,1.104559) +- (-0.006441,0.006441)(825.000000,1.100541) +- (-0.005088,0.005088)(850.000000,1.096610) +- (-0.004571,0.004571)(875.000000,1.094286) +- (-0.004043,0.004043)(900.000000,1.089737) +- (-0.003998,0.003998)(925.000000,1.085930) +- (-0.004152,0.004152)(950.000000,1.083545) +- (-0.003987,0.003987)(975.000000,1.080296) +- (-0.003744,0.003744)(1000.000000,1.078684) +- (-0.003890,0.003890)(1000.000000,1.078684) +- (-0.003890,0.003890)(1050.000000,1.073495) +- (-0.003676,0.003676)(1100.000000,1.070380) +- (-0.002971,0.002971)(1150.000000,1.067601) +- (-0.003814,0.003814)(1200.000000,1.065024) +- (-0.004058,0.004058)(1250.000000,1.062782) +- (-0.004025,0.004025)(1300.000000,1.059817) +- (-0.003707,0.003707)(1350.000000,1.057220) +- (-0.003487,0.003487)(1400.000000,1.056267) +- (-0.003477,0.003477)(1450.000000,1.054069) +- (-0.003332,0.003332)(1500.000000,1.051345) +- (-0.003287,0.003287)(1550.000000,1.050279) +- (-0.003380,0.003380)(1600.000000,1.047931) +- (-0.003313,0.003313)(1650.000000,1.046576) +- (-0.003354,0.003354)(1700.000000,1.046299) +- (-0.002985,0.002985)(1750.000000,1.044651) +- (-0.002535,0.002535)(1800.000000,1.043960) +- (-0.002257,0.002257)(1850.000000,1.042636) +- (-0.002284,0.002284)(1900.000000,1.041085) +- (-0.001919,0.001919)(1950.000000,1.039708) +- (-0.001948,0.001948)(2000.000000,1.039127) +- (-0.001992,0.001992)(2050.000000,1.038050) +- (-0.002084,0.002084)(2100.000000,1.036998) +- (-0.002181,0.002181)(2150.000000,1.036425) +- (-0.002169,0.002169)(2200.000000,1.035619) +- (-0.002093,0.002093)(2250.000000,1.034897) +- (-0.002186,0.002186)(2300.000000,1.034239) +- (-0.002309,0.002309)(2350.000000,1.033818) +- (-0.002481,0.002481)(2400.000000,1.032820) +- (-0.002372,0.002372)(2450.000000,1.032260) +- (-0.002361,0.002361)(2500.000000,1.031307) +- (-0.002144,0.002144)(2550.000000,1.031242) +- (-0.002011,0.002011)(2600.000000,1.030794) +- (-0.002042,0.002042)(2650.000000,1.030009) +- (-0.001900,0.001900)(2700.000000,1.028634) +- (-0.001681,0.001681)(2750.000000,1.027901) +- (-0.001478,0.001478)(2800.000000,1.027301) +- (-0.001602,0.001602)(2850.000000,1.026604) +- (-0.001506,0.001506)(2900.000000,1.026093) +- (-0.001396,0.001396)(2950.000000,1.025691) +- (-0.001385,0.001385)(3000.000000,1.025582) +- (-0.001385,0.001385)(3050.000000,1.025581) +- (-0.001459,0.001459)(3100.000000,1.025132) +- (-0.001539,0.001539)(3150.000000,1.024658) +- (-0.001456,0.001456)(3200.000000,1.024059) +- (-0.001414,0.001414)(3250.000000,1.023561) +- (-0.001381,0.001381)(3300.000000,1.022769) +- (-0.001374,0.001374)(3350.000000,1.022628) +- (-0.001277,0.001277)(3400.000000,1.022374) +- (-0.001115,0.001115)(3450.000000,1.022218) +- (-0.001052,0.001052)(3500.000000,1.022040) +- (-0.001020,0.001020)(3550.000000,1.021572) +- (-0.000934,0.000934)(3600.000000,1.021160) +- (-0.001003,0.001003)(3650.000000,1.021033) +- (-0.000917,0.000917)(3700.000000,1.020642) +- (-0.000879,0.000879)(3750.000000,1.020291) +- (-0.000894,0.000894)(3800.000000,1.020133) +- (-0.000850,0.000850)(3850.000000,1.019973) +- (-0.000867,0.000867)(3900.000000,1.019642) +- (-0.000807,0.000807)(3950.000000,1.019385) +- (-0.000850,0.000850)(4000.000000,1.019036) +- (-0.000865,0.000865)(4050.000000,1.018699) +- (-0.000800,0.000800)(4100.000000,1.018553) +- (-0.000749,0.000749)(4150.000000,1.018313) +- (-0.000758,0.000758)(4200.000000,1.018154) +- (-0.000685,0.000685)(4250.000000,1.017985) +- (-0.000724,0.000724)(4300.000000,1.018101) +- (-0.000789,0.000789)(4350.000000,1.018087) +- (-0.000876,0.000876)(4400.000000,1.018068) +- (-0.000835,0.000835)(4450.000000,1.018003) +- (-0.000847,0.000847)(4500.000000,1.017857) +- (-0.000824,0.000824)(4550.000000,1.017659) +- (-0.000786,0.000786)(4600.000000,1.017284) +- (-0.000728,0.000728)(4650.000000,1.017036) +- (-0.000675,0.000675)(4700.000000,1.016647) +- (-0.000622,0.000622)(4750.000000,1.016302) +- (-0.000595,0.000595)(4800.000000,1.016256) +- (-0.000566,0.000566)(4850.000000,1.016060) +- (-0.000572,0.000572)(4900.000000,1.015767) +- (-0.000619,0.000619)(4950.000000,1.015628) +- (-0.000618,0.000618)(5000.000000,1.015394) +- (-0.000582,0.000582)
};
\addlegendentry{$r = 200$}


\addplot [
mark=none,
mark size=1.0pt,
color=col6,
line width=1pt,
opacity=0.7,
%error bars/.cd,
%error bar style={line width=0.2pt},
%y dir=both,
%y explicit
]
coordinates{
(50.000000,3.149117) +- (-0.282843,0.282843)(75.000000,2.264024) +- (-0.223238,0.223238)(100.000000,1.814651) +- (-0.112540,0.112540)(125.000000,1.628818) +- (-0.104302,0.104302)(150.000000,1.525063) +- (-0.087945,0.087945)(175.000000,1.443626) +- (-0.071883,0.071883)(200.000000,1.403740) +- (-0.110289,0.110289)(225.000000,1.344200) +- (-0.081892,0.081892)(250.000000,1.294523) +- (-0.041978,0.041978)(275.000000,1.253955) +- (-0.027117,0.027117)(300.000000,1.217736) +- (-0.019564,0.019564)(325.000000,1.194529) +- (-0.013544,0.013544)(350.000000,1.172008) +- (-0.011660,0.011660)(375.000000,1.161622) +- (-0.011837,0.011837)(400.000000,1.148713) +- (-0.010381,0.010381)(425.000000,1.141888) +- (-0.010901,0.010901)(450.000000,1.130745) +- (-0.011373,0.011373)(475.000000,1.124956) +- (-0.011730,0.011730)(500.000000,1.117721) +- (-0.011047,0.011047)(525.000000,1.110970) +- (-0.008376,0.008376)(550.000000,1.105657) +- (-0.007674,0.007674)(575.000000,1.103838) +- (-0.006687,0.006687)(600.000000,1.098507) +- (-0.006226,0.006226)(625.000000,1.096179) +- (-0.005428,0.005428)(650.000000,1.094091) +- (-0.004618,0.004618)(675.000000,1.090155) +- (-0.004740,0.004740)(700.000000,1.087007) +- (-0.005031,0.005031)(725.000000,1.082733) +- (-0.004357,0.004357)(750.000000,1.080225) +- (-0.003591,0.003591)(775.000000,1.077969) +- (-0.003519,0.003519)(800.000000,1.075273) +- (-0.003465,0.003465)(825.000000,1.072597) +- (-0.003193,0.003193)(850.000000,1.069003) +- (-0.002960,0.002960)(875.000000,1.066573) +- (-0.003001,0.003001)(900.000000,1.065202) +- (-0.002730,0.002730)(925.000000,1.063144) +- (-0.002933,0.002933)(950.000000,1.061204) +- (-0.002879,0.002879)(975.000000,1.060193) +- (-0.003104,0.003104)(1000.000000,1.058535) +- (-0.003045,0.003045)(1000.000000,1.058535) +- (-0.003045,0.003045)(1050.000000,1.056745) +- (-0.002815,0.002815)(1100.000000,1.056124) +- (-0.002897,0.002897)(1150.000000,1.052847) +- (-0.002693,0.002693)(1200.000000,1.050010) +- (-0.002431,0.002431)(1250.000000,1.046940) +- (-0.002182,0.002182)(1300.000000,1.045004) +- (-0.002566,0.002566)(1350.000000,1.042911) +- (-0.002131,0.002131)(1400.000000,1.041073) +- (-0.001813,0.001813)(1450.000000,1.039370) +- (-0.001806,0.001806)(1500.000000,1.037620) +- (-0.001461,0.001461)(1550.000000,1.036601) +- (-0.001590,0.001590)(1600.000000,1.036241) +- (-0.001575,0.001575)(1650.000000,1.035731) +- (-0.001496,0.001496)(1700.000000,1.034672) +- (-0.001646,0.001646)(1750.000000,1.033663) +- (-0.001718,0.001718)(1800.000000,1.032975) +- (-0.001658,0.001658)(1850.000000,1.032252) +- (-0.001512,0.001512)(1900.000000,1.031410) +- (-0.001540,0.001540)(1950.000000,1.029980) +- (-0.001349,0.001349)(2000.000000,1.029429) +- (-0.001368,0.001368)(2050.000000,1.028663) +- (-0.001103,0.001103)(2100.000000,1.028452) +- (-0.001231,0.001231)(2150.000000,1.028018) +- (-0.001446,0.001446)(2200.000000,1.027307) +- (-0.001364,0.001364)(2250.000000,1.026653) +- (-0.001440,0.001440)(2300.000000,1.026057) +- (-0.001335,0.001335)(2350.000000,1.025733) +- (-0.001432,0.001432)(2400.000000,1.025418) +- (-0.001571,0.001571)(2450.000000,1.025073) +- (-0.001566,0.001566)(2500.000000,1.024825) +- (-0.001607,0.001607)(2550.000000,1.024163) +- (-0.001513,0.001513)(2600.000000,1.023332) +- (-0.001370,0.001370)(2650.000000,1.023118) +- (-0.001403,0.001403)(2700.000000,1.022815) +- (-0.001636,0.001636)(2750.000000,1.022240) +- (-0.001578,0.001578)(2800.000000,1.021946) +- (-0.001535,0.001535)(2850.000000,1.021326) +- (-0.001420,0.001420)(2900.000000,1.020810) +- (-0.001404,0.001404)(2950.000000,1.020476) +- (-0.001287,0.001287)(3000.000000,1.019922) +- (-0.001221,0.001221)(3050.000000,1.019520) +- (-0.001185,0.001185)(3100.000000,1.019125) +- (-0.001140,0.001140)(3150.000000,1.018550) +- (-0.001058,0.001058)(3200.000000,1.018372) +- (-0.001099,0.001099)(3250.000000,1.018339) +- (-0.001076,0.001076)(3300.000000,1.017884) +- (-0.001070,0.001070)(3350.000000,1.017300) +- (-0.001000,0.001000)(3400.000000,1.016873) +- (-0.000931,0.000931)(3450.000000,1.016615) +- (-0.000903,0.000903)(3500.000000,1.016109) +- (-0.000838,0.000838)(3550.000000,1.015799) +- (-0.000837,0.000837)(3600.000000,1.015767) +- (-0.000745,0.000745)(3650.000000,1.015776) +- (-0.000760,0.000760)(3700.000000,1.015505) +- (-0.000801,0.000801)(3750.000000,1.015301) +- (-0.000725,0.000725)(3800.000000,1.015020) +- (-0.000750,0.000750)(3850.000000,1.015123) +- (-0.000760,0.000760)(3900.000000,1.015017) +- (-0.000762,0.000762)(3950.000000,1.014803) +- (-0.000758,0.000758)(4000.000000,1.014458) +- (-0.000740,0.000740)(4050.000000,1.014234) +- (-0.000689,0.000689)(4100.000000,1.013958) +- (-0.000720,0.000720)(4150.000000,1.013937) +- (-0.000724,0.000724)(4200.000000,1.013684) +- (-0.000706,0.000706)(4250.000000,1.013432) +- (-0.000646,0.000646)(4300.000000,1.013269) +- (-0.000695,0.000695)(4350.000000,1.013045) +- (-0.000613,0.000613)(4400.000000,1.012839) +- (-0.000552,0.000552)(4450.000000,1.012701) +- (-0.000483,0.000483)(4500.000000,1.012546) +- (-0.000455,0.000455)(4550.000000,1.012339) +- (-0.000420,0.000420)(4600.000000,1.012249) +- (-0.000459,0.000459)(4650.000000,1.012046) +- (-0.000435,0.000435)(4700.000000,1.011907) +- (-0.000408,0.000408)(4750.000000,1.011739) +- (-0.000401,0.000401)(4800.000000,1.011521) +- (-0.000417,0.000417)(4850.000000,1.011439) +- (-0.000428,0.000428)(4900.000000,1.011446) +- (-0.000420,0.000420)(4950.000000,1.011224) +- (-0.000392,0.000392)(5000.000000,1.011154) +- (-0.000377,0.000377)
};
\addlegendentry{$r = 500$}

\end{axis}
\end{tikzpicture}

    \vspace{\scspacey}
    \caption{\hspace{\scspacex}\textsc{Water}}
    \label{fig:water2}
  \end{subfigure}
  \hspace{1em}
  \begin{subfigure}[b]{\subflen}
    \begin{tikzpicture}

\colorlet{col1}{blue!50!black}
\colorlet{col2}{red!10!darkgray}
\colorlet{col3}{red!30!darkgray}
\colorlet{col4}{red!50!darkgray}
\colorlet{col5}{red!70!darkgray}
\colorlet{col6}{red!90!darkgray}


\begin{axis}[%
tick label style={font=\tiny},
label style={font=\scriptsize},
legend style={font=\tiny},
view={0}{90},
width=\figurewidth,
height=\figureheight,
xmin=0, xmax=5000,
xtick={0, 1000, 2000, 3000, 4000, 5000},
xticklabels={0, 1k, 2k, 3k, 4k, 5k},
scaled x ticks=false,
xlabel={Samples},
xlabel shift=-0.3em,
ymin=1, ymax=1.52,
ytick={1, 1.5},
ylabel={PSRF},
ylabel shift=-1.5em,
tick label style={/pgf/number format/fixed},
major tick length=2pt,
axis lines*=left,
legend cell align=left,
clip marker paths=true,
legend style={at={(1.05,1.05)},draw=none,fill=none,row sep=-0.35em}]

\addplot [
mark=none,
mark size=1.0pt,
mark options={solid},
color=col1,
densely dashed,
line width=1pt,
opacity=0.7,
%error bars/.cd,
%error bar style={solid, line width=0.2pt},
%y dir=both,
%y explicit
]
coordinates{
(200.000000,3.597939) +- (-0.282843,0.282843)(225.000000,3.040631) +- (-0.197440,0.197440)(250.000000,2.691309) +- (-0.160233,0.160233)(275.000000,2.554452) +- (-0.165238,0.165238)(300.000000,2.413226) +- (-0.160241,0.160241)(325.000000,2.330296) +- (-0.148466,0.148466)(350.000000,2.215460) +- (-0.130796,0.130796)(375.000000,2.169512) +- (-0.130263,0.130263)(400.000000,2.082272) +- (-0.130343,0.130343)(425.000000,2.028475) +- (-0.144012,0.144012)(450.000000,2.014070) +- (-0.189635,0.189635)(475.000000,2.059465) +- (-0.282843,0.282843)(500.000000,1.948577) +- (-0.207447,0.207447)(525.000000,1.888214) +- (-0.187895,0.187895)(550.000000,1.800753) +- (-0.146235,0.146235)(575.000000,1.749965) +- (-0.125414,0.125414)(600.000000,1.717381) +- (-0.149604,0.149604)(625.000000,1.704961) +- (-0.179373,0.179373)(650.000000,1.685182) +- (-0.188536,0.188536)(675.000000,1.651440) +- (-0.194696,0.194696)(700.000000,1.565033) +- (-0.110935,0.110935)(725.000000,1.494971) +- (-0.060617,0.060617)(750.000000,1.448536) +- (-0.037580,0.037580)(775.000000,1.423904) +- (-0.030389,0.030389)(800.000000,1.404557) +- (-0.025724,0.025724)(825.000000,1.388615) +- (-0.024165,0.024165)(850.000000,1.375496) +- (-0.022723,0.022723)(875.000000,1.365812) +- (-0.022431,0.022431)(900.000000,1.359307) +- (-0.022667,0.022667)(925.000000,1.357107) +- (-0.023374,0.023374)(950.000000,1.353670) +- (-0.024974,0.024974)(975.000000,1.349603) +- (-0.026614,0.026614)(1000.000000,1.341914) +- (-0.026115,0.026115)(1000.000000,1.341914) +- (-0.026115,0.026115)(1050.000000,1.327010) +- (-0.025499,0.025499)(1100.000000,1.309038) +- (-0.022776,0.022776)(1150.000000,1.291903) +- (-0.021510,0.021510)(1200.000000,1.276404) +- (-0.019627,0.019627)(1250.000000,1.265232) +- (-0.019791,0.019791)(1300.000000,1.257915) +- (-0.022160,0.022160)(1350.000000,1.245602) +- (-0.022382,0.022382)(1400.000000,1.229363) +- (-0.019007,0.019007)(1450.000000,1.212160) +- (-0.014853,0.014853)(1500.000000,1.202877) +- (-0.014921,0.014921)(1550.000000,1.194305) +- (-0.016019,0.016019)(1600.000000,1.187061) +- (-0.016566,0.016566)(1650.000000,1.181264) +- (-0.015779,0.015779)(1700.000000,1.176979) +- (-0.015684,0.015684)(1750.000000,1.173301) +- (-0.016778,0.016778)(1800.000000,1.166561) +- (-0.015625,0.015625)(1850.000000,1.160849) +- (-0.014327,0.014327)(1900.000000,1.157492) +- (-0.013240,0.013240)(1950.000000,1.152301) +- (-0.012648,0.012648)(2000.000000,1.146251) +- (-0.010389,0.010389)(2050.000000,1.142146) +- (-0.008755,0.008755)(2100.000000,1.138389) +- (-0.008279,0.008279)(2150.000000,1.134830) +- (-0.007781,0.007781)(2200.000000,1.131334) +- (-0.007385,0.007385)(2250.000000,1.129364) +- (-0.007120,0.007120)(2300.000000,1.127263) +- (-0.007972,0.007972)(2350.000000,1.124395) +- (-0.008520,0.008520)(2400.000000,1.119219) +- (-0.008856,0.008856)(2450.000000,1.117366) +- (-0.010033,0.010033)(2500.000000,1.114105) +- (-0.008819,0.008819)(2550.000000,1.113725) +- (-0.008468,0.008468)(2600.000000,1.111648) +- (-0.008166,0.008166)(2650.000000,1.109619) +- (-0.007836,0.007836)(2700.000000,1.107001) +- (-0.007309,0.007309)(2750.000000,1.104818) +- (-0.006968,0.006968)(2800.000000,1.104075) +- (-0.006324,0.006324)(2850.000000,1.102976) +- (-0.006398,0.006398)(2900.000000,1.100722) +- (-0.006288,0.006288)(2950.000000,1.097873) +- (-0.005578,0.005578)(3000.000000,1.095111) +- (-0.005033,0.005033)(3050.000000,1.093478) +- (-0.005158,0.005158)(3100.000000,1.091610) +- (-0.005177,0.005177)(3150.000000,1.090405) +- (-0.005001,0.005001)(3200.000000,1.089527) +- (-0.004982,0.004982)(3250.000000,1.088855) +- (-0.004821,0.004821)(3300.000000,1.087810) +- (-0.004819,0.004819)(3350.000000,1.086178) +- (-0.004890,0.004890)(3400.000000,1.083872) +- (-0.004823,0.004823)(3450.000000,1.082651) +- (-0.004757,0.004757)(3500.000000,1.082000) +- (-0.004411,0.004411)(3550.000000,1.081654) +- (-0.004404,0.004404)(3600.000000,1.080664) +- (-0.004261,0.004261)(3650.000000,1.079681) +- (-0.004000,0.004000)(3700.000000,1.078145) +- (-0.004157,0.004157)(3750.000000,1.077135) +- (-0.004285,0.004285)(3800.000000,1.076356) +- (-0.004383,0.004383)(3850.000000,1.075704) +- (-0.004286,0.004286)(3900.000000,1.074393) +- (-0.004038,0.004038)(3950.000000,1.072686) +- (-0.003808,0.003808)(4000.000000,1.071898) +- (-0.003671,0.003671)(4050.000000,1.070577) +- (-0.003569,0.003569)(4100.000000,1.068997) +- (-0.003373,0.003373)(4150.000000,1.067786) +- (-0.003359,0.003359)(4200.000000,1.067305) +- (-0.003437,0.003437)(4250.000000,1.066197) +- (-0.003413,0.003413)(4300.000000,1.064406) +- (-0.003289,0.003289)(4350.000000,1.062878) +- (-0.003119,0.003119)(4400.000000,1.061796) +- (-0.003118,0.003118)(4450.000000,1.061438) +- (-0.003328,0.003328)(4500.000000,1.061019) +- (-0.003352,0.003352)(4550.000000,1.060327) +- (-0.003275,0.003275)(4600.000000,1.059904) +- (-0.003248,0.003248)(4650.000000,1.059435) +- (-0.003126,0.003126)(4700.000000,1.058691) +- (-0.002941,0.002941)(4750.000000,1.057888) +- (-0.002763,0.002763)(4800.000000,1.057200) +- (-0.002759,0.002759)(4850.000000,1.056826) +- (-0.002693,0.002693)(4900.000000,1.056174) +- (-0.002721,0.002721)(4950.000000,1.055308) +- (-0.002654,0.002654)(5000.000000,1.054398) +- (-0.002602,0.002602)
};
\addlegendentry{δ = 1}


\addplot [
mark=none,
mark size=1.0pt,
color=col2,
line width=1pt,
opacity=0.7,
%error bars/.cd,
%error bar style={line width=0.2pt},
%y dir=both,
%y explicit
]
coordinates{
(100.000000,2.621969) +- (-0.216457,0.216457)(125.000000,2.297357) +- (-0.187623,0.187623)(150.000000,1.929898) +- (-0.098471,0.098471)(175.000000,1.919706) +- (-0.157751,0.157751)(200.000000,1.791757) +- (-0.188065,0.188065)(225.000000,1.639682) +- (-0.080145,0.080145)(250.000000,1.542704) +- (-0.063290,0.063290)(275.000000,1.471847) +- (-0.036402,0.036402)(300.000000,1.412291) +- (-0.025134,0.025134)(325.000000,1.378787) +- (-0.020530,0.020530)(350.000000,1.351956) +- (-0.019694,0.019694)(375.000000,1.334862) +- (-0.020965,0.020965)(400.000000,1.300941) +- (-0.018053,0.018053)(425.000000,1.283399) +- (-0.014974,0.014974)(450.000000,1.263625) +- (-0.013533,0.013533)(475.000000,1.256375) +- (-0.013396,0.013396)(500.000000,1.244612) +- (-0.013826,0.013826)(525.000000,1.227621) +- (-0.015686,0.015686)(550.000000,1.213323) +- (-0.014516,0.014516)(575.000000,1.205058) +- (-0.015556,0.015556)(600.000000,1.196273) +- (-0.014260,0.014260)(625.000000,1.185715) +- (-0.013073,0.013073)(650.000000,1.176979) +- (-0.011684,0.011684)(675.000000,1.168043) +- (-0.010403,0.010403)(700.000000,1.161667) +- (-0.008939,0.008939)(725.000000,1.156242) +- (-0.009373,0.009373)(750.000000,1.150361) +- (-0.009637,0.009637)(775.000000,1.146105) +- (-0.010002,0.010002)(800.000000,1.140929) +- (-0.009564,0.009564)(825.000000,1.137991) +- (-0.009556,0.009556)(850.000000,1.134710) +- (-0.009999,0.009999)(875.000000,1.131515) +- (-0.009836,0.009836)(900.000000,1.127509) +- (-0.008323,0.008323)(925.000000,1.122749) +- (-0.007437,0.007437)(950.000000,1.120846) +- (-0.007149,0.007149)(975.000000,1.119496) +- (-0.007644,0.007644)(1000.000000,1.114321) +- (-0.006490,0.006490)(1000.000000,1.114321) +- (-0.006490,0.006490)(1050.000000,1.108567) +- (-0.005243,0.005243)(1100.000000,1.101689) +- (-0.004625,0.004625)(1150.000000,1.096266) +- (-0.005206,0.005206)(1200.000000,1.090501) +- (-0.005004,0.005004)(1250.000000,1.086002) +- (-0.004544,0.004544)(1300.000000,1.082727) +- (-0.004615,0.004615)(1350.000000,1.080648) +- (-0.005537,0.005537)(1400.000000,1.077570) +- (-0.006625,0.006625)(1450.000000,1.075386) +- (-0.006185,0.006185)(1500.000000,1.072016) +- (-0.005119,0.005119)(1550.000000,1.069326) +- (-0.004385,0.004385)(1600.000000,1.067256) +- (-0.004323,0.004323)(1650.000000,1.066128) +- (-0.004004,0.004004)(1700.000000,1.063679) +- (-0.003696,0.003696)(1750.000000,1.061838) +- (-0.003365,0.003365)(1800.000000,1.059677) +- (-0.003507,0.003507)(1850.000000,1.058514) +- (-0.003671,0.003671)(1900.000000,1.057154) +- (-0.003539,0.003539)(1950.000000,1.054866) +- (-0.003601,0.003601)(2000.000000,1.054002) +- (-0.003427,0.003427)(2050.000000,1.053683) +- (-0.003401,0.003401)(2100.000000,1.052993) +- (-0.003242,0.003242)(2150.000000,1.051927) +- (-0.003400,0.003400)(2200.000000,1.050262) +- (-0.003195,0.003195)(2250.000000,1.049267) +- (-0.003055,0.003055)(2300.000000,1.047896) +- (-0.002897,0.002897)(2350.000000,1.046632) +- (-0.002816,0.002816)(2400.000000,1.046466) +- (-0.002731,0.002731)(2450.000000,1.045147) +- (-0.002677,0.002677)(2500.000000,1.043673) +- (-0.002657,0.002657)(2550.000000,1.042224) +- (-0.002366,0.002366)(2600.000000,1.040801) +- (-0.002092,0.002092)(2650.000000,1.040012) +- (-0.001983,0.001983)(2700.000000,1.039262) +- (-0.001970,0.001970)(2750.000000,1.038966) +- (-0.001944,0.001944)(2800.000000,1.038080) +- (-0.001895,0.001895)(2850.000000,1.037721) +- (-0.001941,0.001941)(2900.000000,1.037586) +- (-0.001929,0.001929)(2950.000000,1.037313) +- (-0.001818,0.001818)(3000.000000,1.036527) +- (-0.001784,0.001784)(3050.000000,1.036051) +- (-0.001963,0.001963)(3100.000000,1.035793) +- (-0.002004,0.002004)(3150.000000,1.034877) +- (-0.001996,0.001996)(3200.000000,1.034190) +- (-0.001891,0.001891)(3250.000000,1.033680) +- (-0.001864,0.001864)(3300.000000,1.033705) +- (-0.001936,0.001936)(3350.000000,1.033530) +- (-0.002136,0.002136)(3400.000000,1.033732) +- (-0.002412,0.002412)(3450.000000,1.033454) +- (-0.002576,0.002576)(3500.000000,1.033016) +- (-0.002573,0.002573)(3550.000000,1.032536) +- (-0.002435,0.002435)(3600.000000,1.032135) +- (-0.002251,0.002251)(3650.000000,1.031804) +- (-0.002049,0.002049)(3700.000000,1.031606) +- (-0.002112,0.002112)(3750.000000,1.031939) +- (-0.001990,0.001990)(3800.000000,1.031062) +- (-0.001904,0.001904)(3850.000000,1.030663) +- (-0.001872,0.001872)(3900.000000,1.030234) +- (-0.001819,0.001819)(3950.000000,1.029873) +- (-0.001736,0.001736)(4000.000000,1.029544) +- (-0.001581,0.001581)(4050.000000,1.028936) +- (-0.001475,0.001475)(4100.000000,1.028275) +- (-0.001376,0.001376)(4150.000000,1.027776) +- (-0.001249,0.001249)(4200.000000,1.027205) +- (-0.001233,0.001233)(4250.000000,1.026404) +- (-0.001200,0.001200)(4300.000000,1.025712) +- (-0.001146,0.001146)(4350.000000,1.025224) +- (-0.001131,0.001131)(4400.000000,1.024927) +- (-0.001201,0.001201)(4450.000000,1.024702) +- (-0.001187,0.001187)(4500.000000,1.024452) +- (-0.001149,0.001149)(4550.000000,1.024149) +- (-0.001173,0.001173)(4600.000000,1.023677) +- (-0.001226,0.001226)(4650.000000,1.023261) +- (-0.001218,0.001218)(4700.000000,1.022834) +- (-0.001255,0.001255)(4750.000000,1.022696) +- (-0.001263,0.001263)(4800.000000,1.022321) +- (-0.001185,0.001185)(4850.000000,1.022249) +- (-0.001149,0.001149)(4900.000000,1.021876) +- (-0.001081,0.001081)(4950.000000,1.021828) +- (-0.001035,0.001035)(5000.000000,1.021425) +- (-0.001014,0.001014)
};
\addlegendentry{δ = 0.8}


\addplot [
mark=none,
mark size=1.0pt,
color=col3,
line width=1pt,
opacity=0.7,
%error bars/.cd,
%error bar style={line width=0.2pt},
%y dir=both,
%y explicit
]
coordinates{
(50.000000,4.299835) +- (-0.282843,0.282843)(75.000000,2.681273) +- (-0.282843,0.282843)(100.000000,2.635776) +- (-0.282843,0.282843)(125.000000,2.018942) +- (-0.173918,0.173918)(150.000000,1.692553) +- (-0.121874,0.121874)(175.000000,1.522687) +- (-0.083932,0.083932)(200.000000,1.488431) +- (-0.115950,0.115950)(225.000000,1.414498) +- (-0.095975,0.095975)(250.000000,1.347309) +- (-0.050171,0.050171)(275.000000,1.313640) +- (-0.035286,0.035286)(300.000000,1.282907) +- (-0.026953,0.026953)(325.000000,1.262391) +- (-0.023335,0.023335)(350.000000,1.240199) +- (-0.022312,0.022312)(375.000000,1.227289) +- (-0.022931,0.022931)(400.000000,1.212939) +- (-0.022395,0.022395)(425.000000,1.200887) +- (-0.018990,0.018990)(450.000000,1.187814) +- (-0.016422,0.016422)(475.000000,1.180031) +- (-0.016388,0.016388)(500.000000,1.162770) +- (-0.014085,0.014085)(525.000000,1.152871) +- (-0.011136,0.011136)(550.000000,1.145848) +- (-0.010945,0.010945)(575.000000,1.139114) +- (-0.010405,0.010405)(600.000000,1.130235) +- (-0.009391,0.009391)(625.000000,1.124868) +- (-0.008597,0.008597)(650.000000,1.120613) +- (-0.007955,0.007955)(675.000000,1.117473) +- (-0.007037,0.007037)(700.000000,1.112841) +- (-0.006350,0.006350)(725.000000,1.108100) +- (-0.005264,0.005264)(750.000000,1.104251) +- (-0.004672,0.004672)(775.000000,1.100145) +- (-0.004595,0.004595)(800.000000,1.094880) +- (-0.004764,0.004764)(825.000000,1.091624) +- (-0.004627,0.004627)(850.000000,1.088819) +- (-0.004204,0.004204)(875.000000,1.088142) +- (-0.004598,0.004598)(900.000000,1.085910) +- (-0.004483,0.004483)(925.000000,1.084108) +- (-0.004492,0.004492)(950.000000,1.081979) +- (-0.004483,0.004483)(975.000000,1.080881) +- (-0.004809,0.004809)(1000.000000,1.078696) +- (-0.005229,0.005229)(1000.000000,1.078696) +- (-0.005229,0.005229)(1050.000000,1.073086) +- (-0.004717,0.004717)(1100.000000,1.071274) +- (-0.004671,0.004671)(1150.000000,1.070317) +- (-0.004566,0.004566)(1200.000000,1.069010) +- (-0.004136,0.004136)(1250.000000,1.068118) +- (-0.004324,0.004324)(1300.000000,1.065506) +- (-0.005231,0.005231)(1350.000000,1.063234) +- (-0.004714,0.004714)(1400.000000,1.060268) +- (-0.004130,0.004130)(1450.000000,1.057731) +- (-0.003772,0.003772)(1500.000000,1.055776) +- (-0.003841,0.003841)(1550.000000,1.053775) +- (-0.003655,0.003655)(1600.000000,1.050994) +- (-0.003307,0.003307)(1650.000000,1.049685) +- (-0.003387,0.003387)(1700.000000,1.047174) +- (-0.003192,0.003192)(1750.000000,1.045074) +- (-0.002733,0.002733)(1800.000000,1.043975) +- (-0.002588,0.002588)(1850.000000,1.041654) +- (-0.002417,0.002417)(1900.000000,1.039879) +- (-0.002223,0.002223)(1950.000000,1.039625) +- (-0.002738,0.002738)(2000.000000,1.038862) +- (-0.002826,0.002826)(2050.000000,1.037900) +- (-0.002455,0.002455)(2100.000000,1.036545) +- (-0.002044,0.002044)(2150.000000,1.035802) +- (-0.001858,0.001858)(2200.000000,1.035160) +- (-0.001527,0.001527)(2250.000000,1.034495) +- (-0.001489,0.001489)(2300.000000,1.033563) +- (-0.001345,0.001345)(2350.000000,1.032475) +- (-0.001349,0.001349)(2400.000000,1.031914) +- (-0.001405,0.001405)(2450.000000,1.031209) +- (-0.001382,0.001382)(2500.000000,1.030318) +- (-0.001206,0.001206)(2550.000000,1.029588) +- (-0.001136,0.001136)(2600.000000,1.028842) +- (-0.001123,0.001123)(2650.000000,1.028873) +- (-0.001145,0.001145)(2700.000000,1.028434) +- (-0.001202,0.001202)(2750.000000,1.027512) +- (-0.001283,0.001283)(2800.000000,1.027132) +- (-0.001189,0.001189)(2850.000000,1.026781) +- (-0.001186,0.001186)(2900.000000,1.026437) +- (-0.001233,0.001233)(2950.000000,1.025994) +- (-0.001254,0.001254)(3000.000000,1.025812) +- (-0.001258,0.001258)(3050.000000,1.025575) +- (-0.001308,0.001308)(3100.000000,1.025161) +- (-0.001276,0.001276)(3150.000000,1.024753) +- (-0.001199,0.001199)(3200.000000,1.024774) +- (-0.001247,0.001247)(3250.000000,1.024500) +- (-0.001241,0.001241)(3300.000000,1.024430) +- (-0.001221,0.001221)(3350.000000,1.023553) +- (-0.001155,0.001155)(3400.000000,1.022764) +- (-0.001053,0.001053)(3450.000000,1.022105) +- (-0.000941,0.000941)(3500.000000,1.021695) +- (-0.000915,0.000915)(3550.000000,1.021370) +- (-0.000862,0.000862)(3600.000000,1.021266) +- (-0.000856,0.000856)(3650.000000,1.021004) +- (-0.000825,0.000825)(3700.000000,1.020771) +- (-0.000872,0.000872)(3750.000000,1.020540) +- (-0.000899,0.000899)(3800.000000,1.020104) +- (-0.000977,0.000977)(3850.000000,1.019918) +- (-0.000961,0.000961)(3900.000000,1.019843) +- (-0.000922,0.000922)(3950.000000,1.019711) +- (-0.001001,0.001001)(4000.000000,1.019750) +- (-0.001046,0.001046)(4050.000000,1.019534) +- (-0.000950,0.000950)(4100.000000,1.019262) +- (-0.000864,0.000864)(4150.000000,1.019020) +- (-0.000892,0.000892)(4200.000000,1.018802) +- (-0.000969,0.000969)(4250.000000,1.018689) +- (-0.000959,0.000959)(4300.000000,1.018585) +- (-0.000965,0.000965)(4350.000000,1.018254) +- (-0.000980,0.000980)(4400.000000,1.017872) +- (-0.000936,0.000936)(4450.000000,1.017773) +- (-0.000970,0.000970)(4500.000000,1.017616) +- (-0.000930,0.000930)(4550.000000,1.017394) +- (-0.000896,0.000896)(4600.000000,1.017298) +- (-0.000886,0.000886)(4650.000000,1.017160) +- (-0.000879,0.000879)(4700.000000,1.016898) +- (-0.000870,0.000870)(4750.000000,1.016601) +- (-0.000839,0.000839)(4800.000000,1.016365) +- (-0.000825,0.000825)(4850.000000,1.016291) +- (-0.000811,0.000811)(4900.000000,1.016217) +- (-0.000815,0.000815)(4950.000000,1.015874) +- (-0.000829,0.000829)(5000.000000,1.015701) +- (-0.000809,0.000809)
};
\addlegendentry{δ = 0.6}


\addplot [
mark=none,
mark size=1.0pt,
color=col4,
line width=1pt,
opacity=0.7,
%error bars/.cd,
%error bar style={line width=0.2pt},
%y dir=both,
%y explicit
]
coordinates{
(50.000000,3.182692) +- (-0.270973,0.270973)(75.000000,2.545131) +- (-0.282843,0.282843)(100.000000,1.990985) +- (-0.135731,0.135731)(125.000000,1.739998) +- (-0.112859,0.112859)(150.000000,1.634286) +- (-0.103875,0.103875)(175.000000,1.594593) +- (-0.140943,0.140943)(200.000000,1.510778) +- (-0.102251,0.102251)(225.000000,1.450242) +- (-0.095100,0.095100)(250.000000,1.368776) +- (-0.056447,0.056447)(275.000000,1.325561) +- (-0.043039,0.043039)(300.000000,1.296844) +- (-0.040780,0.040780)(325.000000,1.258283) +- (-0.031964,0.031964)(350.000000,1.236421) +- (-0.024303,0.024303)(375.000000,1.212642) +- (-0.017954,0.017954)(400.000000,1.199838) +- (-0.017898,0.017898)(425.000000,1.187104) +- (-0.018005,0.018005)(450.000000,1.181880) +- (-0.017540,0.017540)(475.000000,1.173890) +- (-0.016082,0.016082)(500.000000,1.167214) +- (-0.015976,0.015976)(525.000000,1.163412) +- (-0.017556,0.017556)(550.000000,1.157312) +- (-0.017998,0.017998)(575.000000,1.146878) +- (-0.017128,0.017128)(600.000000,1.135877) +- (-0.015245,0.015245)(625.000000,1.131070) +- (-0.014522,0.014522)(650.000000,1.123094) +- (-0.013958,0.013958)(675.000000,1.119041) +- (-0.012561,0.012561)(700.000000,1.114495) +- (-0.011657,0.011657)(725.000000,1.110554) +- (-0.011331,0.011331)(750.000000,1.106802) +- (-0.010370,0.010370)(775.000000,1.103615) +- (-0.008362,0.008362)(800.000000,1.099423) +- (-0.007600,0.007600)(825.000000,1.096946) +- (-0.006986,0.006986)(850.000000,1.094692) +- (-0.006685,0.006685)(875.000000,1.092315) +- (-0.006877,0.006877)(900.000000,1.088462) +- (-0.007061,0.007061)(925.000000,1.086207) +- (-0.006670,0.006670)(950.000000,1.083689) +- (-0.005633,0.005633)(975.000000,1.082683) +- (-0.005649,0.005649)(1000.000000,1.080939) +- (-0.005997,0.005997)(1000.000000,1.080939) +- (-0.005997,0.005997)(1050.000000,1.078202) +- (-0.006100,0.006100)(1100.000000,1.075488) +- (-0.006677,0.006677)(1150.000000,1.073161) +- (-0.007626,0.007626)(1200.000000,1.069625) +- (-0.007149,0.007149)(1250.000000,1.066020) +- (-0.006414,0.006414)(1300.000000,1.062852) +- (-0.005624,0.005624)(1350.000000,1.059979) +- (-0.005288,0.005288)(1400.000000,1.059189) +- (-0.005727,0.005727)(1450.000000,1.057617) +- (-0.005974,0.005974)(1500.000000,1.055258) +- (-0.005447,0.005447)(1550.000000,1.053584) +- (-0.004839,0.004839)(1600.000000,1.050953) +- (-0.004177,0.004177)(1650.000000,1.048720) +- (-0.003805,0.003805)(1700.000000,1.045894) +- (-0.003301,0.003301)(1750.000000,1.043717) +- (-0.003084,0.003084)(1800.000000,1.042326) +- (-0.002487,0.002487)(1850.000000,1.040742) +- (-0.002349,0.002349)(1900.000000,1.039460) +- (-0.002301,0.002301)(1950.000000,1.038248) +- (-0.002284,0.002284)(2000.000000,1.037092) +- (-0.002150,0.002150)(2050.000000,1.036336) +- (-0.002057,0.002057)(2100.000000,1.035524) +- (-0.002079,0.002079)(2150.000000,1.035670) +- (-0.002098,0.002098)(2200.000000,1.035534) +- (-0.002074,0.002074)(2250.000000,1.034717) +- (-0.002207,0.002207)(2300.000000,1.033963) +- (-0.002148,0.002148)(2350.000000,1.033222) +- (-0.002253,0.002253)(2400.000000,1.032397) +- (-0.002047,0.002047)(2450.000000,1.031940) +- (-0.002069,0.002069)(2500.000000,1.030997) +- (-0.001881,0.001881)(2550.000000,1.030278) +- (-0.001825,0.001825)(2600.000000,1.030035) +- (-0.001685,0.001685)(2650.000000,1.029670) +- (-0.001599,0.001599)(2700.000000,1.029531) +- (-0.001635,0.001635)(2750.000000,1.028829) +- (-0.001498,0.001498)(2800.000000,1.027859) +- (-0.001339,0.001339)(2850.000000,1.027661) +- (-0.001319,0.001319)(2900.000000,1.027326) +- (-0.001421,0.001421)(2950.000000,1.026907) +- (-0.001490,0.001490)(3000.000000,1.026576) +- (-0.001544,0.001544)(3050.000000,1.026006) +- (-0.001489,0.001489)(3100.000000,1.025998) +- (-0.001386,0.001386)(3150.000000,1.025275) +- (-0.001376,0.001376)(3200.000000,1.024364) +- (-0.001411,0.001411)(3250.000000,1.023942) +- (-0.001437,0.001437)(3300.000000,1.023563) +- (-0.001370,0.001370)(3350.000000,1.023373) +- (-0.001533,0.001533)(3400.000000,1.023330) +- (-0.001550,0.001550)(3450.000000,1.023329) +- (-0.001493,0.001493)(3500.000000,1.023122) +- (-0.001357,0.001357)(3550.000000,1.022968) +- (-0.001204,0.001204)(3600.000000,1.022859) +- (-0.001103,0.001103)(3650.000000,1.022684) +- (-0.001120,0.001120)(3700.000000,1.022142) +- (-0.001135,0.001135)(3750.000000,1.021683) +- (-0.001133,0.001133)(3800.000000,1.021468) +- (-0.001070,0.001070)(3850.000000,1.021101) +- (-0.000983,0.000983)(3900.000000,1.020713) +- (-0.000921,0.000921)(3950.000000,1.020317) +- (-0.000940,0.000940)(4000.000000,1.020434) +- (-0.001001,0.001001)(4050.000000,1.020249) +- (-0.001047,0.001047)(4100.000000,1.020096) +- (-0.001068,0.001068)(4150.000000,1.019717) +- (-0.001080,0.001080)(4200.000000,1.019494) +- (-0.001079,0.001079)(4250.000000,1.019283) +- (-0.001120,0.001120)(4300.000000,1.019052) +- (-0.001045,0.001045)(4350.000000,1.018700) +- (-0.000992,0.000992)(4400.000000,1.018619) +- (-0.001002,0.001002)(4450.000000,1.018426) +- (-0.001044,0.001044)(4500.000000,1.018208) +- (-0.000971,0.000971)(4550.000000,1.018072) +- (-0.000853,0.000853)(4600.000000,1.017769) +- (-0.000836,0.000836)(4650.000000,1.017474) +- (-0.000815,0.000815)(4700.000000,1.017215) +- (-0.000821,0.000821)(4750.000000,1.016908) +- (-0.000863,0.000863)(4800.000000,1.016798) +- (-0.000932,0.000932)(4850.000000,1.016764) +- (-0.000930,0.000930)(4900.000000,1.016723) +- (-0.000940,0.000940)(4950.000000,1.016430) +- (-0.000921,0.000921)(5000.000000,1.016203) +- (-0.000963,0.000963)
};
\addlegendentry{δ = 0.4}


\addplot [
mark=none,
mark size=1.0pt,
color=col5,
line width=1pt,
opacity=0.7,
%error bars/.cd,
%error bar style={line width=0.2pt},
%y dir=both,
%y explicit
]
coordinates{
(50.000000,3.319101) +- (-0.282843,0.282843)(75.000000,2.373216) +- (-0.222011,0.222011)(100.000000,2.082572) +- (-0.191951,0.191951)(125.000000,2.020236) +- (-0.238520,0.238520)(150.000000,1.819040) +- (-0.163331,0.163331)(175.000000,1.696413) +- (-0.157798,0.157798)(200.000000,1.625150) +- (-0.148839,0.148839)(225.000000,1.533201) +- (-0.118934,0.118934)(250.000000,1.427692) +- (-0.076921,0.076921)(275.000000,1.363607) +- (-0.050148,0.050148)(300.000000,1.306499) +- (-0.040954,0.040954)(325.000000,1.268136) +- (-0.037086,0.037086)(350.000000,1.236974) +- (-0.032412,0.032412)(375.000000,1.220337) +- (-0.029620,0.029620)(400.000000,1.208380) +- (-0.035484,0.035484)(425.000000,1.200712) +- (-0.041524,0.041524)(450.000000,1.187821) +- (-0.034591,0.034591)(475.000000,1.176409) +- (-0.028089,0.028089)(500.000000,1.172886) +- (-0.026474,0.026474)(525.000000,1.169293) +- (-0.027062,0.027062)(550.000000,1.161123) +- (-0.024722,0.024722)(575.000000,1.154346) +- (-0.022149,0.022149)(600.000000,1.145458) +- (-0.020953,0.020953)(625.000000,1.139606) +- (-0.019719,0.019719)(650.000000,1.135158) +- (-0.018738,0.018738)(675.000000,1.129923) +- (-0.018080,0.018080)(700.000000,1.126268) +- (-0.018734,0.018734)(725.000000,1.123946) +- (-0.019868,0.019868)(750.000000,1.120207) +- (-0.018433,0.018433)(775.000000,1.115891) +- (-0.016340,0.016340)(800.000000,1.112616) +- (-0.015407,0.015407)(825.000000,1.107983) +- (-0.014404,0.014404)(850.000000,1.104812) +- (-0.014306,0.014306)(875.000000,1.100951) +- (-0.013193,0.013193)(900.000000,1.098397) +- (-0.012532,0.012532)(925.000000,1.095490) +- (-0.011961,0.011961)(950.000000,1.092618) +- (-0.011246,0.011246)(975.000000,1.089061) +- (-0.010452,0.010452)(1000.000000,1.085012) +- (-0.009152,0.009152)(1000.000000,1.085012) +- (-0.009152,0.009152)(1050.000000,1.078034) +- (-0.007184,0.007184)(1100.000000,1.073728) +- (-0.005861,0.005861)(1150.000000,1.068672) +- (-0.004835,0.004835)(1200.000000,1.065246) +- (-0.004803,0.004803)(1250.000000,1.061976) +- (-0.004196,0.004196)(1300.000000,1.059891) +- (-0.003816,0.003816)(1350.000000,1.058662) +- (-0.004052,0.004052)(1400.000000,1.056371) +- (-0.004234,0.004234)(1450.000000,1.054390) +- (-0.004126,0.004126)(1500.000000,1.052761) +- (-0.004238,0.004238)(1550.000000,1.052258) +- (-0.004592,0.004592)(1600.000000,1.051504) +- (-0.004489,0.004489)(1650.000000,1.050322) +- (-0.004643,0.004643)(1700.000000,1.048618) +- (-0.004459,0.004459)(1750.000000,1.046722) +- (-0.004334,0.004334)(1800.000000,1.045259) +- (-0.004060,0.004060)(1850.000000,1.045178) +- (-0.003960,0.003960)(1900.000000,1.045016) +- (-0.004000,0.004000)(1950.000000,1.043881) +- (-0.003929,0.003929)(2000.000000,1.042104) +- (-0.003630,0.003630)(2050.000000,1.041164) +- (-0.003645,0.003645)(2100.000000,1.040657) +- (-0.003547,0.003547)(2150.000000,1.039775) +- (-0.003607,0.003607)(2200.000000,1.039113) +- (-0.003782,0.003782)(2250.000000,1.038149) +- (-0.003763,0.003763)(2300.000000,1.036867) +- (-0.003356,0.003356)(2350.000000,1.035819) +- (-0.003135,0.003135)(2400.000000,1.035232) +- (-0.003145,0.003145)(2450.000000,1.034080) +- (-0.003195,0.003195)(2500.000000,1.033341) +- (-0.003454,0.003454)(2550.000000,1.032511) +- (-0.003478,0.003478)(2600.000000,1.031724) +- (-0.003395,0.003395)(2650.000000,1.031396) +- (-0.003358,0.003358)(2700.000000,1.030858) +- (-0.003364,0.003364)(2750.000000,1.030406) +- (-0.003307,0.003307)(2800.000000,1.029696) +- (-0.003224,0.003224)(2850.000000,1.029181) +- (-0.003209,0.003209)(2900.000000,1.028955) +- (-0.003111,0.003111)(2950.000000,1.028485) +- (-0.003006,0.003006)(3000.000000,1.028007) +- (-0.002831,0.002831)(3050.000000,1.027232) +- (-0.002760,0.002760)(3100.000000,1.026545) +- (-0.002718,0.002718)(3150.000000,1.025958) +- (-0.002699,0.002699)(3200.000000,1.025229) +- (-0.002471,0.002471)(3250.000000,1.024695) +- (-0.002210,0.002210)(3300.000000,1.024124) +- (-0.002084,0.002084)(3350.000000,1.023834) +- (-0.002013,0.002013)(3400.000000,1.023042) +- (-0.001916,0.001916)(3450.000000,1.022490) +- (-0.001900,0.001900)(3500.000000,1.022394) +- (-0.001919,0.001919)(3550.000000,1.022138) +- (-0.001791,0.001791)(3600.000000,1.021947) +- (-0.001722,0.001722)(3650.000000,1.021424) +- (-0.001708,0.001708)(3700.000000,1.021240) +- (-0.001619,0.001619)(3750.000000,1.021190) +- (-0.001705,0.001705)(3800.000000,1.020724) +- (-0.001684,0.001684)(3850.000000,1.020637) +- (-0.001728,0.001728)(3900.000000,1.020592) +- (-0.001908,0.001908)(3950.000000,1.020642) +- (-0.002015,0.002015)(4000.000000,1.020499) +- (-0.001991,0.001991)(4050.000000,1.020480) +- (-0.001885,0.001885)(4100.000000,1.020221) +- (-0.001845,0.001845)(4150.000000,1.019982) +- (-0.001844,0.001844)(4200.000000,1.019590) +- (-0.001725,0.001725)(4250.000000,1.019488) +- (-0.001698,0.001698)(4300.000000,1.019543) +- (-0.001737,0.001737)(4350.000000,1.019590) +- (-0.001817,0.001817)(4400.000000,1.019524) +- (-0.001863,0.001863)(4450.000000,1.019331) +- (-0.001923,0.001923)(4500.000000,1.019289) +- (-0.001931,0.001931)(4550.000000,1.019030) +- (-0.001901,0.001901)(4600.000000,1.018709) +- (-0.001917,0.001917)(4650.000000,1.018495) +- (-0.001810,0.001810)(4700.000000,1.018337) +- (-0.001699,0.001699)(4750.000000,1.017955) +- (-0.001607,0.001607)(4800.000000,1.017658) +- (-0.001541,0.001541)(4850.000000,1.017391) +- (-0.001467,0.001467)(4900.000000,1.017111) +- (-0.001456,0.001456)(4950.000000,1.016921) +- (-0.001417,0.001417)(5000.000000,1.016711) +- (-0.001379,0.001379)
};
\addlegendentry{δ = 0.2}


\addplot [
mark=none,
mark size=1.0pt,
color=col6,
line width=1pt,
opacity=0.7,
%error bars/.cd,
%error bar style={line width=0.2pt},
%y dir=both,
%y explicit
]
coordinates{
(50.000000,3.152155) +- (-0.282843,0.282843)(75.000000,2.754148) +- (-0.282843,0.282843)(100.000000,2.557727) +- (-0.282843,0.282843)(125.000000,2.264309) +- (-0.182768,0.182768)(150.000000,2.199507) +- (-0.205809,0.205809)(175.000000,2.197334) +- (-0.224749,0.224749)(200.000000,2.181027) +- (-0.259015,0.259015)(225.000000,2.082653) +- (-0.205395,0.205395)(250.000000,2.106887) +- (-0.231928,0.231928)(275.000000,2.023690) +- (-0.203258,0.203258)(300.000000,1.966227) +- (-0.194856,0.194856)(325.000000,1.947306) +- (-0.198179,0.198179)(350.000000,1.914576) +- (-0.195643,0.195643)(375.000000,1.899204) +- (-0.196363,0.196363)(400.000000,1.870889) +- (-0.184128,0.184128)(425.000000,1.859716) +- (-0.187494,0.187494)(450.000000,1.866785) +- (-0.198573,0.198573)(475.000000,1.869916) +- (-0.209406,0.209406)(500.000000,1.856565) +- (-0.203838,0.203838)(525.000000,1.848881) +- (-0.200749,0.200749)(550.000000,1.861784) +- (-0.211146,0.211146)(575.000000,1.844786) +- (-0.197391,0.197391)(600.000000,1.827764) +- (-0.189617,0.189617)(625.000000,1.808641) +- (-0.183252,0.183252)(650.000000,1.791222) +- (-0.177802,0.177802)(675.000000,1.775408) +- (-0.172729,0.172729)(700.000000,1.771543) +- (-0.176891,0.176891)(725.000000,1.760737) +- (-0.180336,0.180336)(750.000000,1.754458) +- (-0.185618,0.185618)(775.000000,1.755482) +- (-0.195198,0.195198)(800.000000,1.746463) +- (-0.190171,0.190171)(825.000000,1.733410) +- (-0.187312,0.187312)(850.000000,1.713535) +- (-0.177877,0.177877)(875.000000,1.698082) +- (-0.168968,0.168968)(900.000000,1.689195) +- (-0.166201,0.166201)(925.000000,1.674739) +- (-0.159303,0.159303)(950.000000,1.663527) +- (-0.153797,0.153797)(975.000000,1.652834) +- (-0.150697,0.150697)(1000.000000,1.637354) +- (-0.146396,0.146396)(1000.000000,1.637354) +- (-0.146396,0.146396)(1050.000000,1.618787) +- (-0.148357,0.148357)(1100.000000,1.608466) +- (-0.146028,0.146028)(1150.000000,1.589625) +- (-0.137772,0.137772)(1200.000000,1.582859) +- (-0.137341,0.137341)(1250.000000,1.574530) +- (-0.134197,0.134197)(1300.000000,1.573821) +- (-0.136491,0.136491)(1350.000000,1.573086) +- (-0.137668,0.137668)(1400.000000,1.563656) +- (-0.134711,0.134711)(1450.000000,1.557725) +- (-0.131479,0.131479)(1500.000000,1.551687) +- (-0.128239,0.128239)(1550.000000,1.551219) +- (-0.130042,0.130042)(1600.000000,1.548177) +- (-0.133716,0.133716)(1650.000000,1.552049) +- (-0.138282,0.138282)(1700.000000,1.551850) +- (-0.141420,0.141420)(1750.000000,1.550453) +- (-0.145784,0.145784)(1800.000000,1.543191) +- (-0.145172,0.145172)(1850.000000,1.545773) +- (-0.149570,0.149570)(1900.000000,1.539784) +- (-0.148280,0.148280)(1950.000000,1.538441) +- (-0.151141,0.151141)(2000.000000,1.537655) +- (-0.153289,0.153289)(2050.000000,1.534850) +- (-0.154781,0.154781)(2100.000000,1.524392) +- (-0.152375,0.152375)(2150.000000,1.518895) +- (-0.154622,0.154622)(2200.000000,1.515507) +- (-0.153939,0.153939)(2250.000000,1.515471) +- (-0.159488,0.159488)(2300.000000,1.518744) +- (-0.165445,0.165445)(2350.000000,1.520072) +- (-0.170895,0.170895)(2400.000000,1.514742) +- (-0.169592,0.169592)(2450.000000,1.511923) +- (-0.169082,0.169082)(2500.000000,1.505290) +- (-0.164577,0.164577)(2550.000000,1.500003) +- (-0.161531,0.161531)(2600.000000,1.498721) +- (-0.162107,0.162107)(2650.000000,1.491077) +- (-0.158844,0.158844)(2700.000000,1.486961) +- (-0.158920,0.158920)(2750.000000,1.484456) +- (-0.159887,0.159887)(2800.000000,1.481816) +- (-0.161353,0.161353)(2850.000000,1.475643) +- (-0.159303,0.159303)(2900.000000,1.472410) +- (-0.159387,0.159387)(2950.000000,1.468427) +- (-0.157866,0.157866)(3000.000000,1.462244) +- (-0.156403,0.156403)(3050.000000,1.456562) +- (-0.154198,0.154198)(3100.000000,1.452228) +- (-0.152107,0.152107)(3150.000000,1.446965) +- (-0.149329,0.149329)(3200.000000,1.443102) +- (-0.146369,0.146369)(3250.000000,1.436463) +- (-0.140542,0.140542)(3300.000000,1.430902) +- (-0.136785,0.136785)(3350.000000,1.427831) +- (-0.135342,0.135342)(3400.000000,1.423780) +- (-0.132935,0.132935)(3450.000000,1.418794) +- (-0.130490,0.130490)(3500.000000,1.412899) +- (-0.127559,0.127559)(3550.000000,1.410693) +- (-0.127489,0.127489)(3600.000000,1.408017) +- (-0.127525,0.127525)(3650.000000,1.402495) +- (-0.124917,0.124917)(3700.000000,1.398559) +- (-0.124040,0.124040)(3750.000000,1.394606) +- (-0.122662,0.122662)(3800.000000,1.392826) +- (-0.123008,0.123008)(3850.000000,1.389480) +- (-0.122707,0.122707)(3900.000000,1.384567) +- (-0.120802,0.120802)(3950.000000,1.381225) +- (-0.120036,0.120036)(4000.000000,1.379686) +- (-0.119918,0.119918)(4050.000000,1.376941) +- (-0.119084,0.119084)(4100.000000,1.375287) +- (-0.119564,0.119564)(4150.000000,1.373203) +- (-0.118413,0.118413)(4200.000000,1.372232) +- (-0.118720,0.118720)(4250.000000,1.372016) +- (-0.119508,0.119508)(4300.000000,1.366913) +- (-0.115660,0.115660)(4350.000000,1.363683) +- (-0.113928,0.113928)(4400.000000,1.360521) +- (-0.112120,0.112120)(4450.000000,1.356006) +- (-0.109878,0.109878)(4500.000000,1.352499) +- (-0.108229,0.108229)(4550.000000,1.349609) +- (-0.107535,0.107535)(4600.000000,1.347104) +- (-0.107053,0.107053)(4650.000000,1.344320) +- (-0.106124,0.106124)(4700.000000,1.343126) +- (-0.106539,0.106539)(4750.000000,1.342398) +- (-0.106974,0.106974)(4800.000000,1.340270) +- (-0.106528,0.106528)(4850.000000,1.338350) +- (-0.105868,0.105868)(4900.000000,1.336157) +- (-0.105200,0.105200)(4950.000000,1.335106) +- (-0.105226,0.105226)(5000.000000,1.334493) +- (-0.105864,0.105864)
};
\addlegendentry{δ = 0}

\end{axis}
\end{tikzpicture}

    \vspace{\scspacey}
    \caption{\hspace{\scspacex}\textsc{Water}}
    \label{fig:water3}
  \end{subfigure}
  \caption{
    (a) Increasing the number of mixture components improves performance.
    (b) The combination of Gibbs and \Ms{} performs better than either of them does individually.
  }
  \label{fig:exp2}
\end{figure*}

We now evaluate the performance of our proposed sampler on the Ising model we analyzed earlier, as well as the following three models learned from real-world data sets.
\paragraph{\textsc{Water}.} A (log-submodular) facility location model, which was used in a problem of sensor placement in a water distribution network \citep{krause08}.
The function $F$ is of the form
\begin{align*}
F(S) = \sum_{j = 1}^L \max_{i \in S}c_{ij}.
\end{align*}
We randomly subsample the original facility location matrix $C = (c_{ij})$, so that $n = 50$, and $L = 500$.
\paragraph{\textsc{Sensor}.} A (log-submodular) determinantal point process \citep{kulesza12}, which was used in a problem of sensor placement for indoor temperature monitoring \citep{guestrin05}.
The function $F$ is of the form
\begin{align*}
F(S) = \log |K + \sigma^2 I|,
\end{align*}
where $K$ is a kernel matrix, and $\sigma$ is a noise parameter.
The size of the ground set is $n = 46$.
\paragraph{\textsc{Game}.} A (log-submodular) facility location diversity model \citep{tschiatschek16}, which represents the characters that are chosen by players in the popular online game ``Heroes of the Storm''.
We learned the model from an online data set of approximately $8,000$ teams of $5$ characters\footnote{https://www.hotslogs.com} using noise-contrastive estimation, as described by \cite{tschiatschek16}.
The function $F$ is of the form
\begin{align*}
F(S) = \sum_{v \in S}w_v + \sum_{j = 1}^L \max_{i \in S}c_{ij},
\end{align*}
with $n = 48$, and $L = 10$.
In practice, we would only be interested in sampling sets of fixed size $\ell = 5$.
The Gibbs sampler can be easily modified to sample under a cardinality constraint by using moves that swap an element in the current set $X_t$ with an element in $V \setminus X_t$.
Extending the \Ms{} chain to sample from cardinality-constrained models is also straightforward.
In fact, the only additional ingredient required is a procedure to sample a set of size $\ell$ from a log-modular distribution, which can be easily done, as before, in $\bO(n)$ time.

In what follows, we compare the performance of the Gibbs sampler (\textsc{Gibbs}) against our proposed combined sampler using a proposal mixture $q$ constructed by \algoref{alg:mixture} (\textsc{Combo-I}).
We also compare against a variation where we substitute the greedy procedure in \lineref{lin:perm} of \algoref{alg:mixture} with picking a permutation $\sigma$ of the ground set uniformly at random (\textsc{Combo-R}).

To assess convergence we use the potential scale reduction factor (PSRF) \citep{brooks11} using $20$ parallel chains.
We compute the PSRF using single-element marginal probabilities averaged over $50$ repetitions of each simulation.

In \figsref{fig:ising6}--\ref{fig:ising8} we show the results for the Ising model ($n = 6, 7, 8$) with the additional \textsc{Combo-f} line denoting the combined sampler with two mixture components described in \sectref{sect:ising}.
The other two combined samplers use mixtures of size $r = 20$.
Note that Gibbs mixes dramatically slower than the combined sampler, even for such small $n$.

In \figsref{fig:water1}--\ref{fig:hots1} we show the results on the three log-submodular models described before using mixtures of size $r = 200$.
It is interesting to see that even random permutations are enough to significantly improve over the performance of Gibbs.
Similar observations can be made with respect to computation time, as shown in \figsref{fig:water1-time}--\ref{fig:hots1-time}, which measure wall-clock time on the $x$-axis.

In \figref{fig:water2} we show how mixture size affects performance; as expected, adding more components to the mixture results in a proposal that approximates the target distribution better, and, therefore, mixes faster.
Finally, in \figref{fig:water3} we see that both Gibbs ($\alpha = 1$) and \Ms{} ($\alpha = 0$, $r = 200$) perform poorly by themselves, but combining them results in much improved performance.
This highlights again the complementary nature of the two chains (local vs. global moves) we discussed earlier.

\section{Conclusion}
We considered the problem of sampling from general discrete probabilistic models, and presented the \Ms{} sampler that proposes global moves using a mixture of log-modular distributions.
We theoretically analyzed the effect of combining our sampler with the Gibbs sampler on a class of Ising models, and proved an exponential improvement in mixing time.
We also demonstrated notable improvements when combining the two samplers on three models of practical interest.
We believe that our work represents a step towards moving beyond local samplers, and incorporating ideas from optimization, such as semigradients, into probabilistic inference.